% !TEX program = pdflatex

\documentclass{lnotes}

% Packages
% \usepackage[osf]{libertineRoman}
\usepackage{epigraph}
\usepackage{tikz-cd}

% Customization
\let\mathbb\mathbf
\def\Hom{\operatorname{Hom}}

% Metadata
\title{Fields and Galois Theory}
\course{\textsc{math} 370}
\place{yale university}
\term{spring}
\year{2019}
\blurb{
	These are lecture notes for \textsc{math} 370b, ``Fields and Galois Theory,'' taught by Asher Auel at Yale University during the spring of 2019.
	These notes are not official, and have not been proofread by the instructor for the course.
	They live in my lecture notes respository at 
	\[\text{\url{https://github.com/jopetty/lecture-notes/tree/master/MATH-370}.}\] 
	If you find any errors, please open a bug report describing the error and label it with the course identifier, or open a pull request so I can correct it.
}

% Bibliography
\usepackage[
	style = alphabetic,
]{biblatex}
\addbibresource{references.bib}

\begin{document}

\section*{Syllabus}

\begin{tabularx}{\textwidth}{rX}
\toprule
\textbf{Instructor} & Asher Auel, \url{asher.auel@yale.edu} \\
\textbf{Lecture} & \textsc{tr} 11:35 \textsc{am} -- 12:50 \textsc{pm} in \textsc{lom} 215 \\
\textbf{Peer Tutor} & Arthur Azvolinsky, \url{arthur.azvolinsky@yale.edu} \\ 
\textbf{Section} & Math Lounge \\
\textbf{Exams} & Midterm 1: Febuary 19; \quad Midterm 2: April 9; \quad Final: May 3. \\
\textbf{Textbook} & \fullcite{textbook} \\
\bottomrule
\end{tabularx} \\

The main object of study in Galois theory are roots of single variable polynomials.
Many ancient civilizations (Babylonian, Egyptian, Greek, Chinese, Indian, Persian) knew about the importance of solving quadratic equations.
Today, most middle schoolers memorize the ``quadratic formula'' by heart.
While various incomplete methods for solving cubic equations were developed in the ancient world, a general ``cubic formula'' (as well as a ``quartic formula'') was not known until the 16th century Italian school.
It was conjectured by Gauss, and nearly proven by Ruffini, and then finally by Abel, that the roots of the general quintic polynomial could not be solvable in terms of nested roots.
Galois theory provides a satisfactory explanation for this, as well as to the unsolvability (proved independently in the 19th century) of several classical problems concerning compass and straight-edge constructions (e.g., trisecting the angle, doubling the cube, squaring the circle).
More generally, Galois theory is all about symmetries of the roots of polynomials.
An essential concept is the field extension generated by the roots of a polynomial.
The philosophy of Galois theory has also impacted other branches of higher mathematics (Lie groups, topology, number theory, algebraic geometry, differential equations).

This course will provide a rigorous proof-based modern treatment of the main results of field theory and Galois theory.
The main topics covered will be irreducibility of polynomials, Gauss's lemma, field extensions, minimal polynomials, separability, field automorphisms, Galois groups and correspondence, constructions with ruler and straight-edge, theory of finite fields.
The grading in Math 370 is very focused on precision and correct details.
Problem sets will consist of a mix of computational and proof-based problems.

Your final grade for the course will be determined by
\[ \max\left\{
	\begin{array}{cccc}
		\text{20\% homework} + \text{25\% midterm 1} + \text{25\% midterm 2} + \text{30\% final} \\
		\text{20\% homework} + \text{25\% midterm 1} + \text{15\% midterm 2} + \text{40\% final} \\
		\text{20\% homework} + \text{15\% midterm 1} + \text{25\% midterm 2} + \text{40\% final}
	\end{array}
\right\}. \]

\printbibliography

\section{Lecture 1}

Didn't attend class.
\section{Lecture 2}

Didn't attend class.
% !TEX root = ../notes.tex

\section{Monday, January 28}


% !TEX root = ../notes.tex

\section{Wednesday, January 13}

\begin{multicols}{2}
\begin{enumerate}
\item Cogito ergo sum
\item Ambulo ergo sum
\item Transparency
\item Cogitatur
\item Quid igitur sum?
\item Sum res cogitant
\item The essence of the mind
\item Truth rule
\item Ugly head
\item The great chain of being
\item Degrees of reality
\end{enumerate}
\end{multicols}

Doubt presupposes existence, so the existence of doubt is an argument for the existence of `I.' Something like \emph{walking} wouldn't work since we can only be certain that we \emph{think} that we're walking, but we can be sure that we are thinking. In this way, Descartes believes that he is certain of the content of his own mind. This certainty opens a gap between the mind and the world, everything the mind percieves. The mind is transparent to itself, however.

\begin{problem}
How can Descartes be sure that the `I' which thinks is preserved between thoughts? Likewise, how can't we really only be sure that thought is going on?
\[ \text{thinking exists} \implies \text{there is a thinking thing} \implies \text{I am that thinking thing} \]
Descartes first argues that thought must have a bearer in some substance, and then argues that there is an inherent quality of self about thoughts (you can't be aware of other people's thoughts, so any thoughts one percieve must be one's own). \note{How does Descartes address the idea that the I which thinks each thought might be different each time, just with different false memories of other beings? Kant considers this, but Descartes likely views this as an implicit part of a thinking substance.}
\end{problem}

Descartes makes some claims about the nature of a thinking mind. Here are some possible claims one could make about the essence of the mind/self: The first two claims are addressed in the second meditation, the third is addressed in the sixth meditation.
\begin{enumerate}
\item I am essentially (necessarily) a thinking thing; Descartes said ``A thought alone is not seperable from me.''
\item I am not essentially (necessarily) extended;
\item I am not an extended thing; \note{What exactly does \emph{extended} mean?}
\item I am not identical to my body;
\end{enumerate}

\subsection{The Truth Rule}

Descartes reflects on his certainty that he is a thinking thing, and seeks to extrapolate to understand what is a sufficient condition for being sure of something. He ends up with \emph{whatever I perceive very clearly and distinctly is true.} What constitutes clarity and distinctness is not really something Descartes addresses. Descartes also worries that a deceiving god could construct a clear and distinct argument which is actually false, and we could not be sure that we do not fall victim to our own assumtions about this claim. Thus Descartes seeks to first establish that God exists and that God is not a deceiver, and then the truth rule can necessarily follow. If he cannot then even the \emph{cogito ergo sum} argument is called into doubt once again. The proof Descartes proposed is almost universally taken to be wrong (but that doesn't mean that it's not valuable or not worth studying).

\subsection{Descartes' Proof of God's Existence}

There are many different possible kinds of beings which exist in the world, and they have different degrees of reality. All beings are made of substance, meaning that they have properties but are not themselves properties of anything else. \note{Need substances be material? Is the mind a substance?} Furthermore, substances have a greater degree of reality than their properties, states, or modes. This is because the properties of a substance depend on the substance yet the substance doesn't depend on the properties: You can have a table which isn't rectangular, but you can't just have `rectangular' without it being a rectangular \emph{something}. The highest degree of being would be an infinite substance since it would be independent of anything else, uncreated and permanent. Below this are created, finite substances like a table or a mind. Below even this are the properties of finite substances. Since God is superlative, if God exists then he must be an infinite substance. The kind of reality which substances have is called \emph{formal reality}, reality by the virtue of existance. This is contrasted with objective reality.
\include{Lectures/5}
% !TEX root = ../notes.tex

\section{Thursday, January 31}

\epigraph{``I went to the pound since Tuesday's class and bought a stray cat. Brought it home, doused it in gasoline, and lit it on fire while it shreiked and died an agonizing death....and that seems wrong.''}{Shelly}

The results of the experience machine experiment tell us that most people doen't really buy Hedonism --- there's more to well-being than just the experience. There are many different theories which propose ideas about what the missing pieces are. One alternative is that we must have the things we want, and when we accomplish this our life is better than when we don't. This theory explains why the experience machine fails; we want the real things, not merely the experience. These Preference Theories are the in-house philosophy of the department of economics.

It also seems like it ought to matter how welfare is aggregated. In some cases this is easy; if everybody is better off in world $A$ than in world $B$ its clear than $A$ presents a better existence. But it's often not so clear cut. Perhaps we ought to use the maximum happiness in the world, or perhaps we should use the average amount of happiness. There are scenarios which make either look bad. This has been going on for forty years and philosophers still don't have a consensus. These theories all require us to have interpersonal comparisons of happiness or welfare, which may not even be possible.

\begin{problem}
Do animals count in the welfare aggregation?
\end{problem}
% !TEX root = ../notes.tex

\section{Tuesday, February 5}

\subsection{Ruler and Compass Construction}

Euclid had this book which tried to lay the foundations for geometry. Among these are \emph{constructions}.

\begin{enumerate}
\item You have a set $S, \mathcal{P} \subseteq \C$ which contain $0$ and $1$.
\item Given any two points $p,q$ in $S$, you can draw a line $\overline{PQ}$ through $p$ and $q$.
\item Given any two points $p,q$ in $S$ you can draw the circle $C_p(q)$ centered at $p$ going through $q$.
\item Any intersections of lines or circles which can be drawn are now points in $\mathcal{P}$.
\end{enumerate}

We call $\mathcal{P} \subseteq \C$ the set of constructable numbers which contains $0,1$ and is closed under this method of constructions. We can also create this idea algorithmicly. We proceed with induction for $n \geq 0$. Let $\mathcal{P}_0 = \{0,1\}$ and let $\mathcal{L}_0 = \mathcal{C}_0 = \emptyset$. Let $\mathcal{L}_{n+1} = \{\ell_{p,q} \mid p,q \in \mathcal{P}_n\}$ and let $\mathcal{C}_{n+1} = \{C_p(q) \mid p,q \in \mathcal{P}_{n}\}$. Then \[\mathcal{P}_{n+1} = \{z \in \C \mid z \in L \cap L', C \cap L, C \cap C', L,L' \in \mathcal{L}_{n+1},C,C' \in \mathcal{C}_{n+1}\}.\]
Hence $\mathcal{P}_0 \subseteq \mathcal{P}_1 \subseteq \cdots \subseteq \mathcal{P}$, and each $\mathcal{P}_n$ are finite. Then $\bigcup_{n \geq 0} \mathcal{P}_n = \varinjlim \mathcal{P}_n = \mathcal{P}$. Then $\mathcal{P}$ is countable.

\begin{theorem}
The set $\mathcal{P} \subset \C$ is a field called Pythagorean closure of $\Q$ in $\Q$.
\end{theorem}

\subsection{Some Basic Constructions}

\begin{enumerate}
\item Given a `line segment' $\overline{PQ}$ we can bisect it by drawing the circles $C_p(q)$ and $C_q(p)$ and then drawing the line connecting the intersections of these circles.

\item Given a line $\ell$ and a $p \in \ell$ we can draw the perpendicular line to $\ell$ at $p$. Draw a circle $C_p(q)$ where $q$ is any other point on $\ell$. Then draw two circles of twice the radius centered at the intersections of the smaller circle with $\ell$, and then connect their intersections.

\item Give a line $\ell$ and a point $p \notin \ell$ we can draw the parallel line to $\ell$ through $p$. Draw a circle centered at $p$ intersecting $\ell$. Draw the perpendicular bisector of $\ell$ through $p$.
\end{enumerate}
% !TEX root = ../notes.tex

\section{Thusrday, February 7}

\epigraph{``They're graduate students in philosophy, they won't contribute \emph{anything} to society.''}{Shelly}

\subsection*{Does Utilitarianism Give Plausible Answers?}

\begin{problem}
Why is murder wrong?

\begin{solution}[Utilitarianism Response]
Because you had other options, and the option to murder someone decreased happiness/welfare.
\end{solution}
\end{problem}

\begin{problem}
Should we steal from the blind man in the subway?

\begin{solution}[Utilitarianism Response]
No.
\end{solution}
\end{problem}

\begin{problem}
We are sailing a boat when we see someone drowning; we turn to rescue them, but then we notice five people drowning somewhere else. We don't have time to save everyone. What do we do?

\begin{solution}[Utilitarianism Response]
It's better to save five than to save one.
\end{solution}

What if we tweak the scenario? What if the one was about the find the cure for cancer, and the five were all graduate students in philosophy. What about the impacts of what these people contribute to society?

\begin{solution}[Utilitarianism Response]
Those things matter. We can't say that everything is equal, as before.
\end{solution}
\end{problem}

What if we aren't sure of the outcomes? We can't know everything, after all. What should a utilitarianism (or a consequentialist) do in light of this? What if you accidentally save Hitler from drowning?

\begin{example}
There's a distinction here in what we're asking: Did you do the right thing or are you a morally good person? Utilitarianism and consequentialism are concerned with actions, and so in and of themselves do not concern themselves with whether or not people are good. We can develop a utilitarian theory of this, but it is supplemental to the core of the philosophy.
\end{example}
% !TEX root = ../notes.tex

\section{No Notes}
% !TEX root = ../notes.tex

\section{Wednesday, 23 February}

Recall the question of when is $(a/p) = \pm1$ when $a\in\Z$, $(a,p) = 1$, and $p$ is prime. We defined $\abs{x} = \min\{x, p-x\}$ for $0 \leq x \leq p-1$. Recall Gauss' Lemma, which states

\begin{lemma}[Gauss]
Let $s$ be the number of $\ell$ such that $1 \leq \ell \leq (p-1)/2$ such that $\abs{a\ell} = -a\ell$. Then $(a/p) = (-1)^s$.
\end{lemma}

\begin{example}
Let $p = 17$ and let $a = 2$. Then $\{a\ell\}$ is $\{2, 4, 6, 8, 10, 12, 14, 16\}$. We see that $(p+1)/2 = 9$. Then $s=4$ and so $(2/p) = 1$.
\end{example}

\begin{theorem}
Let $p$ be an odd prime. Then 
\[
	\qty(\frac{2}{p}) = \begin{cases}
		1 & p \cong_8 \pm 1, \\
		-1 & p \cong_8 \pm 3.
	\end{cases}
\]
\end{theorem}

\begin{proof}
We just need to look at how many even numbers are between $1$ and $(p-1)/2$ verses $(p+1)/2$ and $p$.
\begin{enumerate}
\item[\textbf{Case 1:}] $p = 1 + 8k$. Then $(p-1)/2 = 4k$, so $s = 2k$ and $(2/p) = (-1)^{2k} = 1$.
\item[\textbf{Case 3:}] $p = 3 + 8k$. Then $(p-1)/2 = 4k + 1$, so there are $2k$ even numbers ebtween $1$ and $(p-1)/2$ and $2k+1$ even numbers between $(p+1)/2$ and $p$, so $(2/p) = -1$. \qedhere
\end{enumerate}
\end{proof}

What about $(p/q)$ when $p$ and $q$ are odd primes.

\begin{theorem}[Quadratic Reciprocity]
If $p \cong_4 1$ or $q \cong_4 1$ then $(p/q) = (q/p)$. Otherwise, if $p \cong_4 q \cong_4 3$ then $(p/q) = -(q/p)$. Equivalently,
\[ \qty(\frac{p}{q})\qty(\frac{q}{p}) = (-1)^{(p-1)/2 \cdot (q-1)/2} \]
\end{theorem}

\begin{proof}
Consider $f(z) = 2i\sin(2\pi z)$, which has some nice properties. It's odd, so $-f(z) = f(-z)$. It's also $1$-periodic, so $f(z) = f(z+1)$. Note that $i \sin(z) = \sinh(iz)$, so 
\[ f(z) = e^{2\pi i z} - e^{-2\pi i z}. \]
Define $\zeta = \zeta_p$ to be the $p$\textsuperscript{th} root of unity $2^{2\pi i/p}$. Note that $\zeta^m \cdot \zeta^n = \zeta^{m+n \mod p}$ and $\qty(\zeta^m)^\ell = \zeta^{m\ell \mod p}$

We also have the following Proposition and Lemma, listed after the proof.

Now let $p,q$ be odd primes. Then 
\[ \qty(\frac{q}{p}) = \prod_{\ell=1}^{(p-1)/2} \frac{f(q\ell/p)}{f(\ell/p)} = \prod_{\ell=1}^{(p-1)/2}\prod_k^{(q-1)/2} f\qty(\frac{\ell}{p} + \frac{k}{q})f\qty(\frac{\ell}{p} - \frac{k}{q}). \]
Notice how this expression is almost symmetric in $p$ and $q$, with only one difference in the final term. In fact, switching them out only requires $(-1)^{(p-1)(q-1)/2}$.
\end{proof}

\begin{proposition}\label{prop:periodicity}
Consider that
\[ \prod_{\ell = 1}^{(p-1)/2} f\qty(\frac{a\ell}{p}) = \qty(\frac{a}{p}) \cdot  \prod_{\ell = 1}^{(p-1)/2} f\qty(\frac{\ell}{p}).\]
The justification comes from the periodicity of $f$.
\end{proposition}

\begin{proof}
Consider that if $1/2 < a\ell/p < 1$ then 
\[ f\qty(\frac{a\ell}{p}) = -f\qty(\frac{\abs{a\ell}}{p}). \]
Then 
\[ \prod_{\ell = 1}^{(p-1)/2} f\qty(\frac{a\ell}{p}) = (-1)^s \prod_{\ell = 1}^{(p-1)/2} f\qty(\frac{\abs{a\ell}}{p}) = \qty(\frac{a}{p})\prod_{\ell = 1}^{(p-1)/2} f\qty(\frac{\abs{a\ell}}{p}). \]
Then recall that the sequence $\{\abs{a}, \abs{2a}, \dotsc, \abs{(p-1)a/2}\}$ is just $\{1, 2, \dotsc, (p-1)/2\}$, which gets us that 
\[ \qty(\frac{a}{p})\prod_{\ell = 1}^{(p-1)/2} f\qty(\frac{\abs{a\ell}}{p}) = \qty(\frac{a}{p})\prod_{\ell = 1}^{(p-1)/2} f\qty(\frac{\ell}{p}). \qedhere \]
\end{proof}

\begin{lemma}
If $n$ is odd then \[
x^n-y^n = \prod_{k=0}^n \qty(x\zeta^k - y\zeta^{-k}),
\]
where $\zeta = \zeta_n$.
\end{lemma}
\begin{proof}
It suffices to look at $(x/y)^n - 1 = z^n - 1$, which factors as 
\[ z^n - 1 = \prod_{k=0}^{n-1} \qty(z-\zeta^k) = \prod_{k=0}^{n-1} \qty(z-\zeta^{-2k}). \]
Then 
\[ x^n - y^n = \prod \qty(x-y\zeta^{-2k}) = \qty[\prod \qty(x\zeta^k - y\zeta^{-k})] \zeta^{-(n-1)/2}, \]
where $\zeta^{-(n-1)/2} = 1$ since $n$ is odd.
\end{proof}

\begin{lemma}
The value
\[ \frac{f(nz)}{f(z)} = \prod_{k=1}^{(n-1)/2} f\qty(z + \frac{k}{n})f\qty(z - \frac{k}{n}) \]
if $n$ is odd.
\end{lemma}
\begin{proof}
Notice that $f(nz) = e^{2\pi i zn} - e^{-2\pi i zn}$. Now just apply the previous lemma.
\end{proof}

\subsection{Applications of QR}
\pagebreak
\begin{example}[We can use it to compute Legendre Symbols!]
Compute
\[ \qty(\frac{713}{1009}). \]
We factor as 
\[ \qty(\frac{23 \cdot 31}{1009}) = \qty(\frac{23}{1009})\qty(\frac{31}{1009}) = \qty(\frac{1009}{23})\qty(\frac{1009}{31}) = \qty(\frac{20}{23})\qty(\frac{17}{31}). \]
\end{example}

\begin{definition}[Jacobi Symbol]
Let $(q/n)$ be the Jacobi symbol, where $n$ is a product of primes, then it is multiplicative.
\[
\qty(\frac{q}{n}) = \begin{cases}
0 & (a,n) \not= 1, \\
\prod \qty(\frac{a}{p_i})^{1/i} & (a,n) = 1.
\end{cases}
\]
\end{definition}

\begin{theorem}[Jacobi Reciprocity]
Some facts:
\begin{itemize}
\item $(-1/n) = (-1)^{(n-1)/2}$
\item $(2/n) = (-1)^{(n^2-1)/8}$
\item If $m,n$ odd then $(m/n)(n/m) = (-1)^{(m-1)(n-1)/4}$
\end{itemize}
\end{theorem}

\subsubsection*{Next Class}
\begin{itemize}
\item Use Jacobi to talk about when $a$ is a quadratic residue for almost all primes
\item RSA/Diffie-Hellman, Zero Knowledge Proofs
\end{itemize}
% !TEX root = ../notes.tex

\section{Monday, 18 February}

Recall the definition of the Jacobi Symbol. This is like the multiplicative extension of the  Legendre Symbol, although we loose the nice property that $(a/n) = 1$ if and only if $a$ is a quadratic residue modulo $n$. We do that the following properties:
\[ \qty(\frac{a}{n})\qty(\frac{b}{n}) = \qty(\frac{ab}{n}) \qquad \text{and} \qquad \qty(\frac{a}{n})\qty(\frac{a}{\ell}) = \qty(\frac{a}{n\ell}). \]
Also recall the theorem of Jacobi Reciprocity, restated here:
\begin{theorem}[Jacobi Reciprocity]
Some facts:
\begin{itemize}
\item $(-1/n) = (-1)^{(n-1)/2}$
\item $(2/n) = (-1)^{(n^2-1)/8}$
\item If $m,n$ odd then $(m/n)(n/m) = (-1)^{(m-1)(n-1)/4}$
\end{itemize}
\end{theorem}

\begin{theorem}
If $a$ is a non-square, there are infinitely many primes such that\/ $(a/p) = -1$, that is where $a$ is not a residue modulo $p$.
\end{theorem}

\begin{proof}
Assume that $a = 2^e \cdot \prod q_i$, where $q_i$ are distinct primes and $e \in \{0,1\}$. We assume here that $a$ is square free, since we can always reduce the exponents modulo $2$ to get rid of this square part. Fix any set of primes $\ell_1, \dotsc, \ell_k$ distinct from $2,q_i$. We want to show that there is a prime $p$ not in this list such that $(a/p) = -1$. We do this by building such a number. By CRT we know there is a $x$ such that $x \cong_8 1 \cong_{\ell_i} 1 \cong_{q_{i < m}} 1 \cong_{q_m} s$, where $(s/q_m) = -1$. Consider that 
\[ \qty(\frac{a}{x}) = \qty(\frac{2^e}{x}) \prod \qty(\frac{q_i}{x}) = 1 \cdot \prod \qty(\frac{x}{q_i}) \cdot (-1)^{(x-1)/2 \cdot (q-1)/2} = \prod \qty(\frac{x}{q_i}). \]
Now, since $x \cong_{q_{i<m}} 1$, we get that 
\[ \prod \qty(\frac{x}{q_i}) = 1^{m-1} \qty(\frac{s}{q_m}) = -1. \]
Then we use the multiplicative nature of the Jacobi symbol to say that 
\[ \qty(\frac{a}{x}) = -1 = \prod \qty(\frac{a}{p_i}) \qquad \text{where~} x = \prod p_i^{v_i}, \]
and we know that $p_i \not= q_i$ since otherwise its congruence modulo $q_i$ would be zero. Since we already know this equals $-1$, there must be \emph{some} (at least one) $p_i$ such that $(a/p_i) = -1$. This is really similar to Euclid's proof of the infinitude of primes.
\end{proof}

Note: The above assumes that $a \not= 2$ since we implicitly assumed there was at least one odd prime factor. If $a = 2$, then it is a nonresidue if and only if $p \cong_8 3,5$. There are infinitely many primes $p \cong_8 3$.

\subsection{RSA Cryptography}

RSA is a public-key cryptography system. Historically, crypto systems has the same encoding and decoding key. An example is something like a cryptogram, like \texttt{XYQ ABCX}, where each letter is a swap for another letter in the alphabet, so \texttt{XYQ ABCX} $\to$ \texttt{THE BOAT}. If you know the bijection $f\colon A \to A$ then you can both encode and decode the message.

\parshape=1
0.1\textwidth 0.8\textwidth
Fun fact, \texttt{BEBOPBOP} is a valid cryptogram for exactly one English word.

Another example is Enigma (yay Alan Turing) from WWII, where the ability to read or send the messages was dependent on a (very high) number of possible dial-combinations, which made it easy to use but computationally difficult to break.

Public Key cryptography sets up a system where anyone can encrypt a message for Alice, but only she may decrypt such a message. This is accomplished by having two different keys. In private, Alice will pick two primes $p,q$ and publicly announces their product $n = pq$. Privately, she can compute $\varphi(n) = (p-1)(q-1)$. She the picks an encryption exponent $e$ and announces this too, and then privately computes $d = e^{-1} \bmod \varphi(n)$. Let's say that Bob wants to send a message to Alice. Suppose this message is some number $P$ between $2$ and $n$ ($1$ fails for obvious reasons). Bob takes $P$ and encrypts is via $C = P^e$ and sends $C$ to Alice. When she receives it, she takes $C^d = P^{e^{d}} \cong P^{e \cdot e^{-1}} \bmod \varphi(n) = P$. If Eve is looking in on this transmission, she can see $C = P^e$ and she even knows what $e$ is! However, given a composite number $n$, it is computationally easy to compute $P^e \bmod n$, but it is nearly intractable to find $P$ given $P^e$. This means that Alice can't really decrypt the message by brute force. Nor can she compute $\varphi(n)$, since it is also very hard to compute $\varphi(n)$ from $n$; we believe it to be as hard as factoring $n$, which is not easy to do\footnote{We think this is the case.}.

\subsection{Diffie-Hellman Key Exchange}

Suppose that Alice has a secret she needs to share with Bob. They don't care what the secret is, but they both need to know it (like an Enigma Key!). It would be bad for Alice to announce it publicly, since anyone could hear it and it's not a secret anymore. One better method is for Alice to take the secret and lock it in a box and send it to Bob. Bob can't open it, but neither can anyone else. However, Bob \emph{can}\/ add his own lock to the box, and send it back to Alice. She then unlocks her lock, and sends the box back to Bob, who can not unlock the last remaining lock and read the secret message without anyone else having read it. This is (more or less) how Diffie-Hellman Key Exchange works.

Alice and Bob agree publicly on a public prime $p$, and some primitive root-ish\footnote{This might be hard to find, so we can find something with a large enough order and just go with that.} $r$. Now they will each privately choose keys $k_A$ and $k_B$, which they don't reveal to anyone. Alice then transmits $c_A = r^{k_A} \bmod p$ and Bob transmits $c_B = r^{k_B} \bmod p$. Alice takes $c_B^{k_A} = r^{k_Ak_B} \bmod p$ and Bob takes $c_A^{k_B} = r^{k_Ak_B} \bmod p$. This is their shared secret. Note however that the secret they end up with $r^{k_Ak_B} \bmod p$ is different than what they started with it, but they both end up with a shared secret.

\subsection{Zero-Knowledge Proofs}

Suppose that Paula knows something (this is good) and she wants to prove that she knows it, but doesn't want to reveal the knowledge. A \emph{Zero-Knowledge Proof} is a protocol whereby she may interact with Vince, the verifier, that she knows this secret.

\begin{example}
Imagine that Vince is color-blind, and cannot tell red from green. Paula has a red sock and a greens sock. When Vince sees these, he can't tell which is which, but Paula wants to prove to Vince that she can distinguish between them. To do this, Paula hands both socks to Vince. In each round, 
\begin{itemize}
\item Vince will produce a sock (he doesn't know which one), shows it to Paula, and puts it behind his back, and then produces a second sock (either $S_1$ or $S_2$) and then asks Paula whether or not it is the same sock. 
\item Paula answers him each time. If she couldn't tell the difference, she would have to guess which sock it was, and so in total she fails with probability $1-0.5^n$ after $n$ rounds. If, however, she \emph{does} see the difference then Vince is confident that she does so with the same probability.
\end{itemize}
Thus Vince can be as sure as he wants to be that Paula can see the colors without every actually learning which sock is which.
\end{example}

\begin{example}
Paula wants to prove her identity to the world. She picks primes $p,q,u$ in private, and announces to the world ``I am Paula! $n=pq$ and $v=u^2$.'' Then Paula is anyone who knows $\sqrt{v} = u$ without showing what $u$ is. To do so, she will
\begin{enumerate}
\item Pick an $r$ at random and sends $x=r^2 \bmod n$ to Vince.
\item Vince receives $x$ and flips a coin. If it is heads, Vince asks, ``Send me $r$.'' If it is tails, Vince asks, ``Send me $r^{-1} \cdot u \bmod n$''.
\end{enumerate}
Paula answers with $A$. Vince verifies that Paula is telling the truth. In the `heads' regime, Vince checks if $A^2 = x \bmod n$. In the `tails' regime, he checks if $A^2x = v \bmod n$.
\end{example}
% !TEX root = ../notes.tex

\section{Monday, 25 February 2019}
\section{No Notes}
% !TEX root = ../notes.tex

\section{Tuesday, 5 March 2019}

\epigraph{``Let's do more boring stuff.''}{Asher}

Recall splitting fields. Let $K$ be the splitting field of $x^3 - 2$. We showed that $K = \Q(\sqrt[3]{2}, \omega)$ with the following structure.

\begin{tikzcd}
                                                & K \arrow[ldd, "3" description, no head] \arrow[rd, "2" description, no head] \arrow[rrd, "2" description, no head] \arrow[rrrd, "2" description, no head] &                                                         &                                                                &                                                                       \\
                                                &                                                                                                                                                           & {\Q(\sqrt[3]{2})} \arrow[ldd, "3" description, no head] & {\Q(\sqrt[3]{2}\omega)} \arrow[lldd, "3" description, no head] & {\Q(\sqrt[3]{2}\bar{\omega})} \arrow[llldd, "3" description, no head] \\
\Q(\omega) \arrow[rd, "2" description, no head] &                                                                                                                                                           &                                                         &                                                                &                                                                       \\
                                                & \Q                                                                                                                                                        &                                                         &                                                                &                                                                      
\end{tikzcd}

We know that $\#\Aut_\Q(K) \leq 6$ since any $\phi\colon K \to K$ is determined by the three choices for $\phi(\sqrt[3]{2})$ and the two choices for $\phi(\omega)$.

\paragraph*{Fact:} $K/\Q(\sqrt[3]{2}\omega^2)$ is a degree two extension.

Consider $\sigma_i\colon K \to K$ where $\sigma\colon \omega \mapsto \bar{\omega}$. Then we get the following possible automorphisms:

\begin{align*}
	\sigma_0\colon &= \sqrt[3]{2} \mapsto \sqrt[3]{2}, \omega \mapsto \bar{\omega} \\
	\sigma_1\colon &= \sqrt[3]{2}\omega \mapsto \sqrt[3]{2}\omega, \omega \mapsto \bar{\omega} \\
	\sigma_2\colon &= \sqrt[3]{2}\bar{\omega} \mapsto \sqrt[3]{2}\bar{\omega}, \omega \mapsto \bar{\omega}
\end{align*}

Now consider 
\begin{align*}
	\rho = \sigma_2\sigma_0\colon &K \longrightarrow K \\
	                              & \sqrt[3]{2} \mapsto \sqrt[3]{2} \mapsto \sqrt[3]{2}\omega \\
	                              & \omega \mapsto \bar{\omega} \mapsto \omega
\end{align*}
and
\begin{align*}
	\tau = \sigma_1\sigma_0\colon &K \longrightarrow K \\
	                              & \sqrt[3]{2} \mapsto \sqrt[3]{2} \mapsto \sqrt[3]{2}\bar{\omega} \\
	                              & \omega \mapsto \bar{\omega} \mapsto \omega
\end{align*}

Then we've found six automorphisms $\{\operatorname{id}_K, \sigma_0, \sigma_1, \sigma_2, \rho, \tau\}$. Since our upper bound was six, we've found all possible automorphisms of $\Aut_\Q(K)$. Furthermore, note that every $\sigma_i$ has order $2$. Now we wan't to find some relations. Note that $\rho^2 = \tau$ and $\rho^3 = \operatorname{id}_K$. From this, we can tell that $\Aut_\Q(K) \isom D_6 \isom S_3$.

Remember that $\Aut_\Q(K)$ permutes the roots of $x^3 - 2$. This is a general phenomenon. If $f$ is a polynomial over $F$ and $E$ is the splitting field of $f$ over $F$ then $\Gal(f) = \Aut_F(E)$. Assume $f$ is seperable of degree $n$ with roots $\{\alpha_i\}$ and so $E = F(\{\alpha_i\})$. Then $\Gal(f)$ acts on $\{\alpha_i\}$ and $\Gal(f)$ acts on $E$ by $F$-automorphism. The action of $\Gal(f)$ on $\{\alpha_i\}$ is faithful, so the kernel of the action is trivial.

\begin{proof}
If $\sigma \in \Gal(f)$ fixes all $\alpha_i$ then $\sigma = \operatorname{id}$ on $E$ since $\alpha_i$ generate $E$ over $F$.
\end{proof}

Recall if a group $G$ acts on a set $X$ of order $h$ then there is an induced permutaiton representation 
\[ \rho\colon G \mapsto S_X = \operatorname{Bij}(x) \isom S_n : g \mapsto (x \mapsto g \cdot x), \]
and so $\rho$ is a homomorphism of groups. This $\rho$ is injective if and only if $G$ acts faithfully on $X$.

\begin{corollary}
If $f$ is seperable of degree $n$ over $F$ then $\Gal(f)$ is isomorphic to a subgroup of $S_n$. We will soon prove that $[E : F] = \#\Gal(f)$, which gives immediately that $[E:F] \mid n!$ by Lagrange. Given any $K/F$ finite, we'll soon prove that $\#\Aut_FK \leq [K : F]$.
\end{corollary}

\begin{definition}[Galois Extension]
A finite extension $K/F$ is Galois if \[\#\Aut_F K = [K : F].\]
\end{definition}

We'll soon see that the splitting field of any seperable polynomial is Galois.

\begin{definition}
Define the following set
\[  \Hom_{F\text{-vs}}(K, \Omega) = \left\{ \varphi\colon K \to \Omega \mid \text{$\varphi$ is $F$-linear} \right\}. \]
\end{definition}
These aren't the field embeddings we've been dealing with. We require that $\phi(\lambda a) = \lambda\phi(a)$ and $\phi(a+b) = \phi(a) + \phi(b)$ but nothing more. THese are basically just matrices; if $[K:F] = n$ and $[\Omega:F] = m$ then $\Hom_{F\text{-vs}}\isom \operatorname{Mat}_{m\times n}(F)$. These aren't super nice but are very easy to describe. In particular, we know that it is an $F$-vector space of dimension $nm$. Something actually interesting and non-obvious about this is that it is also an $\Omega$-vector space.
% !TEX root = ../notes.tex

\section{Monday, 4 March 2019}

It's been a while.

Let $p$ be prime. Recall the following definitions:
\begin{itemize}
\item $\Pi(x) = \sum_{p \leq x} 1$;
\item $\vartheta(x) = \sum_{p \leq x} \log p$;
\item $\psi(x) = \sum_{p^n \leq x} \log p = \sum_{n \leq x} \Lambda(x)$.
\end{itemize}

We had two important results from these definitions.

\begin{theorem}
$\abs{\vartheta(x) - \psi(x)} \leq O(\sqrt{x}\log^2x)$.
\end{theorem}

\begin{theorem}
$\Pi(x) = \vartheta(x)/\log x - \int_1^x \vartheta(x)/t\log^2x \dd{t}$.
\end{theorem}

\begin{theorem}
The following are equivalent:
\begin{itemize}
\item $\psi(x) \sim x$,
\item $\vartheta(x) \sim x$,
\item $\Pi(x) \sim x/\log x$.
\end{itemize}
\end{theorem}

\begin{lemma}
$M_{\log x} = \sum_{n \leq x} \log n = x\log x - x + O(\log x)$.
\end{lemma}

\begin{proof}
Consider 
\[ \sum_{n \leq x} \log n \cdot 1. \]
We apply Abel summation using $f(n) = 1$ and $\phi(x) = \log x$ to get that 
\[ \sum_{n \leq x} \log n \cdot 1 = \lfloor x \rfloor \log x - \int_1^x \frac{\lfloor t \rfloor}{t} \dd{t}, \]
noting that $M_1 = \lfloor x \rfloor$. Notice that 
\[ \int_1^x \frac{\lfloor t \rfloor}{t} \dd{t} = \int_1^x 1 - \frac{t - \lfloor t \rfloor}{t} \dd{t} = x - 1 - O\qty(\int_1^x \frac{1}{t} \dd{t}) = x + O(\log x). \]
This acheives the result we wanted.
\end{proof}

\begin{theorem}[Chebyshev]
The inequality 
\[ x \log 2 + O(\log x) \leq \psi(x) \leq x\log 4 + O(\log^2x) \] holds.
\end{theorem}

\begin{proof}
Recall that $M_{\log(x)} = \sum_{n \leq x} \log n$, and that 
\[ \log n = \sum_{d \mid n} \Lambda(d). \]
Then 
\[ M_{\log(x)} = \sum_{n \leq x} \sum_{d \mid n} \Lambda(d) = \sum_{dq \leq x} \Lambda(d) = \sum_{q \leq x}\sum_{d \leq x/q} \Lambda(d) = \sum_{q \leq x} \psi\qty(\frac{x}{q}). \]
Define the quantity
\[ D(x) = M_{\log}(x) - 2M_{\log}\qty(\frac{x}{2}) = \log {x \choose x/2}. \]
On one had, we can show that 
\[ D(x) = x\log(2) + O(\log x), \]
and on the other had we know that 
\[ D(x) = \sum \psi(x/q) - 2\sum\psi(x/2q) = \psi(x) - \psi(x/2) + \psi(x/3) + \cdots. \]
Note that $\psi(x)$ is ``sorta monotone increasing''
\end{proof}
% !TEX root = ../notes.tex

\section{Monday, 25 March 2019}

Recall from last lecture (I don't) that we have a theorem:

\begin{theorem*}
	There are infinitely many primes $p$ satisfying $p \cong 3 \pmod{4}$.
\end{theorem*}

\begin{lemma*}
	Let $q$ be prime. There are infinitely many primes $p \cong 1 \pmod{q}$.
\end{lemma*}

\begin{proof}
	Look at $\Phi_q(x) = \frac{x^q-1}{x-1} = x^{q-1} + \cdots + 1$.
	We know that if $p \mid \Phi_q(x)$ then $p \cong 1 \pmod{q}$.
	Note that $p \mid x^q - 1$ or $x \cong 1 \pmod{p}$.
	In the former case, we know that the order of $x$ is $q$, so $q \mid p-1$, 
	which implies that $p \cong 1 \pmod{q}$.
	In the latter case, we would get that $p \mid q$ which is really problematic 
	since $p$ and $q$ are distinct primes.

	To prove that there are infinitely many such primes, suppose we have some 
	finite list of primes $p_1, \dotsc, p_\ell$.
	Notice that $\Phi_q(x) \cong 1 \pmod{x}$ for all $x$, so let $x$ be
	the product of these finite primes. Then any prime which divides $\prod p_i$
	must itself be congruent to $1$ modulo $q$, and since these prime factors
	must exist we know that there exist infinitely many primes which are 
	congruent to $1$ modulo any prime $q$.
\end{proof}

\subsection*{Outline of Dirichlet's Proof}

\begin{theorem}
	For any $q$ and any $(a,q) = 1$ there are infinitely many primes of the form
	$p = a + kq$.
\end{theorem}

First, some definitions:

\begin{definition}[Character]
	A character is a homomorphism from an Abelian group $G$ to the complex
	numbers $\C$.
\end{definition}

\begin{example}[Example of a character]
	Let $G \isom (\Z/m\Z, +)$.
	There are $m$ characters of the form 
	\[ \psi_a(x) = \zeta^{ax}, \]
	where $\zeta$ is the $m$\textsuperscript{th} root of unity.
	One may trivially check that this is, in fact, a homomorphism.
\end{example}

A list of facts:

\begin{enumerate}
	\item The characters of $G$ themselves form an Abelian group $\widehat{G}$, 
	and there are exactly $\abs{G}$ of them.
	\item Orthogonality relations.
	\begin{itemize}
		\item \[ \sum_{a \in G} \psi(g) = \begin{cases}
			\abs{G} & \text{if $\psi \cong 1$,} \\
			0 & \text{otherwise}.
		\end{cases} \]
		\item \[ \sum_{\psi \in \widehat{G}} \psi(g) = \begin{cases}
			\abs{G} & \text{if $g = 0$,} \\
			0 & \text{otherwise}.
		\end{cases} \]
	\end{itemize}
\end{enumerate}

Let $\chi$ be some character of the group $(\Z/q\Z)^\times$.

\begin{definition}[Dirichlet Character]
	The Dirichlet character is a map $\chi\colon \N \to \C$ which extends
	a regular character by saying that
	\[ \chi(m) = \chi(\bar{m}) \quad \text{if $m \cong \bar{m} \pmod{q}$}, \]
	and that $\chi(m) = 0$ if $(m,q) \not= 1$.
\end{definition}

\begin{example} Let $q$ = 5.
	\begin{itemize}
		\item The trivial character: 
		\[ \chi_o = 1,1,1,1,0,1,1,1,1,0,\dotsc \]
		\item The Legendre symbol
		\[ \qty(\frac{m}{5}) = \begin{cases}
			-1 & \text{$m$ non residue,} \\
			1 & \text{$m$ residue,} \\
			0 & \text{$m \cong 0 \pmod{5}$.}
		\end{cases} \]
	\end{itemize}
\end{example}

The character is a completely multiplicative function.

\begin{definition}[$L$-function]
	Given a Dirichlet character $\chi$, we define the Dirichlet $L$-function 
	$L(s,\chi)\colon \text{``$\C$''} \to \C$ by 
	\[ 
		L(s,\chi) = \sum_{n=1}^\infty \frac{\chi(n)}{n^s},
	\]
	which is just the Dirichlet series of $\chi$.
\end{definition}

\begin{example}
	Let $\chi = \chi_0$ and let $q$ be prime.
	Then $L(s, \chi_0) = \sum 1/n^s$ where $n \not\cong 0 \pmod{q}$.
\end{example}

\begin{remark*}
	Why care about these characters? Well, they are really good at picking out
	numbers which are $1$ modulo $q$.
\end{remark*}

\begin{theorem*}
	There are infinitely numbers $x \cong a \pmod{q}$.
	Yeah, this is easy.
\end{theorem*}

\begin{proof}
	Let's look at $\psi$ characters of $(\Z/m\Z, +)$.
	Extend to Dirichlet character $\psi\colon \N \to \C$.
	Look at 
	\begin{align*}
		\sum_{\psi} L(s,\psi) &= \sum_\psi \sum_{n=1}^\infty \frac{\psi(n)}{n^s} \\
							&= \sum_{n=1}^\infty \frac{1}{n^s}\sum_\psi \psi(n) \\
							  &= m \sum_{n =km } \frac{1}{n^s}.
	\end{align*}
	If there were finitely many $n \cong 0 \pmod{m}$, then the right hand side
	would need to be finite.
	However, we can show rather easily that
	\[ \lim_{s \to 1^+} \sum_\psi L(s,\psi) = \infty, \]
	and so there must be infinitely many $n$.
	The obstacles we'll face are 
	\begin{itemize}
		\item Actually analyze the above sum;
		\item We want to sum over only primes which are congruent to $1$ modulo
		$q$;
		\item We want to restrict our attention to things that are only 
		relatively prime to $q$.
	\end{itemize}
\end{proof}

\subsubsection*{Convergence of \texorpdfstring{$L$}{L}-functions}

Let $\chi$ be a character of $G \isom (\Z/q\Z)^\times$.
Assume that $\chi \not= \chi_0$.
We know that 
\[ 
	\sum_{x\in G} \chi(x) = 0 \implies \sum_{x=0}^{q-1} \chi(x) = 0, \tag{by orthogonality}
\]
so consider 
\[ 
	\sum_{n \leq x} \chi(n) = \sum_{n \leq kq} \chi(n) + \sum_{kq+1}^x \chi(n)
	\leq \varphi(q),	
\]
where the first term must be $0$, and in the latter, there are at most 
$\varphi(q)$ summands which are relatively prime to $q$.
This implies that the $L$-functions always converge.

\begin{problem}[Homework]
	Let $f(n)$ be a monotonically decreasing positive function. 
	Then $\sum_{n =M}^N f(n)\chi(n) \leq 3\varphi(q)f(M)$.
	\textbf{Hint:} We know that $\chi$ is a periodic function.
	Use summation by parts.
\end{problem}

\begin{corollary}
	If $\chi$ is not $\chi_0$ and $s > 0$ then 
	\[ L(s,\chi) = \sum \frac{\chi(n)}{n^s} \]
	converges.
\end{corollary}

\begin{proof}
	Use the homework problem to get that 
	\[ \abs{L(s,\chi) - \sum_{n \leq x} \frac{\chi(n)}{n^s}} \leq \frac{3\varphi(q)}{n^s}, \]
	which goes to zero.
\end{proof}

Since this function converges, we get an Euler product
\begin{align*}
	L(s, \chi) &= \prod_{\text{$p$ prime}} \qty(1 + \frac{\chi(p)}{p^2} + \frac{\chi(p^2)}{p^{2s}} + \cdots) \\
	&= \prod_{\text{$p$ prime}} \qty(\frac{1}{1-\chi(p)/p^s}).
\end{align*}

In particular, 
\[ L(s,\chi_0) = \zeta(s)\prod_{p \mid q}\qty(1-\frac{1}{p^s}). \]
% !TEX root = ../notes.tex

\section{Thursday, 28 March 2019}

\subsection{Grace Hopper's Thesis}



\end{document}