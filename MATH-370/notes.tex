% !TEX program = pdflatex

\documentclass{lnotes}

% Packages
\usepackage[osf]{libertineRoman}
\usepackage{epigraph}

% Metadata
\title{Fields and Galois Theory}
\course{\textsc{math} 370}
\place{yale university}
\term{spring}
\year{2019}
\blurb{
	These are lecture notes for \textsc{math} 370b, ``Fields and Galois Theory,'' taught by Asher Auel at Yale University during the spring of 2019.
	These notes are not official, and have not been proofread by the instructor for the course.
	They live in my lecture notes respository at 
	\[\text{\url{https://github.com/jopetty/lecture-notes/tree/master/MATH-370}.}\] 
	If you find any errors, please open a bug report describing the error and label it with the course identifier, or open a pull request so I can correct it.
}

% Bibliography
\usepackage[
	style = alphabetic,
]{biblatex}
\addbibresource{references.bib}

\begin{document}

\section*{Syllabus}

\begin{tabularx}{\textwidth}{rX}
\toprule
\textbf{Instructor} & Asher Auel, \url{asher.auel@yale.edu} \\
\textbf{Lecture} & \textsc{tr} 11:35 \textsc{am} -- 12:50 \textsc{pm} in \textsc{lom} 215 \\
\textbf{Peer Tutor} & Arthur Azvolinsky, \url{arthur.azvolinsky@yale.edu} \\ 
\textbf{Section} & Math Lounge \\
\textbf{Exams} & Midterm 1: Febuary 19; \quad Midterm 2: April 9; \quad Final: May 3. \\
\textbf{Textbook} & \fullcite{textbook} \\
\bottomrule
\end{tabularx} \\

The main object of study in Galois theory are roots of single variable polynomials.
Many ancient civilizations (Babylonian, Egyptian, Greek, Chinese, Indian, Persian) knew about the importance of solving quadratic equations.
Today, most middle schoolers memorize the ``quadratic formula'' by heart.
While various incomplete methods for solving cubic equations were developed in the ancient world, a general ``cubic formula'' (as well as a ``quartic formula'') was not known until the 16th century Italian school.
It was conjectured by Gauss, and nearly proven by Ruffini, and then finally by Abel, that the roots of the general quintic polynomial could not be solvable in terms of nested roots.
Galois theory provides a satisfactory explanation for this, as well as to the unsolvability (proved independently in the 19th century) of several classical problems concerning compass and straight-edge constructions (e.g., trisecting the angle, doubling the cube, squaring the circle).
More generally, Galois theory is all about symmetries of the roots of polynomials.
An essential concept is the field extension generated by the roots of a polynomial.
The philosophy of Galois theory has also impacted other branches of higher mathematics (Lie groups, topology, number theory, algebraic geometry, differential equations).

This course will provide a rigorous proof-based modern treatment of the main results of field theory and Galois theory.
The main topics covered will be irreducibility of polynomials, Gauss's lemma, field extensions, minimal polynomials, separability, field automorphisms, Galois groups and correspondence, constructions with ruler and straight-edge, theory of finite fields.
The grading in Math 370 is very focused on precision and correct details.
Problem sets will consist of a mix of computational and proof-based problems.

Your final grade for the course will be determined by
\[ \max\left\{
	\begin{array}{cccc}
		\text{20\% homework} + \text{25\% midterm 1} + \text{25\% midterm 2} + \text{30\% final} \\
		\text{20\% homework} + \text{25\% midterm 1} + \text{15\% midterm 2} + \text{40\% final} \\
		\text{20\% homework} + \text{15\% midterm 1} + \text{25\% midterm 2} + \text{40\% final}
	\end{array}
\right\}. \]

\printbibliography

% !TEX root = ../notes.tex

\section{January 14, 2019}

Given an interval $(a,b) \subset \R$, we know that the size of this interval is $b-a$. The focus of this course will be the study of the generalization of this idea using the \emph{Lebesgue measure} on $\R$. Equipped with this, we can talk of the \emph{Lebesgue integral} of ``nice'' functions, which is more powerful than the Riemannian equivalent.

\subsection{The Metric Space}

\begin{definition}[Metric Space]
Given a set $X$, a metric function $d$ is a function $d : X \times X \to \R$ obeying the following three properties.
\begin{enumerate}
\item \textbf{Positivity:} $d(x,y) \geq 0$ and $d(x,y) = 0$ if and only if $x = y$;
\item \textbf{Symmetry:} $d(x,y) = d(y,x)$ for all $x,y \in X$;
\item \textbf{Triangle Inequality:} $d(x,y) \leq d(x,z) + d(z,y)$ for all $x,y,z \in X$. 
\end{enumerate}
A metric space is a pair $(X,d)$ where $d$ is a metric function on $X$.
\end{definition}

\begin{example}[Metric Spaces]
\begin{parts}
\part In $\R$, we have the traditional $d(x,y) = \abs{x-y}$.
\part In $\R^2$, we have $d\qty((x_1,x_2),(y_1,y_2)) = \sqrt{(x_1-y_1)^2 + (x_2-y_2)^2}$.
\part In $\R^2$, we also have $d\qty((x_1,x_2),(y_1,y_2)) = \max\{\abs{x_1-y_2}, \abs{x_2-y_2}\}$.
\part The discrete metric on a set $X$ is defined by 
\[ d(x,y) = \begin{cases}
0 & \text{if $x = y$,} \\
1 & \text{otherwise.}
\end{cases} \]
\part Given a metric space $(X,d)$ and $Y \subset X$ then $(Y,d)$ is also a metric space where $d$ is restricted to $Y \times Y$.
\end{parts}
\end{example}

\begin{definition}[Neighborhood]
Fix a metric space $(X,d)$. For some $r \geq 0$, the $r$-neighborhood of $x$ is $B(x,r)$, the set $\{y \in X \mid d(x,y) < r\}$. Notice that this depends on the metric! In $\R$ with the discrete metric, $B(0,1) = \{0\}$ while $B(0,2) = \R$ which is not what we expect from the traditional metric.
\end{definition}

\begin{definition}[Interior Points]
Let $A \subset X$. A point $x \in A$ is an interior point of $A$ if there exists some $r > 0$ such that $B(x,r) \subset A$. That is, we can draw a ball around $x$ which lies entirely in $A$.
\end{definition}

\begin{example}
If $A = [0,1)$, then the interior points of $A$ are $(0,1)$ but $0$ is not an interior point.
\end{example}

\begin{definition}[Open Sets]
A subset $A \subset X$ is open if every point in $A$ is interior. The empty set is vacuously open.
\end{definition}

\begin{proposition}
For any $x \in X$ the $r$-neighborhood of $x$ is an open subset of $X$.
\end{proposition}

\begin{proof}
Let $y \in B(x,r)$. Let $r_0 = r - d(x,y)$. Then $r_0 > 0$ and $B(y,r_0) \subset B(x,r)$ regardless of which $y$ is chosen since for any $z \in B(y,r_0)$ we know that $d(x,z) \leq d(x,y) + d(r,z) < d(x,y) + r_0 < r$. Then every point of $B(x,r)$ is interior and so it is open.
\end{proof}

Using this, we can now call $B(x,r)$ the open ball of radius $r$ centered at $x$.

\begin{example}
In $\R^2$ with the standard metric, an open ball looks like an open disc. With the maximum metric, it looks like an open square. In $\R$, we can look at the set of all rational numbers $\Q$. This set is not open since for all $q \in \Q$ and all $r > 0$ there exists an $x \in B(q,r)$ where $x \notin \Q$.
\end{example}

\begin{proposition}\label{prop:1-2}
The intersection of finitely many open sets is open. The union of any open sets is open.
\end{proposition}

\begin{example}
The intersection of infinitely many open sets is not necessarily open. Consider $\bigcap\, (0,1/n)$ as $n \to \infty$. The intersection is simply $\{0\}$ which is not an open set.
\end{example}

\begin{proof}[Proof of Proposition~\ref{prop:1-2}]
Let $A_1, \cdots, A_k$ be open subsets of $X$. Let $x \in A_1 \cap \cdots \cap A_k$. Since each $A_i$ is open we know that $x$ is an interior point of $A_i$, so there exists some $r_i$ such that $B(x,r_i) \subset A_i$. Let $r$ be the minimum of all such $r_i$. Then $B(x,r) \subset A_i$ for all $i$, so this open ball is contained in the intersection.

Now let $\{A_\alpha \mid \alpha \in I \}$ be a collection of open subsets. Let $x \in \bigcup\,A_\alpha$. Then $x$ is contained in some open $A_\alpha$, and so there exists an $r_\alpha$ such that $B(x,r_\alpha) \subset A_\alpha$, so $B(x,r_\alpha) \subset \bigcup\,A_\alpha$.
\end{proof}

\begin{definition}[Interior of a Set]
For $A \subset X$, the set of all interior points of $A$ is called the interior of $A$. This is usually written as $\operatorname{Int}(A)$ or $A^\circ$.
\end{definition}

\begin{example}
If $A = [a,b]$ then $A^\circ = (a,b)$. If $A = \Q$ then $\Q^\circ = \emptyset$.
\end{example}

\begin{proposition}
For all $A$, the interior of $A$ is open. Furthermore, $A^\circ$ is the largest open subset of $A$ in the sense that it contains all other open subsets of $A$.
\end{proposition}

\begin{proof}
It's just the definitions.
\end{proof}

\begin{proposition}
If $A \subset B$ then $A^\circ \subset B^\circ$.
\end{proposition}

\begin{corollary}
A set $A$ is open if and only if $A = A^\circ$.
\end{corollary}

\begin{definition}[Limit Point]
Let $A \subset X$. A point $x \in X$ is a limit point of $A$ if for any $r > 0$ we know that $B(x,r) \cap A \not= \emptyset$. Notice that every point $a \in A$ is a limit point of $A$.
\end{definition}

\begin{example}
Let $A = [0,1)$. Then $0$ is a limit point of $A$ since every open ball centered centered at $0$ interesects $A$. Furthermore, $1$ is also a limit point for the same reason. If $A = \Q$, then the set of limit points of $\Q$ is all of $\R$.
\end{example}

\begin{definition}[Closed Set]
A set $A \subset X$ is called closed if every limit point of $A$ is contained in $A$.
\end{definition}

\begin{example}
The interval $[0,1]$ is closed but $[0,1)$ is not. To show that something isn't a limit point, use the minimum distance between this point and the interval. This must be positive since otherwise it would be in the interval. Then let your $r$ be smaller than this, and the open ball with this radius centered at this point will not intersect the original interval. Generalize to higher dimensions as needed.
\end{example}

\begin{corollary}
Given any metric space $X$, we know that $\emptyset$ is closed. Furthermore, $\bar{B}(y,r) = B[y,r] = \{y \mid d(x,y) \leq r \}$ is closed for any $r$.
\end{corollary}

\begin{proposition}
Let $A \subset X$. We know that $A$ is open if and only if $A^\complement$ is closed.
\end{proposition}

\end{document}