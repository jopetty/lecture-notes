% !TEX root = ../notes.tex

\section{January 17, 2019}

\epigraph{``This is true by the Freshma---First Year's Dream''}{Asher}

\subsection{Reminders from \textsc{math} 350}

\begin{enumerate}
\item $F$ is a field, or a commutative unital ring where every $F^\times = F \setminus \{0\}$.
\item $F[x]$ is the ring of polynomials with coefficients in $F$.
\item $\partial : F[x] \to \N$ is the degree function. We know that $\partial(fg) = \partial(f) + \partial(g)$ and $\partial(f+g) \leq \max\{\partial(f), \partial(g)\}$. Because of multiplicativity, we know that $F[x]^\times$ are nonzero constant polynomials.
\end{enumerate}

\begin{theorem}
Given a field $F$, we know that $F[x]$ is a Euclidean Doman with respect to $\partial$. That us, given $f,g \in F[x]$ where $g \not= 0$, there exist $q,r \in F[x]$ such taht $f = q \cdot g + r$ where either $r = 0$ or $\partial(r) < \partial(g)$.
\end{theorem}

\begin{theorem}
Every Euclidean Doman is a Principal Ideal Domain. 
\end{theorem}

\begin{definition}[Ideal]
A subset $I \subset R$ of a ring is an ideal if $I$ is a subring and it is closed under multiplication, so that $rI \subset I$ for every $r \in R$ (we implicitly assume that $R$ is commutative here). Any ideal which contains a unit is the whole ring.
\end{definition}

\begin{definition}[Principal Ideal]
An ideal $I \subset R$ is principal if it is generated by one thing, so $I = (r) = rR$ for some $r \in R$,
\end{definition}

\begin{corollary}
$F[x]$ is a PID. If $I = (f)$ then $I = (cf)$ for some $c \in F^\times$, so we choose our polynomials to be \emph{monic}.
\end{corollary}

\begin{corollary}
The set of all ideals $\{I \subseteq F[x]\}$ is in bijection with the set of monic polynomials $\{f \in F[x]\}$.
\end{corollary}

\begin{definition}[Prime Idea]
An ideal $I \subseteq R$ is prime if $ab \in I$ implies that either $a$ or $b$ is an element of $I$. Equivalently, this says that $R / I$ is an integral domain.
\end{definition}

\begin{definition}[Maximal Ideal]
An ideal $I \subseteq R$ is maximal if $I \subseteq J \subseteq R$ then either $I = J$ or $J = R$. Equivalently, $R/I$ is a field, so there are no nonzero nonunit elements of $R/I$.
\end{definition}

\begin{example}[Ideals in $\Z$]
A prime ideal $I$ in $\Z$ is either $(0)$ or $(p)$ for some prime $p$. In any integral domain, $(0)$ is prime. The only maximal ideals in $\Z$ are $(p)$ by Bezout's Theorem.
\end{example}

\begin{definition}[Prime and Irreducible Elements]
An element $r$ in $R$ is prime if $(r)$ is prime, or if $r$ divides $ab$ then $r$ divides either $a$ or $b$. An element $r$ is irreducible if $r = ab$ then $a \in R^\times$ or $b \in R^\times$. Primality always implies irreducibility, but the converse only holds in PIDs.
\end{definition}

\begin{lemma}
If $R$ is a PID then every irreducible element is prime.
\end{lemma}

\begin{definition}[UFD]
A Unique Factorization Domain is a commutative unital ring $R$ with the following two properties for every nonzero $r \in R$:
\begin{enumerate}
\item $r = r_1 \cdot r_n$ where $r_1$ through $r_n$ are irreducible;
\item this finite product is unique, so if $r = r_1 \cdots r_n$ and $r = s_1 \cdots s_m$ then $n = m$ and $r_i = u_i s_i$ for some unit $u_i \in R^\times$.
\end{enumerate}
\end{definition}

\begin{theorem}
Every PID is a UFD.
\end{theorem}

\begin{corollary}
$F[x]$ is a UFD. Equivalently, if $f \in F[x]$ is nonzero then $f = c \cdot f_1 \cdot f_2 \cdots f_n$ where $c \in F^\times$ and $f_i$ are monic irreducible polynomials, and this decomposition is unique.
\end{corollary}

\begin{problem}
Given a polynomial $f \in F[x]$, how do we check if $f$ is irreducible?
\end{problem}

\subsection{Roots and Irreducibility}

\begin{definition}
If $f$ is a nonzero element of $F[x]$ then $a \in F$ is a root of $f$ if $f(a) = 0$. We can use the division algorithm to write $f$ as $g \cdot (x-a) + r$ where $r = 0$ or $\partial(r) \leq \partial(x-a) = 1$, so $r$ is a constant.
\end{definition}

\begin{corollary}
An element $a \in F$ is a root of $f$ if and only if we can write it as $f = g \cdot (x-a)$.
\end{corollary}

\begin{corollary}
If $f \in F[x]$ and $\partial(f) = n > 0$ then $f$ has at most $n$ roots.
\end{corollary}

\begin{proof}
Either by induction using $f = g \cdot (x-a) = h \cdot (x-b) \cdot (x-a) = \cdots$ or we use the fact that $F[x]$ is a UFD, so we write $f$ as $f_1 \cdots f_m$ and we know that $m \leq n$ and if any $f_i$ is nonlinear then it has no roots. 
\end{proof}

\begin{example}[Polynomials with no roots]
\item Consider $f\colon x \mapsto x^2 + 1$ in $\R[x]$.
\item Consdier $f\colon x \mapsto x^2-2$ in $\Q[x]$.
\item Consider $f\colon x \mapsto x^2 + x + 1$ in $\F_2[x]$.
\end{example}

\begin{proposition}
If $f \in F[x]$ is irreducible of degree greater than or equal to $2$ then $f$ has no roots in $F$.
\end{proposition}

\begin{proof}
$F[x]$ is a unique factorization domain, so we can't write it as a product of anything of a smaller degree, which we showed is the same as writing it as a $(x-r) \cdot g$ in some way.
\end{proof}

\begin{proposition}
If $f \in F[x]$ is a polynomial degree at most $3$ and $f$ has no roots then $f$ is irreducible.
\end{proposition}

\begin{proof}
Unique Factorization! Assume that $f$ is reducible and $\partial(f) \leq 3$. Then $f = c \cdot f_1 f_2$ where $f_1$ and $f_2$ are nonconstant. If $\partial(f) \leq 3$ then $\partial(cf_1f_2) = \partial(f_1) + \partial(f_2) \leq 3$. Since $\partial(f_i) > 0$ then $\partial(f_1) = \partial(f_2) = 1$, which means that we found linear polynomials and therefore roots. 
\end{proof}

\begin{example}[Warning]
Be careful with the above! Consider $(x^2+1)^2 \in \R[x]$. This has no roots but is irreducible.
\end{example}

\begin{example}[Warning]
The fact that the number of roots is at most the degree fails if we don't work in a field. Consider $f(x) = x^2 - 1 \in \Z/8\Z[x]$. This has degree $2$ but all odd numbers less than eight are roots.
\end{example}

\begin{theorem}[Fundamental Theorem of Arithmetic]
Any nonconstant $f \in \C[x]$ has a root. Then any irreducible polynomial $f \in \C[x]$ is linear.
\end{theorem}

\begin{problem}[Proving Irreducibility]
\begin{parts}
\item If $\partial f \leq 3$ then it is irreducbile if it has no roots.
\item 
\begin{theorem}
Assume $f(x) \in \Z[x]$, and assume that $p$ is a prime such that $p$ doesn't divide $a_n$ and $\bar{f}(x) = \bar{a}_nx^n$ is irreducible, then $f$ is irreducible over $\Z[x]$. This is a reduction ring homomorphism $\Z[x] \to \F_p[x]$.
\end{theorem}
\end{parts}
\end{problem}