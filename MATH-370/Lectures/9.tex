% !TEX root = ../notes.tex

\section{Tuesday, 12 January}

Recall that $\mathfrak{p} \subseteq \C$ is the subfield of constructible numbers, and $\Q^{\mathrm{py}} \subseteq \Q$ is the Pythagorean closure of the rationals. We had a theorem that these were equal to one another. A corollary to this is the fact that $\alpha \in \mathfrak{p}$ if and only if there exists a tower of fields of relative degree two ending at $\Q(\alpha)$, which tells us that $[\Q(\alpha) : \Q] = 2^n$. This proves that we cannot double the cube since $[\Q(\sqrt[3]{2}) : \Q] = 3$.

\subsection{Squaring the Circle}
There is also the problem of squaring the circle, which is also impossible since if a circle has area $\pi$ then the sides of the square would need length $\sqrt{\pi}$, which is not algebraic in $\Q$. This implicitly assumes that all constructible numbers are algebraic, which isn't necessarily a trivial observation. However, part of proving that $\mathfrak{p}$ is a field is showing that $\mathfrak{p}/Q$ is algebraic.

\subsection{Trisecting an Arbitrary Angle}
The general statement of this problem is on the homework. Consider $\theta = 2\pi/3$. Then $\theta/3 = 2\pi/9$. The intersection of $\theta$ with the unit circle is $\omega$ and the intersection with $\theta/3$ with the unit circle is $\zeta_9$. Then the constructibility $\theta/3$ is then equivalent to the constructibility of $\zeta_9$. Since the minimal polynomial of $\cos(\zeta_9)$ has degree three, it is not constructible.

\begin{lemma}
The angle $2\pi/n$ is constructible if and only if $\zeta_n$ is constructible if and only if $\cos(2\pi/n)$ and $\sin(2\pi/n)$ are constructible.
\end{lemma}

\subsection{Splitting Fields}
Let $F \subseteq \Omega$ be a field and let $f(x) \in F[x]$ have degree $n$. We say that $f$ splits in $\Omega$ if $f(x) = a \prod (x-\alpha_i)$ for $\alpha_i \in \Omega$, so it separates into linear factors.

\begin{example}
Consider $f(x) = x^3 - 2 \in \Q$. We know that $\sqrt[3]{2}$ is a real root. How does $f(x)$ split in $\Q\qty(\sqrt[3]{2})$. Since it has a root, we know it factors as $f = (x-\sqrt[3]{2})g(x)$ where $g \in \Q\qty(\sqrt[3]{2})[x]$ has degree two. We can calculate what $g$ is through polynomial division to get that
\[ f(x) = \qty(x-\sqrt[3]{2})\qty(x^2 + \sqrt[3]{2}x + \sqrt[3]{2}^2). \]
We can check the discriminant is negative to see that the remaining quadratic has complex roots, and so cannot split further over $\Q\qty(\sqrt[3]{2})$.
\end{example} 

\begin{definition}[Splitting field]
A splitting field for $f(x) \in F[x]$ is an extension $E/F$ such that 
\begin{itemize}
\item $f(x)$ splits in $E$, so $f = a\prod (x-\alpha_i)$
\item $E = F(\alpha_1, \dotsc, \alpha_n)$.
\end{itemize}
In this way, it is kinda like the ``minimal field'' over which $f$ splits.
\end{definition}

\begin{theorem}
Let $f \in F[x]$. Then a splitting field $E/F$ for $f$ exists, and moreover if $\deg(f) = n$ then $[E:F] \leq n!$.
\end{theorem}

\begin{example}
Recall the Cyclotomic polynomial $\Phi_p(x) = (x^p-1)/(x-1)$ for prime $p$. Any $p$\textsuperscript{th} root of $1$ is a root since $\Phi_p(\zeta_p) = (\zeta_p^p-1)/(\zeta_p-1) = 0$. But $\zeta_p^k$ is also a root, and there are $p-1$ of them, which is the degree of $\Phi_p(x)$. So in fact $\Phi_p(x)$ splits over $\Q(\zeta_p)$.
\end{example}

\begin{proof}[Proof of the existence of a finite splitting field]
We can find the splitting field by adjoining all the roots to $\Q$. Then its finitely generated and algebraic, so it is finite.
\end{proof}

\begin{proof}[Proof of the bound on the degree]
Just divide the polynomial by the one root we do have to get a polynomial of degree $n-1$. Then repeat the procedure until we get linear terms. Then we have a tower where the relative degree of the $i$\textsuperscript{th} is $n-i$, and so the degree of the full extension is bounded by $n \cdot (n-1) \cdots 1 = n!$.
\end{proof}

\begin{theorem}
Splitting fields are unique up to an $F$-isomorphism.
\end{theorem}

\begin{theorem}
Let $f \in F[x]$ be a monic polynomial. Let $E/F$ be generated by some subset of the roots of $f$. Let $\Omega/F$ be a field extension where $f$ splits. Then 
\begin{enumerate}[label={\textup{(\alph*)}}]
\item There exist $F$-homomorphisms $\phi_i : E \to \Omega$, and the number of distinct $\phi_i$ is at most $[E : F]$ with equality if $f$ has distinct roots;
\item If $E$ and $\Omega$ are splitting fields then they are isomorphic.
\end{enumerate}
\end{theorem}