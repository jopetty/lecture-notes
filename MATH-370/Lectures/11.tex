% !TEX root = ../notes.tex

\section{Thursday, 21 February 2019}

\epigraph{``You are right to be skeptical of this calculus business''}{Asher}

Recall our proposition:

\begin{proposition}
Let $f \in F[x]$ and let $E/F$ be an extension generated by some subset of the roots of $f$. Suppose that $\Omega/F$ is the splitting field of $f$. Then $1 \leq \#\Hom_F(E,\Omega) \leq [E:F]$ with upper-bound equality when $f$ has all distinct roots.
\end{proposition}

\begin{corollary}
If $E,E'$ are splitting fields of $f$ then $E \isom E'$.
\end{corollary}

\begin{proof}
Let $E = \Omega$ as in the above proposition, and exhibit an $F$-homomorphism. Then let $E' = \Omega$, and do the same. Since $F$-homs are injective, and we have two finite dimensional vector-spaces with injective homs between them, then they are isomorphic.
\end{proof}

\begin{corollary}
If $E/F$ is any finite extension and $L/F$ is any extension then $\#\Hom_F(E,L) \leq [E:F]$ and there always exists an extension $\Omega/L$ such that $\Hom_F(E,\Omega) \not= \varnothing$.
\end{corollary}

\subsection{Multiple Roots}

Let $f \in F[x]$ be a polynomial, and let $\Omega$ be a field in which $f$ splits, and let $\alpha \in \Omega$ is a root of $f$.

\begin{definition}[Multiple, Simple Roots; Seperable]
We say that $\alpha$ is multiple root of $f$ if and only if $(x-\alpha)^k$ divides $f$ over $\Omega$ for multiplicity $k > 1$. Otherwise, we say that $\alpha$ is a simple root. The polynomial $f$ is seperable if every root is simple.
\end{definition}

Over $\C$, we can test for multiple roots. Suppose that $f(x) = (x-\alpha)^2g(x)$. Consider the derivative of $f$, which is 
\[ f'(x) = 2(x-\alpha)g(x) + (x-\alpha)^2g'(x) = (x-\alpha)h(x). \]
Notice that we can factor out a term of $(x-\alpha)$. More generally, if $f$ has a root of multiplicity at least $m$, then $f'$ has the same root with multiplicity at least $m-1$. This gives us the formula
\[ \operatorname{mult}_\alpha f = \max\{i < \deg f \mid f^{(i)}(\alpha) = 0\}. \]
This implies that $(x-\alpha)$ divides $\gcd(f,f')$.

\subsection{The Derivative}

For any polynomial $f = \sum a_ix^i$, define $f' = \sum ia_ix^{i-1}$. This has some nice properties:
\begin{itemize}
\item $d/dx\colon F[x] \to F[x]$ is $F$-linear;
\item $d/dx(a) = 0$ for constant $a$;
\item The Chain Rule;
\item The Product Rule;
\end{itemize}

\begin{proof}
Induction.
\end{proof}

\begin{proposition}
Let $f \in F[x]$. Then $f$ is seperable if and only if $f$ and $f'$ are relatively prime.
\end{proposition}

\begin{proof}
``$\implies$'' Let $f$ be seperable, and let $\Omega$ be a field where $f$ splits, and let $\alpha \in \Omega$ be a root of $f$. Then $\alpha$ is simple if and only if $f(x) = (x-\alpha)h(x)$ and $h(\alpha) \not= 0$. Consider $f'(x) = h(x) + (x-\alpha)h'(x)$ by the chain/product rules, and so $f'(\alpha) \not= 0$. Thus no root of $f$ is a root of $f'$, and hence they share no common factors and so are relatively prime in $F$.

``$impliedby$'' Proof by contrapositive. Assume that $f$ is not seperable, and let $\alpha$ be a multiple root. Hence $f(x) = (x-\alpha)^2g(x)$ over $\Omega$. Then $f' = 2(x-\alpha)g(x) + (x-\alpha)^2g'(x)$, and so $f'(\alpha) = 0$. Then $f$ and $f'$ have a common root, and so $m_\alpha(x)$ divides both $f$ and $f'$ over $F$. Thus their \textsc{gcd} is not $1$, and so they are not relatively prime.
\end{proof}

\begin{exercise}
Let $D\colon F[x] \to F[x]$ be an $F$-linear map satisfying the product rule and $D(x) = 1$. Prove that $D$ is exactly the derivative. (Hint: Try Induction.)
\end{exercise}

\begin{definition}[Seperable Extension]
An algebraic etension $K/F$ is called seperable if every element $\alpha \in K$ satisfies a seperable polynomial over $F$. Equivalently, the minimal polynomial over $F$ of every $\alpha \in K$ is seperable.
\end{definition}

\begin{proposition} Some nice facts:
\begin{enumerate}
\item Let $f \in F$ be irreducible. Then $f$ is seperable if and only if the derivative doesn't vanish.
\item If $\operatorname{char} F = 0$ (so $\Q \subseteq F$) then every irreducible polynomial is seperable.
\item If $\operatorname{char} F = p$ (so $p=0 \in F$) then $f$ is not seperable if and only if $f(x)$ is a polynomial in $x^p$, so $f(x) = g(x^p)$.
\end{enumerate}
\end{proposition}