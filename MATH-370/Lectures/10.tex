% !TEX root = ../notes.tex

\section{Thursday, 14 February \texorpdfstring{$\heartsuit$}{<3}}

\subsection{Splitting Fields}

Splitting fields are unique up to $F$-isomorphism.

\paragraph*{Extension Fields} Let $F$ be a field, and let $K/F$, $\Omega/F$ be extensions of $F$. A map $\phi\colon K\to \Omega$ if $\phi$ is an $F$-homomorphism if it is a ring homomorphism which preserves $F$. This means that $\phi$ is an $F$-linear map of vectorspaces. Let $\operatorname{Hom}_F(K, \Omega)$ be the set of $F$-homomorphisms. Since $F$-homomorphisms are injectives, we refer to them as embeddings or $F$-embeddings.

\begin{example}[Examples of $F$-homomorphisms] \leavevmode
\begin{itemize}
\item Complex conjugation $\sigma\colon \C \to \C$ is an $\R$-homomorphism which sends $i$ to $-i$. In fact, $\operatorname{Hom}_\R(\C,\C) = \{\operatorname{id}, \sigma\}$.
\end{itemize}
\end{example}

\begin{theorem}[Extension Theorem]
Let $F(\alpha)$ be a simple extension of $F$, and let $\Omega/F$ be an extension of $F$.
\begin{enumerate}[label={\textup{(\alph*)}}]
\item Assume that $\alpha$ is trancendental over $F$. Then for every $F$-homomorphism $\phi$ between $F(\alpha)$ and\/ $\Omega$, the image $\phi(\alpha)$ is trancendental over $F$ and there is a bijection between $\Hom_{F}(F(\alpha), \Omega)$ and the set of all trancendental elements of\/ $\Omega$ given by $\alpha \mapsto \phi(\alpha)$. 
\item Assume that $\alpha$ is algebraic over $F$ and let $f \in F[x]$ be its minimal polynomial. Then for every $F$-homomorphism $\phi$ between $F(\alpha)$ and\/ $\Omega$, the image $\phi(\alpha)$ is a root of $f$ and there is a bijection between $\Hom_F(F(\alpha), \Omega)$ and the set of roots of $f$ in $\Omega$. In particular, the size of\/ $\Hom_F(F(\alpha), \Omega)$ is equal the number of distinct roots of $f$ in $\Omega$.
\end{enumerate}
\end{theorem}

\begin{example} \leavevmode
\begin{itemize}
\item Consider $\Q(\sqrt[3]{2})$ whose minimal polynomial is $x^3-2$. The size of $\Hom_\Q(\Q(\sqrt[3]{2}), \R)$ is $1$ since $x^3-2$ has only one real root. The lone element is the identity map $\operatorname{id}$.
\item Consider $\Hom_\Q(\Q(\sqrt[3]{2}), \C)$. We know that this has size $3$ since there are three complex roots of $x^3-2$. We know that $\operatorname{id} \in \Hom_\Q(\Q(\sqrt[3]{2}), \C)$, but what are the other two. Call them $\phi_1$ and $\phi_2$, where $\phi_1\colon \sqrt[3]{2} \mapsto \omega\sqrt[3]{2}$ and $\phi_2\colon \sqrt[3]{2} \mapsto \bar{\omega}^2\sqrt[3]{2}$.
\end{itemize}
\end{example}
\pagebreak
\begin{proof}[Proof of Extension Theorem] \leavevmode
\begin{enumerate}[label={\textup{(\alph*)}}]
\item Assume that $\alpha$ is trancendental over $F$. Recall that $F[\alpha] \isom F[x]$. Given any $\beta \in \Omega$ which is trancendental over $F$, we know that there exists a unique $F$-algebra homomorphism $\phi \colon F[x] \to \Omega$ given by $x \mapsto \beta$. Since $\beta$ is trancendental, we know $\phi$ to be injective. Since it is injective, it extends uniquely to a homomorphism $\phi\colon F(x) \to \Omega$, and so $\tilde{\phi}\colon F(\alpha) \isom F(x) \stackrel{\phi}{\injects} \Omega$ by $\alpha \mapsto x\mapsto \beta$. So we've found an $F$-homomorphism $\tilde{\phi}\colon F(\alpha) \to \Omega$, which means that the map is both injective and surjective and so it's bijective. 
\item Assume that $\alpha$ is algebraic. Note that if $f = \sum a_ix^i$ then consider $0 = f(\alpha)$, and so $\phi(0) = 0 = \phi\qty(\sum a_i \alpha^i)$. Since $\phi$ is an $F$-homomorphism it breaks on addition and multiplication and preserves $a_i$, so $0 = \sum a_i \phi(\alpha)^i$. We can extend this into the statement 
\[ \phi(f(\beta)) = f(\phi(\beta)) \]
for any $\beta$. Then $\phi(\alpha)$ is a root of $f$. As before, we have an injective map between $\Hom_F(F(\alpha), \Omega)$ to the roots of $f$ in $\Omega$ since any $F$-hom is injective, and so $\phi\colon F(\alpha) \to \Omega$ is uniquely determined by where $\alpha$ is sent. To show surjectivity, let $\beta \in \Omega$ be a root of $f$. Now, as before, we must construct an $F$-homomorphism which takes $\alpha$ to $\beta$. Remember that $F(\alpha) \isom F[x]/(f(x))$. Also, look at the map $F[x] \to \Omega$ where $x \mapsto \beta$. Since $\beta$ is a root of the minimal polynomail, this map does have a kernel, whose elements are generated by $(f(x))$. Then by the First Isomorphism Theorem, we know that $F[x]/(f(x)) \isom F(\beta) \subseteq \Omega$, and so it total we have \[\tilde{\phi}\colon F(\alpha) \isom F[x]/(f(x)) \isom F(\beta) \subseteq \Omega,\]
and the composition $\tilde{\phi}$ of $F$-homomoprhisms sends $\alpha$ to $\beta$. \qedhere
\end{enumerate}
\end{proof}

\begin{theorem}[Extension Theorem$'$ (Algebraic Case)]
Let $\psi$ be some embedding from $F$ into $\Omega$ (maybe not an $F$-homomorphism). We say that $\phi\colon F(\alpha) \to \Omega$ extends $\psi$ if $\phi$ is a ring homomorphism and $\phi(c) = \psi(c)$ for all $c \in F$, and the following diagram commutes.
\begin{center}
\begin{tikzcd}
F(\alpha) \arrow[rr, "\phi"] &  & \Omega \\
F \arrow[u, hook] \arrow[rr, "\psi"] &  & \Omega \arrow[u, no head, xshift=-0.3ex] \arrow[u, no head, xshift=0.3ex]
\end{tikzcd}
\end{center}
\end{theorem}

\begin{corollary}
Any $F$-homomorphism is an extension of the identity map $\operatorname{id}(x)$.
\end{corollary}

\begin{proposition}
Let $f \in F[x]$ be monic, and let $E/F$ be an extension generated by some subset of the roots of $f$; i.e., $E = F(\alpha_1, \dotsc, \alpha_n)$ where $f(\alpha_i) = 0$ for all $i$. Let $\Omega$ be an extension of $F$ where $f$ splits.
\begin{enumerate}[label={\textup{(\alph*)}}]
\item There exist $F$-homomorphisms $\phi_i$ from $E$ to $\Omega$, and $\abs{\phi_i} \leq [E : F]$ with equality when $f$ has distinct roots in $\Omega$.
\item If both $E$ and $\Omega$ are splitting fields of $f$ then there exists an $F$-isomorphism from $E$ to $\Omega$.
\end{enumerate}
\end{proposition}

\begin{proof}
Some observations:
\begin{enumerate}
\item If $\alpha$ is a root of $f$ in $\Omega$ and $m(x) \in F[x]$ is the minimal polynomial of $\alpha$ over $F$, then in fact $m$ divides $f$, since the minimal polynomial generates all polynomials which have $\alpha$ as a root.
\item If you have some subextension between $F$ and $\Omega$, like $F \subseteq L \subseteq \Omega$ where $f = g \cdot h$ in $L[x]$, then the roots of $g$ in $\Omega$ are among the roots of $f$ in $\Omega$, and so $g$ must split in $\Omega$ as well.
\end{enumerate}
We'll use these several times. 
\begin{enumerate}[label={\textup{(\alph*)}}]
\item  First write $E$ as $F(\alpha_1, \dotsc, \alpha_n)$. Let $f_1 \in F[x]$ be the minimal polynomial of $\alpha_1$, and so by Observation 1 we know that $f_1 \mid f$, and $f_1$ splits in $\Omega$ by Observation 2, and the roots of $f_1$ are distinct if $f$ has distinct roots. By the extension theorem, there exists an $F$-homomoprhism $\phi_1\colon F(\alpha_1) \to \Omega$, and the number of such $\phi_1$ is bounded by the degree of $f_1$ which is $[F(\alpha) : F]$. Since each subsequent extension is simple over the previous, we can construct the following diagram of extensions.
\begin{center}
\begin{tikzcd}
E \arrow[rr, "\phi" description] &  & \Omega \\
 &  &  \\
{F(\alpha_1,\alpha_2)} \arrow[uu, no head, dotted] \arrow[rruu, "\phi_2" description] &  &  \\
F(\alpha) \arrow[u, no head] \arrow[rruuu, "\phi_1" description] &  &  \\
F \arrow[u, no head] &  & 
\end{tikzcd}
\end{center}
\end{enumerate}
\end{proof}