\documentclass{lnotes}

% Bibliography
\usepackage[
	style = alphabetic ,
]{biblatex}
\addbibresource{references.bib} 

% Packages
\usepackage{epigraph}
\usepackage{wasysym}
\usepackage[osf]{libertineRoman}
\usepackage{tikz-cd}

% Document Information
\title{Introduction to Abstract~Algebra}
\course{\textsc{math} 350}
\place{Yale University}
\term{Fall}
\year{2018}

\blurb{
	These are lecture notes for \textsc{math} 350a, ``Introduction to Abstract Algebra,'' taught by Marketa Havlickova at Yale University during the fall of 2018.
	These notes are not official, and have not been proofread by the instructor for the course.
	These notes live in my lecture notes respository at 
	\[\text{\url{https://github.com/jopetty/lecture-notes/tree/master/MATH-350}.}\] 
	If you find any errors, please open a bug report describing the error, and label it with the course identifier, or open a pull request so I can correct it.
}

\begin{document}

\section*{Syllabus}

\begin{center}
\begin{tabular}{@{}rp{10cm}@{}}
\toprule 
\textbf{Instructor} & Marketa Havlickova, \url{miki.havlickova@yale.edu} \\
\textbf{Lecture} & MWF 10:30--11:20 \textsc{am}, LOM 205 \\
\textbf{Recitation} & TBA \\
\textbf{Textbook} & \fullcite{DF} \\
\textbf{Midterms} & Wednesday, October 10, 2018 \\
& Wednesday, November 14, 2018 \\
\textbf{Final} & Monday, December 17, 2018, 2:00--5:30 \textsc{pm} \\
\bottomrule 
\end{tabular} \\[3ex]
\end{center}

Abstract Algebra is the study of mathematical structures carrying notions of ``multiplication'' and/or ``addition''. Though the rules governing these structures seem familiar from our middle and high school training in algebra, they can manifest themselves in a beautiful variety of different ways. The notion of a group, a structure carrying only multiplication, has its classical origins in the study of roots of polynomial equations and in the study of symmetries of geometrical objects. Today, group theory plays a role in almost all aspects of higher mathematics and has important applications in chemistry, computer science, materials science, physics, and in the modern theory of communications security. The main topics covered will be (finite) group theory, homomorphisms and isomorphism theorems, subgroups and quotient groups, group actions, the Sylow theorems, ring theory, ideals and quotient rings, Euclidean domains, principal ideal domains, unique factorization domains, module theory, and vector space theory. Time permitting, we will investigate other topics. This will be a heavily proof-based course with homework requiring a significant investment of time and thought. The course is essential for all students interested in studying higher mathematics, and it would be helpful for those considering majors such as computer science and theoretical physics.

Your final grade for the course will be determined by
\[ \max\left\{
	\begin{array}{cccc}
		25\%\text{ homework} + 20\%\text{ exam 1} + 20\%\text{ exam 2} + 35\%\text{ final} \\
		25\%\text{ homework} + 10\%\text{ exam 1} + 20\%\text{ exam 2} + 45\%\text{ final} \\
		25\%\text{ homework} + 20\%\text{ exam 1} + 10\%\text{ exam 2} + 45\%\text{ final}
	\end{array}
\right\}. \]

\printbibliography
 
% !TEX root = ../notes.tex

\section{January 14, 2019}

Given an interval $(a,b) \subset \R$, we know that the size of this interval is $b-a$. The focus of this course will be the study of the generalization of this idea using the \emph{Lebesgue measure} on $\R$. Equipped with this, we can talk of the \emph{Lebesgue integral} of ``nice'' functions, which is more powerful than the Riemannian equivalent.

\subsection{The Metric Space}

\begin{definition}[Metric Space]
Given a set $X$, a metric function $d$ is a function $d : X \times X \to \R$ obeying the following three properties.
\begin{enumerate}
\item \textbf{Positivity:} $d(x,y) \geq 0$ and $d(x,y) = 0$ if and only if $x = y$;
\item \textbf{Symmetry:} $d(x,y) = d(y,x)$ for all $x,y \in X$;
\item \textbf{Triangle Inequality:} $d(x,y) \leq d(x,z) + d(z,y)$ for all $x,y,z \in X$. 
\end{enumerate}
A metric space is a pair $(X,d)$ where $d$ is a metric function on $X$.
\end{definition}

\begin{example}[Metric Spaces]
\begin{parts}
\part In $\R$, we have the traditional $d(x,y) = \abs{x-y}$.
\part In $\R^2$, we have $d\qty((x_1,x_2),(y_1,y_2)) = \sqrt{(x_1-y_1)^2 + (x_2-y_2)^2}$.
\part In $\R^2$, we also have $d\qty((x_1,x_2),(y_1,y_2)) = \max\{\abs{x_1-y_2}, \abs{x_2-y_2}\}$.
\part The discrete metric on a set $X$ is defined by 
\[ d(x,y) = \begin{cases}
0 & \text{if $x = y$,} \\
1 & \text{otherwise.}
\end{cases} \]
\part Given a metric space $(X,d)$ and $Y \subset X$ then $(Y,d)$ is also a metric space where $d$ is restricted to $Y \times Y$.
\end{parts}
\end{example}

\begin{definition}[Neighborhood]
Fix a metric space $(X,d)$. For some $r \geq 0$, the $r$-neighborhood of $x$ is $B(x,r)$, the set $\{y \in X \mid d(x,y) < r\}$. Notice that this depends on the metric! In $\R$ with the discrete metric, $B(0,1) = \{0\}$ while $B(0,2) = \R$ which is not what we expect from the traditional metric.
\end{definition}

\begin{definition}[Interior Points]
Let $A \subset X$. A point $x \in A$ is an interior point of $A$ if there exists some $r > 0$ such that $B(x,r) \subset A$. That is, we can draw a ball around $x$ which lies entirely in $A$.
\end{definition}

\begin{example}
If $A = [0,1)$, then the interior points of $A$ are $(0,1)$ but $0$ is not an interior point.
\end{example}

\begin{definition}[Open Sets]
A subset $A \subset X$ is open if every point in $A$ is interior. The empty set is vacuously open.
\end{definition}

\begin{proposition}
For any $x \in X$ the $r$-neighborhood of $x$ is an open subset of $X$.
\end{proposition}

\begin{proof}
Let $y \in B(x,r)$. Let $r_0 = r - d(x,y)$. Then $r_0 > 0$ and $B(y,r_0) \subset B(x,r)$ regardless of which $y$ is chosen since for any $z \in B(y,r_0)$ we know that $d(x,z) \leq d(x,y) + d(r,z) < d(x,y) + r_0 < r$. Then every point of $B(x,r)$ is interior and so it is open.
\end{proof}

Using this, we can now call $B(x,r)$ the open ball of radius $r$ centered at $x$.

\begin{example}
In $\R^2$ with the standard metric, an open ball looks like an open disc. With the maximum metric, it looks like an open square. In $\R$, we can look at the set of all rational numbers $\Q$. This set is not open since for all $q \in \Q$ and all $r > 0$ there exists an $x \in B(q,r)$ where $x \notin \Q$.
\end{example}

\begin{proposition}\label{prop:1-2}
The intersection of finitely many open sets is open. The union of any open sets is open.
\end{proposition}

\begin{example}
The intersection of infinitely many open sets is not necessarily open. Consider $\bigcap\, (0,1/n)$ as $n \to \infty$. The intersection is simply $\{0\}$ which is not an open set.
\end{example}

\begin{proof}[Proof of Proposition~\ref{prop:1-2}]
Let $A_1, \cdots, A_k$ be open subsets of $X$. Let $x \in A_1 \cap \cdots \cap A_k$. Since each $A_i$ is open we know that $x$ is an interior point of $A_i$, so there exists some $r_i$ such that $B(x,r_i) \subset A_i$. Let $r$ be the minimum of all such $r_i$. Then $B(x,r) \subset A_i$ for all $i$, so this open ball is contained in the intersection.

Now let $\{A_\alpha \mid \alpha \in I \}$ be a collection of open subsets. Let $x \in \bigcup\,A_\alpha$. Then $x$ is contained in some open $A_\alpha$, and so there exists an $r_\alpha$ such that $B(x,r_\alpha) \subset A_\alpha$, so $B(x,r_\alpha) \subset \bigcup\,A_\alpha$.
\end{proof}

\begin{definition}[Interior of a Set]
For $A \subset X$, the set of all interior points of $A$ is called the interior of $A$. This is usually written as $\operatorname{Int}(A)$ or $A^\circ$.
\end{definition}

\begin{example}
If $A = [a,b]$ then $A^\circ = (a,b)$. If $A = \Q$ then $\Q^\circ = \emptyset$.
\end{example}

\begin{proposition}
For all $A$, the interior of $A$ is open. Furthermore, $A^\circ$ is the largest open subset of $A$ in the sense that it contains all other open subsets of $A$.
\end{proposition}

\begin{proof}
It's just the definitions.
\end{proof}

\begin{proposition}
If $A \subset B$ then $A^\circ \subset B^\circ$.
\end{proposition}

\begin{corollary}
A set $A$ is open if and only if $A = A^\circ$.
\end{corollary}

\begin{definition}[Limit Point]
Let $A \subset X$. A point $x \in X$ is a limit point of $A$ if for any $r > 0$ we know that $B(x,r) \cap A \not= \emptyset$. Notice that every point $a \in A$ is a limit point of $A$.
\end{definition}

\begin{example}
Let $A = [0,1)$. Then $0$ is a limit point of $A$ since every open ball centered centered at $0$ interesects $A$. Furthermore, $1$ is also a limit point for the same reason. If $A = \Q$, then the set of limit points of $\Q$ is all of $\R$.
\end{example}

\begin{definition}[Closed Set]
A set $A \subset X$ is called closed if every limit point of $A$ is contained in $A$.
\end{definition}

\begin{example}
The interval $[0,1]$ is closed but $[0,1)$ is not. To show that something isn't a limit point, use the minimum distance between this point and the interval. This must be positive since otherwise it would be in the interval. Then let your $r$ be smaller than this, and the open ball with this radius centered at this point will not intersect the original interval. Generalize to higher dimensions as needed.
\end{example}

\begin{corollary}
Given any metric space $X$, we know that $\emptyset$ is closed. Furthermore, $\bar{B}(y,r) = B[y,r] = \{y \mid d(x,y) \leq r \}$ is closed for any $r$.
\end{corollary}

\begin{proposition}
Let $A \subset X$. We know that $A$ is open if and only if $A^\complement$ is closed.
\end{proposition}
% !TEX root = ../notes.tex

\section{January 16, 2019}

\subsection{Review from last time}
Some definitions from last time.

\begin{definition}[Divisibility]
We say that $a$ divides $b$ if $b = ac$ for some $c \in \Z$.
\end{definition}

\begin{definition}[Division Algorithm]
Fix $a$ and $b$. We want to divide $a$ by $b$. Then there exists some unique $q$ and some $0 \leq r \leq a$ such that $b = aq + r$.
\end{definition}

\begin{definition}[Prime]
A number is prime if its only positive divisors are $1$ and itself.
\end{definition}

These are things we learned in grade school.

\begin{theorem}[Well-Ordering Principle]
Every nonempty subset of $\Z_{<0}$ has a least element. This is the defining property of $\Z$.
\end{theorem}

\subsection{Today}

\begin{definition}[GCD]
Let $a,b \in \Z$. The greatest common divisor is the largest common divisor of $a$ and $b$, so $\gcd(a,b) = \max\{d \mid \text{$d$ divides $a$ and $b$}\}$. We know this exists because of well-ordering.
\end{definition}

\begin{definition}[GCD]
Alternatively, the gcd of $a$ and $b$ is a $d$ such that all other common divisors of $a$ and $b$ divide $d$ as well. Eventually we'll prove that these are equivalent.
\end{definition}

\begin{definition}[GCD]
Given $a$ and $b$ in some PID, we say that the GCD is he principal generator $d$ of the ideal $(a,b)$, so $(a,b) = (d)$. Alternatively, the gcd is the smallest positive number in $(a,b)$ if we're working in $\Z$.
\end{definition}

\begin{notation}[GCD]
As a nod to the last definition, we often write the GCD of two numbers as $(a,b)$ to emphasize the relation to ideals.
\end{notation}

Some properties of greatest common divisors:

\begin{lemma}
Let $d$ be the greatest common divisor of $a$ and $b$. Then for any $x \in \Z$ we know that $(a, b+ax) = d$ as well. Then the GCD is unchanged under linear combinations.
\end{lemma}

\begin{proof}
It's clear that $d$ still divides $b+ax$ if it divides $a$ and $b$, so its clear that $(a,b+ax) \geq (a,b)$. Independently, we know that there can't be a larger divisor since if $d'$ divides $b + ax$ then $d'$ divides $b$, and we already know that $d$ is the largest divisor of $b$ which also divides $ax$. Thus $(a,b+ax) \leq (a,b)$ so $(a,b+ax) = d$.
\end{proof}

\begin{lemma}
Let $I = \{ax+by \mid x,y \in \Z\} = (a,b)$. Then $I = \{dx \mid x \in \Z\}$ where $d$ is the greatest common divisor of $a$ and $b$.
\end{lemma}

\begin{proof}
We show containment each way. First we note that $I \subseteq d\Z$ since every element of $I$ is divisible by $d$ since if $d$ divides $a$ and $b$ then it divides $ax+by$. Then we show that $d\Z \subseteq I$ (this is sometimes called Bezout's Lemma). \note{This part could be proved with the Extended Euclidean Algorithm.} By the Well-Ordering property, we know that there exists some $c = \min(I \cap \Z_{>0})$. We know that $c \geq d$ since it must be the case that $d$ divides $c$. On the other hand, if we can show that $c$ is a common divisor of $a$ and $b$ then we know that $c \leq d$ as well. We know that $a = cq + r$ for $0 \leq r \leq c$. Then we know that $c \in I$ implies that $c = ax + by$ so $r = a-cq = a(1-xq) + b(-yq)$ so $r \in I$. Since $c$ is the minimum positive element we know that $c = 0$ and so $a = cq$ so it divides $a$. Repeat for $b$. Then $c \leq d$ and $c \geq d$ so $c=d$. This also gives us the definition of the GCD which is the divisor of $a$ and $b$ which is divisible by all other common divisors.
\end{proof}

\subsection*{Uniqueness of prime factorization}

\begin{lemma}
Let $a$ and $b$ be relatively prime. If $a$ divides $bc$ then $a$ divides $c$.
\end{lemma}

\begin{proof}
Note that $(a,b) = 1$, so there exist some $x,y \in \Z$ such that $1 = ax + by$. Multiplying through by $c$, we get that 
\[ c = cax + cby. \]
Since $a$ divides $cb$ it divides $cby$ and it trivially divides $cax$ so $a$ divides $c$.
\end{proof}

\begin{corollary}
If $p$ is prime and $p$ divides $ab$ then $p$ divides $a$ or $p$ divides $b$.
\end{corollary}

\begin{corollary}
If $p$ divides $\prod a_i$ then for some $i$ we know that $p$ divides $a_i$ (this is the above corollary with induction).
\end{corollary}

\begin{theorem}
All integers have a unique prime factorization. For every $n \in \Z_{\geq 2}$ there exists a unique set of primes $p_1, \cdots, p_k$ and positive integers $a_1 ,\cdots, a_k$ such that $n = \prod_{i=1}^k p_i^{a_i}$.
\end{theorem}

\begin{proof}
Assume that we have two (more than one) such lists of primes and their powers. Denote them $P = p_1, \cdots, p_k$ (possible with repeats) and $Q = q_1, \cdots, q_\ell$. Assyme by way of contradiciton that the lists are disjoint (otherwise we cancel the like terms). We know that $p_1$ divides $\prod_{i=1}^\ell q_i$, so $p_1$ must divide $q_i$ for some $i$. This can happen if and only if $p_1 = q_i$. This contradicts the disjointness of our list and presents a contradiciton.
\end{proof}

\subsection{Before next class}

\begin{enumerate}
\item Read \S1.1 -- \$1.3 in \emph{Ireland and Rosen};
\item Read \S3, \S4.1, and \S4.2 in \emph{Rosen};
\item Think about which textbook is preferred.
\end{enumerate}
% !TEX root = ../notes.tex

\section{Tuesday, January 22}

Recall that there is a ring homomorphism from $\Z[x]$ to $\F_p[x]$ which is reduction modulo $p$ of the coefficients. There is subtlety when we talk about irreducibility in $\Z[x]$; consider that $2x-2 = 2 \cdot (x-1)$ is not irreducible as an element of the ring $\Z[x]$, but it is irreducible as a polynomial in a polynomial ring over a field in the sense that we cannot factor it as $g\cdot h$ where $\partial g, \partial h > 0$. This is why we care about nonconstant factors. This discrepancy occurs when we have $2$ appearing everywhere, which seems to present a problem. We need a way to distinguish between these two conflated notions.

\begin{definition}[Primitive]
A polynomial $f \in \Z[x]$ if its coefficients are \emph{all} relatively prime, so that the ideal $(a_0, \dotsc, a_n)$ is simply $\Z$, or equivalently that there exist $b_0,\dotsc,b_n \in \Z$ such that $\sum b_ia_i = 1$.
\end{definition}

\begin{example}
The polynomial $4x^3 + 6x^2 + 15x + 9$ is primitive while $4x^3 + 6x^2 + 16x + 18$ is not, since every term is divisible by $2$.
\end{example}

\begin{proposition}
A polynomial $f \in \Z[x]$ is primitive if and only if its reduction $\bar{f}$ modulo $p$ is not the zero polynomial for all prime $p$.
\end{proposition}

\begin{lemma}
Let $f \in \Q[x]$. Then there exists a unique rational $c \in \Q$ and a unique primitive polynomial $f_0 \in \Z[x]$ such that $f = c \cdot f_0$.
\end{lemma}

\begin{proof}
We first prove the existence, and then the uniqueness. Existence is given by clearing the denominators in a smart way, first be clearing the denominators and then by factoring out the \textsc{gcd} of the remaining integral coefficients. The uniqueness arises from the following. Suppose that $c \cdot f_0 = c' \cdot f_0'$ where $c,c' \in \Z$ and $f_0,f_0'$ are primitive. Then if we write $f_0 = aa_nx^n + \cdots + a_0$ and $f_0' = a'_nx^n + \cdots + a'_0$ and the original polynomial as $f = A_nx^n + \cdots + A_0$, then $\gcd(A_0, \dotsc, A_n) = \gcd(ca_0, \dotsc, ca_n) = \gcd(c'a_0', \dotsc, c'a_n')$. Since the \textsc{gcd} is multiplicative we know that these are $c \cdot \gcd(a_0, \dotsc, a_n)$ and $c' \cdot \gcd(a_0', \dotsc, a_n')$, and since these polynomials are primitive this \textsc{gcd} is $1$, so $c = c'$.
\end{proof}

\begin{lemma}
If we start with a polynomial in $\Z[x]$, then the $c$ which we pull out will also be an integer.
\end{lemma}

\begin{lemma}
Let $f,g \in \Z[x]$ be primitive. Then $f\cdot g \in \Z[x]$ is also primitive.
\end{lemma}

\begin{proof}
If $f,g$ are primitive then $\bar{f}, \bar{g} \in \F_p$ are nonzero for all $p$. Since $\F_p$ has no zero divisors (integral domain) then we know that $\bar{f} \cdot \bar{g}$ is also nonzero in $\F_p$ for all $p$. Then $fg \in \Z[x]$ is primitive.
\end{proof}

\begin{lemma}
If $f \in \Z[x]$ is primitive and $g \in \Z[x]$ is any polynomial, then if $f$ divides $g$ in $\Q[x]$ then $f$ divides $g$ in $\Z[x]$.
\end{lemma}

\begin{example}[Counter example when not primitive]
Consider that $4x$ divides $x(x-1)$ in $\Q[x]$ but not in $\Z[x]$ because $4x$ is not primitive.
\end{example}

\begin{proof}
There exists some unique way of writing $g$ as $c \cdot g_0$. Since $f$ divides $g$ in $\Q[x]$ then $g = f \cdot h$ for some $h \in \Q[x]$. Thus $f \cdot h = c \cdot g_0$, and we can write $h = d \cdot h_0$. Then $g = d \cdot f \cdot h_0$, and since $f$ and $h_0$ are primitive we know that $fh_0$ is primitive as well. Since this decomosition is unique, this implies that $c = d$ and $g_0 = f h_0$. Then $f$ divides $g_0$ and $g_0$ divides $g$ in $\Z[x]$, so $f$ divides $g$ in $\Z[x]$.
\end{proof}

\begin{lemma}[Gauss]
Let $f \in \Z[x]$. If $f$ is an irreducible polynomial in $\Z[x]$ then $f$ is irreducible in $\Q[x]$.
\end{lemma}

\begin{proof}
Assume $f = gh$ in $\Q[x]$ where $\partial g > 0$. Write $f = bf_0$, $g = cg_0$, and $h = dh_0$. Then $bf_0 = gh = cdg_0h_0$. Then $g_0h_0$ is primitive, so $f_0 = g_0h_0$ and $f = bg_0h_0$.
\end{proof}

\begin{lemma}[Eisenstein's Criterion]
Let $f = \sum a_ix^i \in \Z[x]$. Fix a prime $p$ and assume the following:
\begin{enumerate}
\item $p$ does not divide $a_n$;
\item $p$ divides all $a_i$ where $0 \leq i \leq n-1$; and 
\item $p^2$ does not divide $a_0$.
\end{enumerate}
Then $f$ is irreducible in $\Z[x]$, and by Gauss' Lemma in $\Q[x]$ as well.
\end{lemma}

\begin{proof}
Proof by contradiction. Assume $f$ has the requisite properties but $f$ is reducible in $\Z[x]$, so $f = gh$. Let $g = \sum_{0 \leq i \leq m} g_ix^i$ and $h = \sum_{0 \leq i \leq \ell} h_ix^i$. Consdier $\bar{f} = \bar{a}_nx^n = \bar{g}\bar{h}$. Recall that $a_n = g_mh_\ell$ and $a_0 = g_0h_0$. Then $p$ divides $g_0$ and $h_0$; then $p^2$ divides $g_0h_0 = a_0$ which is a contradiction.
\end{proof}

\begin{example}
Consider that $x^2-p$ is irreducible in $\Q[x]$ by Eisenstein. Thus $\sqrt{p}$ is irrational. And just like that, most of ancient greek mathematics is solved.
\end{example}

\begin{example}
Notice that $x^p - 1 = (x-1) \cdot x^{p-1} + x^{p-2} + \cdots + 1 = (x-1)\Phi_p(x)$. This is irreducible by Eisenstein.

\begin{proof}
Consider $\Phi_p(x+1)$, so $(x+1)^p - 1 = x \cdot \Phi_p(x+1)$. We can use the binomial theorem to say that 
\[ (x+1)^p = \sum {p \choose i}x^i = x^p + px^{p-1} + \cdots + px + 1. \]
Then 
\[ \Phi_p(x+1) = x^{p-1} + px^{p-1} + \cdots + p. \]
Recall that $p$ divides ${p \choose i}$ for all $1 \leq i \leq p-1$. Then $\Phi_p(x+1)$ is Eisenstein, and so $\Phi_p(x+1)$ is irreducible, and so $\Phi_p(x)$ is irreducible.
\end{proof}
\end{example}

\begin{problem}[Challenge, find a better proof than the current one which is terrible]
Prove that $x^n + x + 1 \in \F_2[x]$ is irreducible for $n \geq 2$.
\end{problem}
% !TEX root = ../notes.tex

\section{Thursday, January 24}

\begin{definition}[Extension]
Let $k$ be a field. Then $F \subseteq k$ is a subfield if $F \subseteq k$ is a unital subring and $F$ is a field.
We say that $k$ is an extension of $F$ and write $k/F$.
\end{definition}

\begin{example}
If $k$ is an extension of $F$ then $k$ has the structure of an $F$-vector space. The cannonical example for this is thinking of $\C$ as a two-dimensional $\R$-vector space with basis $\{1, i\}$.
\end{example}

\begin{definition}
An extension $k/F$ is finite if $k$ is a finite dimensional $F$-vector space. The degree of $k/F$, written as $[k : F] = \dim_Fk$ is the $F$ dimension of $k$.
\end{definition}

\begin{example}[Examples of degrees]
\begin{enumerate}
\item $\C/\R$ is finite and $[\C : \R] = 2$.
\item $\R/\Q$ is infinite since $n$-tuples of $\Q$ are countable and $\R$ is uncountable.
\item Let $\Q(i) = \{x + iy \in \C \mid x,y \in \Q\}$. Then $\Q(i)/\Q$ is finite with degree $2$.
\end{enumerate}
\end{example}

\subsection{Field extensions generated by elements}

Let $F$ be a field contained in some field $\Omega$. For $\alpha_1, \dotsc, \alpha_n \in \Omega$ we can consider two objects:
\begin{enumerate}
\item $F[\alpha_1, \dotsc, \alpha_n] \subseteq \Omega$ is a subring
\item $F(\alpha_1, \dotsc, \alpha_n) \subseteq \Omega$ is a subfield
\end{enumerate}

We define these equantities in the following way. 
\[ F[\alpha_1, \dotsc, \alpha_n] = \bigcap_{\substack{R \subseteq \Omega \\ F \subseteq R, \alpha_i \in R}} R = \left\{\sum a_{i_1} \cdots a_{i_n}\alpha_1^{i_1} \cdots \alpha_n^{i_n} \mid a_{i_1,\dotsc,i_n} \in F\right\} \]
and 
\[ 
	F(\alpha_1, \dotsc, \alpha_n) = \bigcap_{\substack{k \subseteq \Omega \\ F \subseteq k, \alpha_i \in k}} k = 
	\left\{
		\frac{\alpha}{\beta} \mid \alpha,\beta \in F[\alpha_1, \dotsc, \alpha_n], \beta \not= 0
	\right\}.
\]
We say that $F[\alpha_1, \dotsc, \alpha_n]$ is the subring of $\Omega$ generated by $\alpha_1, \dotsc, \alpha_n$ and $F(\alpha_1, \dotsc, \alpha_n)$ is the subfield of $\Omega$ generated by $\alpha_1, \dotsc, \alpha_n$.

\begin{example}
Consider $\Q[i]$ and $\Q(i)$. We say that $\Q[i]$ are rational polynomials in $i$ and $\Q(i)$ are quotients of these polynomials, understanding that even powers of $i$ and odd powers of $i$ to collapse it into rational and imaginary components. These are equal to one another. However, $\Q[\pi] \subset \Q(\pi)$ but they are not equal (since the field extension is not finite as $\pi$ is not algebraic).
\end{example}

The Laurent series $F((x))$ is important for Complex Analysis and is analogous to the formal power series $R[[x]]$.

\begin{theorem}[Tower Law]
Let $L/k$ and $k/F$ be field extensions. Then $[L : F] = [L : k] \cdot [k : F]$.
\end{theorem}

\begin{proof}
If $L$ is a finite dimensional $F$-vector space then $L$ is a finite dimensional $k$-vector space since if $z_1,\dotsc,z_p$ is an $F$-basis for $L$ then in turn $\alpha = \sum a_iz_i \in F $ for all $\alpha \in L$ so $z_1,\dotsc,z_p$ is a generating set. On the other hand, $k \subseteq L$ is an $F$-vector space. Assume that $[L : k] = n$ with $x_i$ a $k$-basis for $L$ and $[k : F] = m$ with $y_i$ and $F$-basis for $k$. Then we first show that $\{x_iy_j\}$ is an $F$-basis of $L$ so $[L:F] = nm$. First, linear independence: assume that $\sum a_{i,j}x_iy_j = 0$ for $a_{i,j} \in F$. Rewrite this as $\sum\qty(\sum a_{i,j}y_i)x_i = 0$, so $\sum a_{i,j}y_i = 0$ so $a_{i,j} = 0$ and the set is linearly independent. Next, we show this generates $L$ over $F$. Let $\alpha \in L$ be written as $\alpha = \sum \beta_ix_i$ for $\beta_i \in k$. But for each $i$ we have $\beta_i = \sum a_{i,j}y_i$ so $\alpha = \sum\qty(\sum a_{i,j}y_i)x_i = \sum a_{i,j}x_iy_i$, so these generate. Then this is a basis, and the proposition holds.
\end{proof}
% !TEX root = ../notes.tex

\section{Monday, January 28}

\subsection{Solving \texorpdfstring{$ax \equiv b \pmod{m}$}{ax = b (m)}}

Recall from last lecture that $ax \equiv b \pmod{m}$ is solvable if and only if the \textsc{gcd} divides $m$. If we let $m' = m/d$ then the solutions are unique modulo $m'$.

\begin{proof}
Let $x_1,x_2$ be solutions to $ax_1 \equiv b \pmod{m}$ and $ax_2 \equiv b \pmod{m}$. Consider then that $a(x_1-x_2) \equiv 0 \pmod{m}$. Let $a' = a/d$. Then $da'(x_1-x_2) = dm'k$. We know that $m'$ divides $a'(x_1-x_2)$, and since $(m',a') = 1$ we know that $m'$ divides $x_1-x_2$.
\end{proof}

\begin{corollary}
If $(a,m) = 1$ then there is a  unique solution to $ax \equiv b \pmod{m}$.
\end{corollary}

\begin{corollary}
If $a \not\equiv 0 \pmod{p}$ for prime $p$ then there is a unique solution to $ax \equiv b \pmod{p}$ in $\Z/p\Z$.
\end{corollary}

\subsection*{Chinese Remainder Theorem}

\begin{theorem}[Chinese Remainder Theorem]
If we have $m_1, \dotsc, m_r$ all relatively prime and the system of equations 
\[
	x \equiv a_1 \pmod{m_1}, \dotsc, a \equiv a_r \pmod{m_r},
\]
then there is a unique solution modulo $M = m_1 \cdots m_r$. Alternatively, the rings
\[ \Z/M\Z \cong \bigoplus_{i=1}^r \Z/m_i\Z \]
are isomorphic.
\end{theorem}

\begin{lemma}
If $a_1, \dotsc, a_r$ are pairwise relatively prime to $m$ then the product $a_1 \cdots a_r$ is also relatively prime to $m$ as well.
\end{lemma}

\begin{lemma}
If $a_1, \dotsc, a_r$ all divide $m$ and are all pairwise relatively prime to $m$ then the product $a_1 \cdots a_r$ divides $m$.
\end{lemma}

\begin{proof}[Proof of CRT]
Let $\hat{M}_i = M/m_i = \prod_{j \not= i} m_i$. We find a helper $y_i$ such that $y_i \equiv 0 \pmod{\hat{M}_i}$ and $y_i \equiv 1 \pmod{m_i}$. Then we'll have that $x = \sum a_iy_i$. Note that $(\hat{M_i}, m_i) = 1$ so we know that $1 = x_i\hat{M}_i + y_im_i$ has a solution. Let $y_i = x_i\hat{M}_i$. THis shows existence. To show uniqueness, just apply Lemma~5.2 above.
\end{proof}

\subsection{Algorithmic Speed for the Chinese Remainder Theorem}

The Euclidean Algorithm runs in logarithmic time in the inputs $a,b$. The worst case is when we plug in two consecutive Fibonacci numbers since they are recursively defined in almost the exact opposite way that Euclid's algorithm reduces numbers.
\section{September 12, 2018}

\epigraph{``Oh, I erased my smiley face. How sad.'' (she did not sound sad)}{Miki}

Today we'll officially state something we covered last time.

\begin{theorem}[Caley's Theorem]
Every finite group $G$ is isomorphic to a subgroup of $S_n$ for some $n$.
\end{theorem}
\begin{proof}
Let $n = \abs{G}$.
\end{proof}

\subsection{Kernels}

Let's discuss formally the idea of a kernel of a homomorphism and a kernel of a group action.

\begin{definition}[Kernel]
Let $\phi : G \to H$ be a homomorphism. Then the kernel of $\phi$, written $\ker \phi$, is the set of all elements in $G$ which are mapped to the identity in $H$; i.e., $\ker \phi = \{g \mid \phi(g) = 1_h \}$.
\end{definition}

\begin{definition}
Suppose $G$ acts on $A$ by $\pi$. Then the kernel of the action is the set of all elements of $g$ which act trivially on $A$; i.e., $\ker \pi = \{g \mid ga = a \text{ for all $a \in A$}\}$.
\end{definition}

\begin{example}
Consider the action $\phi : \mathrm{GL}_2(\R) \to (\R^\times, \times) : A \mapsto \det A$. Then the kernel of $\phi$ are all matricies with determinant $1$, called $\mathrm{SL}_2(\R)$.
\end{example}

\begin{definition}[Stabalizer]
Let $\pi : G \times A \to A$ be a group action, and fix $a \in A$. The \emph{stabalizer} is $G_a = \{g \in G \mid ga = a\}$. By this definition, the kernel is contained within any stabalizer, and in fact is equal to the intersection of all stabalizers.
\end{definition}

\begin{example}
Let $G = \mathrm{GL}_2(\R)$ and let $A = \R^2$ defined with the usual action (vector-matrix multiplication). What is the kernel of this action? Then let $c = (0,1)^\top \in \R^2$. What is the stabalizer of $c$?
\end{example}

\begin{corollary}
The kernel of an action is a subgroup of $G$, and $G_a$ is a subgroup of $G$ for any fixed $a \in A$.
\end{corollary}

\begin{definition}[Orbit]
Fix $a \in A$. The orbit of $a$ is the image of $a$ under the group action; i.e., $O_a = \{ga \mid g \in G\}$. Intuitively, it's everywhere $a$ can go under a specific group action. Notice that the orbits partition $A$, and so are equivalence classes in $A$.
\end{definition}

\begin{example}
Let $G = \mathrm{GL}_2(\R)$ and let $A = \R^2$ defined with the usual action (vector-matrix multiplication). What is the orbit of $(1,0)^\top$?
\end{example}

\begin{definition}[Faithful]
An action is faithful if the kernel is the identity. This means that the base element of the action must be the identity. This tells us that $G$ is injective into $S_A$.
\end{definition}

\begin{example}
Consider $D_8$ acting on a square (technically the set $A = \{1,2,3,4\}$). The orbit $O_1$ is all possible vertices, since you can rotate any vertex to any position. The stabalizer is $\{1, s\}$.
\end{example}

\begin{lemma}
As it turns out, for a fixed $a \in A$, we see that $\abs{O_a}\abs{G_a} = \abs{G}$. We'll prove this later. (Orbit-Stabalizer Theorem I think?)
\end{lemma}

\begin{definition}[Conjugation]
Consider the action $\pi : G \times G \to G : (g,a) \mapsto gag^{-1}$. This action is known as \emph{conjugation}.
\end{definition}

\begin{definition}[Centralizer]
Let $S \subset G$. The \emph{centralizer} of $S$ in $G$, written $C_G(S) = \{g \in G \mid gsg^{-1} = s \text{ for all $s \in S$}\}$. This is the set of things that fix $S$ in $G$ pointwise under conjucation. By definition, this is the set of elements in $G$ which commute with all elements in $S$. In the case that $S = \{s\}$ we see that $C_G(S) = G_S$. 
\end{definition}

\begin{definition}[Normalizer]
Let $S \subset G$. The \emph{normalizer} of $S$ in $G$ is $N_G(S) = \{g \in G \mid gSg^{-1} = S\}$. Essentially, this is just a centralizer on a set, except that it may permute the elements of $S$. Then $C_G(S) \subset N_G(S)$.
\end{definition}

\begin{example}
Suppose that $G$ is abelian. For any $S \subset G$, we see that $C_G(S) = N_G(S) = G$.
\end{example}

\begin{example}
Let $G = S_3$, and let $S = G$. What is the normalizer of $S$? (It's the whole thing since $G$ is closed under its operation.) What is the centralizer of $S$? (It's the identity.)
\end{example}
% !TEX root = ../notes.tex

\section{Monday, February 4}

\begin{multicols}{2}
\begin{enumerate}
\item Mind-Body Interaction
\item Pineal gland
\item Intelligibility
\item Princess E.\ 1618--1680
\item Bohemian Rhapsody
\item E.'s Finest Hour
\item Mind-Body Union
\item Malebranche 1638--1715
\item What's the occasion?
\item Not just mind-body, but body-body too
\item No substantial forms
\item Causation + conceptual connection
\item M.'s finest hour
\item Spinoza 1632--1677
\end{enumerate}
\end{multicols}

\subsection{Bohemian Rhapsody}

Descartes was also a scientist in his day, and he conjectured about the physical relation between the mind and the body. He thought that the \emph{pineal gland} was the locus of this interaction. Elisabeth of Bohemia doubted Descartes account of the mind interacts with the body; in particular she wondered how a nonphysical, nonextended thing like the mind could cause interactions with a physical, extended thing like the body. She thought that contact and extension were necessary to induce movement or feeling in a body. Descartes' response was to doubt that these were actually necessary for interaction. He thinks that two things of different natures can interact. This ability is attributed to God, ever the helpful fellow that he is. These causal connections are in our interest, and so that is the reason for their existence. Elisabeth doubts this, and Descartes comes back to say ``Don't think about it too much, it wouldn't make sense for use to have a good idea of the connection anyways.'' Descartes calls this a \emph{primitive notion} which seems like a way for Descartes explain away the flaws in his reason.

\begin{problem}
How does this not violate the principle of sufficient reason?

\begin{solution}[Descarte's Response]
Yeah, it does. Oops. But I only use the PSR to justify God, not to justify anything else.
\end{solution}
\end{problem}

\subsection{Malebranche (1638--1715)}

Malebranche was a Cartesian philosopher who came a bit after Descartes. He was mainly concerned with Theodicy. He believed in the Mind-Body distinction but he had a hard time accepting Descartes' account for how the two interact. Malebranche supposes that, on the occasion of physical interaction with our bodies, a feeling appears in our mind due to God. When we intend to move our body, God moves our body in response to this intention. This divine intervention occurs in a regular fashion. For Malebranche, this is a general theory that also explains body-body interaction; all interactions are resultant from God's will. This theory of intervention by God when certain things happen is called \emph{occasionalism}.

\begin{problem}
Doesn't this just mean that God is the ultimate cause of all suffering and pain?

\begin{proof}[Malebranche's Response]
God is only acting on our intentions; they are the ultimate cause of these things. The only causal power finite objects have is the will to accept the good which God presents to us or to turn away from it.
\end{proof}
\end{problem}

\begin{problem}
What about the phantom limb problem?

\begin{proof}[Malebranche's Response]
God set up the system in the simplest way possible. Extra checks to ensure that the effects are valid are superfluous to the ultimate design of the system.
\end{proof}
\end{problem}

Malebranche, like Descartes, wants to get rid of the  substantial forms of Plato. To do this, they stripped bodies of all causal power and delegate causal power to God, or maybe to the mind in the case of will. For Malebranche, a cause is something which by its nature is connected to the effect.
% !TEX root = ../notes.tex

\section{Wednesday, 7 January}

More constructions.
% !TEX root = ../notes.tex

\section{No Notes}
\section{September 21, 2018}

\epigraph{``Oh I erased my smiley face again. How sad.'' (She did not sound sad.)}{Miki}

\begin{problem}[Warm up]
Draw the lattice diagram for $C_{12}$.
\end{problem}

\begin{figure}[h]
	\begin{center}
	\end{center}
	\caption{Lattice for $C_{12}$}\note{Finish this}
\end{figure}
% !TEX root = ../notes.tex

\section{Monday, 18 February}

Recall the definition of the Jacobi Symbol. This is like the multiplicative extension of the  Legendre Symbol, although we loose the nice property that $(a/n) = 1$ if and only if $a$ is a quadratic residue modulo $n$. We do that the following properties:
\[ \qty(\frac{a}{n})\qty(\frac{b}{n}) = \qty(\frac{ab}{n}) \qquad \text{and} \qquad \qty(\frac{a}{n})\qty(\frac{a}{\ell}) = \qty(\frac{a}{n\ell}). \]
Also recall the theorem of Jacobi Reciprocity, restated here:
\begin{theorem}[Jacobi Reciprocity]
Some facts:
\begin{itemize}
\item $(-1/n) = (-1)^{(n-1)/2}$
\item $(2/n) = (-1)^{(n^2-1)/8}$
\item If $m,n$ odd then $(m/n)(n/m) = (-1)^{(m-1)(n-1)/4}$
\end{itemize}
\end{theorem}

\begin{theorem}
If $a$ is a non-square, there are infinitely many primes such that\/ $(a/p) = -1$, that is where $a$ is not a residue modulo $p$.
\end{theorem}

\begin{proof}
Assume that $a = 2^e \cdot \prod q_i$, where $q_i$ are distinct primes and $e \in \{0,1\}$. We assume here that $a$ is square free, since we can always reduce the exponents modulo $2$ to get rid of this square part. Fix any set of primes $\ell_1, \dotsc, \ell_k$ distinct from $2,q_i$. We want to show that there is a prime $p$ not in this list such that $(a/p) = -1$. We do this by building such a number. By CRT we know there is a $x$ such that $x \cong_8 1 \cong_{\ell_i} 1 \cong_{q_{i < m}} 1 \cong_{q_m} s$, where $(s/q_m) = -1$. Consider that 
\[ \qty(\frac{a}{x}) = \qty(\frac{2^e}{x}) \prod \qty(\frac{q_i}{x}) = 1 \cdot \prod \qty(\frac{x}{q_i}) \cdot (-1)^{(x-1)/2 \cdot (q-1)/2} = \prod \qty(\frac{x}{q_i}). \]
Now, since $x \cong_{q_{i<m}} 1$, we get that 
\[ \prod \qty(\frac{x}{q_i}) = 1^{m-1} \qty(\frac{s}{q_m}) = -1. \]
Then we use the multiplicative nature of the Jacobi symbol to say that 
\[ \qty(\frac{a}{x}) = -1 = \prod \qty(\frac{a}{p_i}) \qquad \text{where~} x = \prod p_i^{v_i}, \]
and we know that $p_i \not= q_i$ since otherwise its congruence modulo $q_i$ would be zero. Since we already know this equals $-1$, there must be \emph{some} (at least one) $p_i$ such that $(a/p_i) = -1$. This is really similar to Euclid's proof of the infinitude of primes.
\end{proof}

Note: The above assumes that $a \not= 2$ since we implicitly assumed there was at least one odd prime factor. If $a = 2$, then it is a nonresidue if and only if $p \cong_8 3,5$. There are infinitely many primes $p \cong_8 3$.

\subsection{RSA Cryptography}

RSA is a public-key cryptography system. Historically, crypto systems has the same encoding and decoding key. An example is something like a cryptogram, like \texttt{XYQ ABCX}, where each letter is a swap for another letter in the alphabet, so \texttt{XYQ ABCX} $\to$ \texttt{THE BOAT}. If you know the bijection $f\colon A \to A$ then you can both encode and decode the message.

\parshape=1
0.1\textwidth 0.8\textwidth
Fun fact, \texttt{BEBOPBOP} is a valid cryptogram for exactly one English word.

Another example is Enigma (yay Alan Turing) from WWII, where the ability to read or send the messages was dependent on a (very high) number of possible dial-combinations, which made it easy to use but computationally difficult to break.

Public Key cryptography sets up a system where anyone can encrypt a message for Alice, but only she may decrypt such a message. This is accomplished by having two different keys. In private, Alice will pick two primes $p,q$ and publicly announces their product $n = pq$. Privately, she can compute $\varphi(n) = (p-1)(q-1)$. She the picks an encryption exponent $e$ and announces this too, and then privately computes $d = e^{-1} \bmod \varphi(n)$. Let's say that Bob wants to send a message to Alice. Suppose this message is some number $P$ between $2$ and $n$ ($1$ fails for obvious reasons). Bob takes $P$ and encrypts is via $C = P^e$ and sends $C$ to Alice. When she receives it, she takes $C^d = P^{e^{d}} \cong P^{e \cdot e^{-1}} \bmod \varphi(n) = P$. If Eve is looking in on this transmission, she can see $C = P^e$ and she even knows what $e$ is! However, given a composite number $n$, it is computationally easy to compute $P^e \bmod n$, but it is nearly intractable to find $P$ given $P^e$. This means that Alice can't really decrypt the message by brute force. Nor can she compute $\varphi(n)$, since it is also very hard to compute $\varphi(n)$ from $n$; we believe it to be as hard as factoring $n$, which is not easy to do\footnote{We think this is the case.}.

\subsection{Diffie-Hellman Key Exchange}

Suppose that Alice has a secret she needs to share with Bob. They don't care what the secret is, but they both need to know it (like an Enigma Key!). It would be bad for Alice to announce it publicly, since anyone could hear it and it's not a secret anymore. One better method is for Alice to take the secret and lock it in a box and send it to Bob. Bob can't open it, but neither can anyone else. However, Bob \emph{can}\/ add his own lock to the box, and send it back to Alice. She then unlocks her lock, and sends the box back to Bob, who can not unlock the last remaining lock and read the secret message without anyone else having read it. This is (more or less) how Diffie-Hellman Key Exchange works.

Alice and Bob agree publicly on a public prime $p$, and some primitive root-ish\footnote{This might be hard to find, so we can find something with a large enough order and just go with that.} $r$. Now they will each privately choose keys $k_A$ and $k_B$, which they don't reveal to anyone. Alice then transmits $c_A = r^{k_A} \bmod p$ and Bob transmits $c_B = r^{k_B} \bmod p$. Alice takes $c_B^{k_A} = r^{k_Ak_B} \bmod p$ and Bob takes $c_A^{k_B} = r^{k_Ak_B} \bmod p$. This is their shared secret. Note however that the secret they end up with $r^{k_Ak_B} \bmod p$ is different than what they started with it, but they both end up with a shared secret.

\subsection{Zero-Knowledge Proofs}

Suppose that Paula knows something (this is good) and she wants to prove that she knows it, but doesn't want to reveal the knowledge. A \emph{Zero-Knowledge Proof} is a protocol whereby she may interact with Vince, the verifier, that she knows this secret.

\begin{example}
Imagine that Vince is color-blind, and cannot tell red from green. Paula has a red sock and a greens sock. When Vince sees these, he can't tell which is which, but Paula wants to prove to Vince that she can distinguish between them. To do this, Paula hands both socks to Vince. In each round, 
\begin{itemize}
\item Vince will produce a sock (he doesn't know which one), shows it to Paula, and puts it behind his back, and then produces a second sock (either $S_1$ or $S_2$) and then asks Paula whether or not it is the same sock. 
\item Paula answers him each time. If she couldn't tell the difference, she would have to guess which sock it was, and so in total she fails with probability $1-0.5^n$ after $n$ rounds. If, however, she \emph{does} see the difference then Vince is confident that she does so with the same probability.
\end{itemize}
Thus Vince can be as sure as he wants to be that Paula can see the colors without every actually learning which sock is which.
\end{example}

\begin{example}
Paula wants to prove her identity to the world. She picks primes $p,q,u$ in private, and announces to the world ``I am Paula! $n=pq$ and $v=u^2$.'' Then Paula is anyone who knows $\sqrt{v} = u$ without showing what $u$ is. To do so, she will
\begin{enumerate}
\item Pick an $r$ at random and sends $x=r^2 \bmod n$ to Vince.
\item Vince receives $x$ and flips a coin. If it is heads, Vince asks, ``Send me $r$.'' If it is tails, Vince asks, ``Send me $r^{-1} \cdot u \bmod n$''.
\end{enumerate}
Paula answers with $A$. Vince verifies that Paula is telling the truth. In the `heads' regime, Vince checks if $A^2 = x \bmod n$. In the `tails' regime, he checks if $A^2x = v \bmod n$.
\end{example}
\section{September 26, 2018}

\epigraph{Let $N$ be a group...I'll call it $N$ suggestively}{Miki}

Recall from last class that we found and example of a non-abelian group and a subgroup for which the left and right cosets of the group were the same; in this case, it was $G = S_3$ and $H = A_3$.

\begin{definition}[Quotient Group]
Let $H \leq G$. The \emph{quotient group} $G/H$ is a group whose elements are the left cosets of $H$. The set for this group is known as the quotient set, and the operation for the group is inherited from $G$ such that $gH \star kH = gk H$. Note that $(gH, \star)$ does not always form a group, so it isn't guaranteed that $G/H$ exists for any $G,H$. 
\end{definition}

\subsection{Mapping from \texorpdfstring{$G$}{G} to \texorpdfstring{$G/H$}{G/H}}

Given a group $G$ and a quotient group $G/H$ we can find a very natural mapping $\pi : G \to G/H$ where $g \mapsto gH$. This map sends elements to their coset, and $\pi(a) = \pi(b)$ if and only if $aH = bH$; thenthe fibers of $\pi$ are the left cosets of $H$, and $\ker \pi = H$. This is why we call it the quotient group --- it's like we're dividing out by $H$. Note that this homomorphism is always going to be surjective since there's no member of $G$ which isn't in some coset of $H$ as they partition $G$.

\begin{definition}[Normal Subgroup]
Let $N \leq G$. Then $N$ is normal if and only if the left and right cosets are the same, so $gN = Ng$. If $N$ is normal then $G/N$ forms a quotient group. Note that this does not mean that $gn = ng$ so $g$ and $n$ do not commute necessarily, but the cosets are preserved. This is equivalent to saying that $\mathrm{N}_N(G) = G$ but $\mathrm{C}_N(G)$ is not necessarily $G$.
\end{definition}

\begin{notation}[$\normal$]
We write $N \normal G$ to mean that $N$ is a normal subgroup of $G$.
\end{notation}

\begin{theorem}
The quotient group $G / N$ exists if and only if $N \normal G$.
\end{theorem}

\begin{proof}[Proof that $N \normal G$ is sufficient]
Observe that $(aN)(bN) = abN$ if $N$ is normal. Then group multiplication is well defined. Observe also that $(aN)^{-1} = a^{-1}N$, so the group is closed under inversion, and by definition our multiplication is associative. Then $G/N$ forms a group if $N$ is normal in $G$.
\end{proof}

\begin{proof}[Proof that $N \normal G$ is necessary]
Suppose $H \leq G$ is not normal. Then there is some $g \in G$ for which $gH \not= Hg$. Then we know that $1HgH \not= gH$, and our group operation $\star$ cannot hold.
\end{proof}

Not that $\abs{G/N} = \abs{G} / \abs{N} = [G : N]$ if $G$ is finite, which we already knew but it's worth remembering it.

\subsection{Testing Normality}

\begin{proposition}
The following are equivalent:
\begin{itemize}
\item $N \normal G$;
\item $gNg^{-1} \subset N$ for all $g \in G$ (note this implies they are equal since conjugation is injective);
\item $N$ is the kernel of some homomorphism $\pi : G \to H$ for some $H \leq G$.
\end{itemize}
\end{proposition}

\begin{proof}[Proof that 1 $\implies$ 2]
Let $g \in G$ and $n \in N$. We know that $gN = Ng$, so there exists $n' \in N$ such that $ng = n'g$. Multiply on the right by $g^{-1}$ and we see that $gng^{-1} = n'$, and so $gng^{-1} \in N$ for all $g,n$.
\end{proof}

\begin{proof}[Proof that 2 $\implies$ 1]
Literally just reverse the above procedure.
\end{proof}

\begin{proof}[Proof that 1 $\implies$ 3]
Let $H = G / N$. Then we know that $\ker \pi = N$ where $\pi : G \to G/N : g \mapsto gN$. Then, rather trivially, we know $N$ is the kernel for some homomorhpism if $N \normal G$.
\end{proof}

\begin{proof}[Proof that 3 $\implies$ 2]
We know that $N = \ker \pi$ for some $\pi : G \to H$. Then take any $g \in G$ and $n \in N$, and consider that $\pi(gng^{-1}) = \pi(g)\pi(n)\pi(g^{-1}) = \pi(g)\pi(g^{-1})$ since $n \in \ker \pi$, and then we conclude that $\pi(g)\pi(g^{-1}) = 1$ and so we know that $gng^{-1} \in \ker\pi$ so $gng^{-1} \in N$ for all $n \in N$ and for all $g \in G$.
\end{proof}
\section{Friday, 28 September 2018}

\epigraph{``I'll leave the cosets for later, where later means 15 seconds from now.''}{Miki}
\epigraph{``Continuous math is not allowed...don't tell anyone I said that.''}{Miki}

Recall Lagrange's Theorem, where if $G$ is a finite group and $H \leq G$ then $\abs{H}$ divides $\abs{G}$; in fact, $\abs{G} / \abs{H} = [G : H]$.

\begin{corollary}
Let $G$ be a finite group and let $x \in G$. Then $\abs{x}$ divides $\abs{G}$ since $x$ generates a cyclic subgroup of order $\abs{x}$, so $\abs{x} = \abs{\langle x \rangle}$ which must divide $\abs{G}$ by Lagrange.
\end{corollary}

\begin{corollary}
If $\abs{G} = p$ is prime, then $\abs{G} \cong Z_p$.
\end{corollary}

\begin{proof}
Since $\abs{G} \not= 1$ there exists $x \in G$ which is not the identity. Then consider $\langle x \rangle$. The order of this cyclic group must divide $p$, and since $p$ is prime it must equal $p$, and so $G = \langle x \rangle$ which means it is isomorphic to $Z_p$.
\end{proof}

If we have some $n \in \Z_{>0}$ where $n$ divides $\abs{G}$ for some $G$, it isn't guaranteed that there exists some $H \leq G$ where $\abs{H} = n$, and/or there isn't always an $x \in G$ where $\abs{x} = n$. For example, consider $G = S_3$ and $n = 6$. However, if prime $p$ divides $\abs{G}$ then there exists an $x \in G$ where $\abs{x} = p$ --- Miki says she will prove this later.

\subsection{Product Subgroups}

Let $G$ be a group and let $H,K \leq K$. Let's consider the product of $HK$, which we recall is defined as \[ HK = \{ hk \mid h \in H, k \in K \}. \]
This may or may not be a subgroup. In general it is not.
\begin{example}
Let $G = S_3$, and let $H = \langle (12) \rangle$ and let $K = \langle (13) \rangle$. Then $HK = \{1, (12), (13), (132)\}$ which is not a subgroup of $S_3$ since $4$ does not divide $6$.
\end{example}
What can we say about $HK$ anyways?
\begin{proposition}
The order of $HK$ is at most $\abs{H}\abs{K}$. In fact, 
\[ \abs{HK} = \frac{\abs{K}\abs{K}}{\abs{H \cap K}}. \]
\end{proposition}

\begin{proof}
We know that $HK$ is the union of left cosets of $K$ where 
\[ HK = \bigcup_{h \in H} hK. \]
Consider $a,b \in H$. We know that $aK = bK$ if and only if $a^{-1}b \in K$ which is true if and only if $a^{-1}b \in K \cap H$. This means that $a K \cap H = b K \cap H$. Then we've reduce the problem to counting the number of distinct cosets $hK$ which is just the index, so it is $\abs{H} / \abs{K \cap H}$. Multiplying through by the size of $K$, we find that 
\[ \abs{HK} = \frac{\abs{H}\abs{K}}{\abs{K \cap H}}.\qedhere \]
\end{proof}

Now we can answer when $HK$ is a subgroup; it happens if and only if $HK = KH$. Intuitively, this happens only when $hkh'k' \in HK$ which can happen if and only if we can commute the $h$ and $k$ elements. It is sufficient to say that $H$ is in the normalizer of $K$ or vice-versa. Another sufficient condition is to say that $K \normal G$, or the other way around. Note that neither of these conditions is necessary. 
\[ H \leq N_G H \implies hK = Kh \implies hk = k'h, \]
but we only need that $hk = k'h'$. That is, we only need that $hK = Kh'$ which is a weaker condition than being in the normalizer.

\subsection{Isomorphism Theorems}

\begin{theorem}[First Isomorphism Theorem]
Given a surjective homomorphism $\phi : G \to H$, we know that $H \cong G/\ker\phi$.
\end{theorem}
\begin{proof}
This was the definition of $G / \ker\phi$, since $\ker\phi \normal G$. See the previous lecture notes for a more in-depth explanation.
\end{proof}

\begin{example}
Consider $\mathrm{GL}_2(\F_3)$ and let $\phi = \det : G \to \F_3^\times$. Then $\ker \phi = \mathrm{SL}_2(\F_3)$, and $\mathrm{GL}_2(\F_3) / \mathrm{SL}_2(\F_3) \cong \F_3^\times$. Since $\mathrm{GL}_2(\F_3)$ has $48$ and $\F_3^\times$ has $2$ elements then we know that $\mathrm{SL}_2(\F_3)$ is of order $2$.
\end{example}

\begin{theorem}[Second Isomorphism Theorem]
Let $G$ be a group with $H,K \leq G$ and let $H \leq N_GK$. Then $HK / L \cong H/H\cap K$.
\end{theorem}

\begin{proof}
We know several things.
\begin{itemize}
\item $HK \leq H$ since $H \leq N_G K$;
\item $K \leq HK$, since we know that $H \leq N_G K$ and $K \leq N_G K$ so $K \leq HK$;
\item Now we can take the quotient $HK/K$, which is the left cosets of $K$ in $HK$. We have shown that $hK = h'K$ if and only if $h H \cap K = h' H \cap K$. Then define the map $\pi : H \to HK/K$ defined by $h \mapsto HK$. This is a homomorphism since $hKh'K = hh'K$ since that's how we defined multiplication. Then $\ker\pi$ is all elements $h$ of $H$ which map to the identity coset which happens if and only if $h \in K$, so $\ker \pi = \{h \in H \cap K\}$. Then by the First Isomorphism Theorem, $H/H \cap K \cong HK/K$. \qedhere
\end{itemize}
\end{proof}

\begin{example}
Let $G = S_3$, let $K = A_3$, and let $H = \langle (12) \rangle$. We know that $HK = S_3$ and $H \cap K = \{e\}$. Then we know that $HK/K = S_3/A_3 \cong \langle (12) \rangle / 1 \cong Z_2$.
\end{example}
\section{Monday, 1 October 2018}

\epigraph{``I've got H's on the brain.''}{}

\epigraph{``That's the third isomorphism theorem, I knew you wouldn't like it. It should take you anywhere from a day to seven years to become comfortable with it.''}{Miki}

\epigraph{``It's math....it keeps doing things like that.''}{Miki}

\subsection{Isomorphism Theorems Continued}

Recall from last lecture we developed the first two isomorphism theorems. Today, we'll cover the last two (or one, depending on your perspective).

\begin{theorem}[Third Isomorphism Theorem]
Let $G$ be a group and let $H,N$ be normal subgroups of $G$ with $N \subseteq H$. THen $G/N \big/ H/N \cong G/H$.
\end{theorem}

\begin{proof}
Consider a map $\phi : G/N \to G/H : gN \mapsto gH$. We need this map to be well-defined. Suppose that $g_1N = g_2N$. Then $g_1^{-1}g_2 \in N$, but $N \subseteq H$, and so $g_1H = g_2H$ and $\phi$ is well defined. We also need to know that this is a homomorphism. Consider $\phi(g_1N)\phi(g_2H) = g_1g_2H = \phi(g_1g_2N)$, and in fact we also know that $\phi$ is surjective. Consider $gH \in G/H$ and suppose that $gH = \phi(gN)$. Since $N \subset H$ this is well defined. Consider then that $\ker \phi : \phi(gN) = gH$. This happens if and only if $g \in H$ so $gN \subset H$ is a coset of $N$ in $H$ and $gN \in H/N$, so $\ker\phi = H/N$. Then by the First Isomorphism Theorem, we know that $G/N \big/ \ker\phi \cong G/H$.
\end{proof}

\begin{example}
Let $G = \Z$ with $N = \langle 10 \rangle$ and $H = \langle 2 \rangle$. Then $G/N = \{0 + N,\dotsc,9+N\}$ and $G/H = \{0 + H, 1 + H\}$. Then $H / N = \{0 + N, 2 + N,\dotsc,8 + N\}$. The idea here is that if you take $\Z \pmod{10}$, and then modulo the result by $2$, then it didn't really matter than we modded out by $10$ to begin with.
\end{example}

\begin{theorem}{The Totally not fourth isomorphism theorem}
Let $N \normal G$. THere is a correspondence (bijection) between subgroups of $G$ which contain $N$ and subgroups of $G/N$. That is,
\[ \pi : H \mapsto \pi(H), \quad \bar{H} \mapsto \pi^{-1}(\bar{H}). \]
Note that for any $H \leq G$ we know that $\pi(H) \leq G/N$. We require normality to ensure that $\pi$ is injective.
\end{theorem}

\begin{example}
Consider $G = S_3$ with $N = A_3$. Then $\pi(S_3) = G/N$ and $\pi(A_3) = N$. What is $\pi(\langle (12) \rangle )$? It's all of $G/N$.
\end{example}

\subsection{Why do people care about normal groups?}

\begin{definition}[Simple]
A group $G$  is simple if $\abs{G} > 1$ and $G$ contains no proper normal subgroups.
\end{definition}

\begin{definition}[Composition Series]
Consdier something like $1 = N_0 \normal N_1 \normal \cdots \normal N_r = G$ where $N_{i+1}/N_i$ is simple for all $0 \leq i \leq r-1$. As an example, $1 \normal A_3 \normal S_3$. Then $S_3/A_3 \cong Z_2$ and $A_3 / 1 \cong Z_3$. These series allow us to construct large groups whose multiplication is unknown, since normal subgroups multiply to form subgroups of something larger. For more information on this, see the \emph{Holder Program}, which was started in 1890 to classify simple groups and it took 103 years to actually classify them all. These series are \emph{almost} unique, where the quotient groups are unique up to a permutaiton, so the set of quotient groups are unique.
\end{definition}

\begin{definition}[Solvable groups]
A group $G$ is solvalble is $1 = N_0 \normal \cdots N_r = G$ and $N_{i+1}/N$ is abelian. This kind of object shows up a lot in Galois Theory. As it turns out, $A_1$ through $A_4$ are solvalble but $A_5$ and higher is not solvable, which is why we can't solve arbitrary quintics.
\end{definition}
\section{Wednesday, 3 October 2018}

\epigraph{``I will not try to decide whether that was happy or sad.''}{Miki}

\epigraph{``Try it if you don't believe me.''}{Miki}

\epigraph{``If you don't have surjectivity, you have nothing.''}{Miki}

Recall from last time that we defined a simple group to be a non-trivial group which has no proper normal subgroups. Observe that if $G$ is abelian and simple then it has no proper subgroups at all, since all subgroups would be normal.

\subsection{Permutations}

We'll take a shortcut throught linear algebra to talk about the signs of permutaitons; the book constructs the notion from scratch. Recall that we can switch the rows of a matrix using the permutation matrix $P_{mn}$, by which left multiplication swaps the rows $m$ and $n$. Now, we talk about this as the cycle $(mn)$, so for example
\[ \begin{pmatrix}
0 & 1 & 0 \\
1 & 0 & 0 \\
0 & 0 & 1
\end{pmatrix} \sim \sigma = (12) \in S_3. \]
Essentially, we start with $I_n$ and permute the rows according to $\sigma$ to yield the corresponding permutation matrix $P_\sigma$.

\begin{definition}[Sign of Permutation]
Let $\varepsilon : S_n \to \{\pm 1\} \cong Z_2$ by $\varepsilon(\sigma) = \det P_\sigma$. Then $\varepsilon$ is the \emph{sign} of $\sigma$.
\end{definition}

Note that $\varepsilon$ is actually a group homomorphism since the determinant is multiplicative; that is $\varepsilon(\tau\sigma) = \det(P_{\tau\sigma}) = \det(P_\tau)\det(P_\sigma) = \varepsilon(\tau)\varepsilon(\sigma)$. Then we can quite naturally ask, what is the kernel of $\varepsilon$. We define the terms \emph{even} and \emph{odd} to mean permutations whose sign is $+1$ and $-1$ respectively. Then $\ker\varepsilon = A_n \leq S_n$ is the set of all even permutations. This gives us a rigorous definition of the alternating group.

Let's note that a two-cycle in $S_n$ is a transposition, and we have already proven on homework that every element in $S_n$ can be written as the product of two-cycles. We can quite easily conclude that every transposition has a sign of $-1$.

\begin{proposition}
Let $\sigma \in S_n$ be a $k$-cycle. Then $\varepsilon(\sigma) = (-1)^{k-1}$.
\end{proposition}

\begin{problem}
How large is $A_n$?
\end{problem}

Since $\varepsilon$ is surjective, we know by the First Isomorphism Theorem that $S_n / A_n \cong Z_2$ (since $\Im\varepsilon = Z_2$), so $\abs{A_n} = n!/2$.

\begin{theorem}
The alternating group on $n$ letters is simple if $n \geq 5$. This was proven by Galois in the 1830's and is the reason for quintic insolubility.
\end{theorem}

\subsection{Actions}

Recall that an action is a map $\phi : G \times A \to A$ by $\phi(g,a) = g \cdot a$. This yields a homomorphism $G \to S_A$ by $g \mapsto \sigma_g$, where $\sigma_g$ is bijective for a fixed $g \in G$. Recall also for $a \in A$ the stabalizer $G_a$ is the set of $g$ for which $ga = a$, and the kernel of the action is the set of $g \in G$ for which $ga = a$ for all $a \in A$. We said that an action is \emph{faithful} if the kernel of the action is the identity; that is, different elements of $g$ give different permutaitons on $A$. Furthermore, the orbit of $a$ is the set of $ga$ for all $g \in G$. We proved on homework that the orbits partition $A$.

\begin{definition}[Transitive]
An action is transitive if all elements of $A$ are in a single orbit; i.e., $a \sim b$ for all $a,b \in A$.
\end{definition}
% !TEX root = ../notes.tex

\section{Tuesday, 26 March 2019}

Today's lecture was just random questions. 
Also, Grace Hopper is Awesome.
% !TEX root = ../notes.tex

\section{Thursday, 28 March 2019}

\subsection{Grace Hopper's Thesis}


\section{Friday, 12 October 2018}

\epigraph{``We have hope. But hope doesn't mean much.''}{Miki}

Let's return to the proposition we described last time, where we said that the equivalency classes in $S_n$ under conjugation are exactly the sets of permutations with the exact same cycle decomposition structure. That is, all elements of the form $(\cdot \cdot \cdot \cdot \cdot) \in S_n$ are conjugates with one another, and the same holds for $(\cdot \cdot \cdot)(\cdot \cdot)$ and all other cycle structures.

\begin{proposition} \hfill
\begin{parts}
\part[prf:cda] If $\sigma \in S_n$ is a $k-cycle$ where $\sigma = (a_1,\cdots,a_k)$, and $\tau \in S_n$, then $\tau\sigma\tau^{-1} = (\tau(a_1), \dotsc, \tau(a_k))$.
\part[prf:cdb] If $\sigma$ is a product of disjoint cycles $\sigma_i \cdots \sigma_r$ then $\tau\sigma\tau^{-1}$ is the product of disjoint cycles $\tau \sigma_i \tau^{-1}$.
\part[prf:cdc] Cycles $\sigma, \sigma'$ are conjugate if and only if they have the same cycle structure.
\end{parts}
\end{proposition}

\begin{proof}[Proof of\/~\ref{prf:cda}]
Let $A = \{1, \dotsc, n\}$ so that $\tau A = A$. Then $A = \{\tau(1), \dotsc, \tau(n)\}$. \note{FINISH THIS}
\end{proof}

\begin{proof}[Proof of\/~\ref{prf:cdb}]
Let $\sigma = \sigma_1 \cdots \sigma_r$. Then $\tau\sigma\tau^{-1}$ can be written as $\tau\sigma_1(\tau^{1}\tau) \cdots (\tau^{1}\tau) \sigma_r \tau^{-1}$, and by associativity the proposition holds. Since the cycles were disjoint to begin with, permuting each $\sigma_i$ under $\tau$ ensure that the products are still disjoint.
\end{proof}

\begin{proof}[Proof of\/~\ref{prf:cdc}]
The forward direction follows immediately from the previous two proofs.
Next, assume $\sigma, \sigma'$ have the same cycle structure. Then \[\sigma = (a_1^1 \cdots a_{k_1}^1)(a_1^2 \cdots a_{k_2}^2) \cdots (a_1^r \cdots a_{k_r}^r),\] and \[\sigma' = (b_1^1 \cdots b_{k_1}^1)(b_1^2 \cdots b_{k_2}^2) \cdots (b_1^r \cdots b_{k_r}^r).\] Then $A = \{1, \dotsc, n\} = \{a_i^j\} = \{b_i^j\}$. Then take $\tau(a_i^j) = b_i^j$, since this is just a permutation on the elements in $A$, so by \ref{prf:cda} and \ref{prf:cdb} this holds.
\end{proof}

\subsection{Proving the simplicity of \texorpdfstring{$A_5$}{A5}}

This is a big deal.

\begin{proof}
We want to show that $A_5$ (or any $A_n$, for that matter) has no proper normal subgroups. Recall the orbit stabalizer theorem, where $\abs{G} = \abs{G_x} \cdot \abs{O_x}$ for any $x \in G$. Recall also that if $N \normal G$ then $N$ is the union of conjugacy classes. Let's start by finding the class equation for $A_5$. Since $A_5$ must have even sign, we know that the only cycles in $A_5$ are of the form $e$, $(\cdot \cdot)$, $(\cdot \cdot \cdot)$, and $(\cdot \cdot \cdot \cdot \cdot)$. Let $O_{x}^{S_5}$ be the orbit of an element $x$ in $S_5$ while $O_{x}^{A_5}$ is the orbit in $A_5$. Note that $\abs{O_x^{A_5}} \leq \abs{O_x^{S_5}}$. Similarly, anything in $A_5$ which fixes $x$ must also fix $x$ in $S_5$ so $\abs{(S_5)_x} \geq \abs{(A_5)_x}$. We also know by Orbit-Stabalizer that $\abs{(A_5)_x} \cdot \abs{O_x^{A_5}} = \abs{A_5} = 60$ while $\abs{(S_5)_x} \cdot \abs{O_x^{S_5}} = \abs{S_5} = 120$. Combining these inequalities with the Orbit-Stabalizer theorem (and recognizing that everything here is an integer), we are left with the option that either the orbits are the same size and the centralizer in $A_5$ is half of the centralizer in $S_5$, or that the centralizers are the same and the orbits in $A_5$ are half that of the orbits in $S_5$.

Let's figure out which of these cases is true. Consider $x = (\cdot \cdot \cdot) = (123) \in S_5$ without a loss of generality. What is the size of the orbit of $x$ in $S_5$? Well, it's all three-cycles, so there are $2 \cdot {5 \choose 3} = 20$ elements in the orbit of $x$ in $S_5$. By Orbit-Stabalizer, the size of the sabalizer is then $120/20 = 6$. Note that $(45) \in (S_5)_x$ since it doesn't move $x$, but because $(45)$ is not in $A_5$ since it has the wrong sign, we know that it is the stablizer which has shrunk and the orbits have the same size.

Let's do the same thing with $x = (\cdot \cdot)(\cdot\cdot) = (12)(34)$ without a loss of generality. Then $\abs{O_x^{S_5}} = {5 \choose 1} \cdot 3 = 15$ elements in the orbit of $x$ in $S_5$. Since this is odd, we know that the orbit can't shrink so it \emph{again} must be the case that the stabalizer has shrunk.

Now let $x = ( \cdot \cdot \cdot \cdot \cdot)$. The orbit of $x$ is then of order $5!/5 = 4!$ while the stabalizer is of order $5!/4! = 5$. In this case, it is now the \emph{orbit} which has shrunk.

Then $\abs{A_5} = 1 + 20 + 15 + 2\cdot 12$ where $20$ comes from the $3$-cycles, $15$ comes from the double $2$-cycles, and the $24$ comes from the two $5$-cycles. Now suppose that $N \normal A_5$. We know it is the union of conjugacy classes and it contains the identity, so $\abs{N} = 1 + \{\text{some of }12,12,15,20\}$, and it must divide $\abs{A_5} = 60$. Note that this can happen \emph{only} if $\abs{N} = 1$ or $\abs{N} = 60$, so $A_5$ contains no proper normal subgroups and is simple.
\end{proof}
\section{Monday, 15 October 2018}

\epigraph{``Perfectly balanced, as all things should be.'' (when referring to left and right actions)}{Miki}
\epigraph{``Our theorem is gone! Oh no!''}{Miki}

\begin{problem}
Does right multiplication define an action of $S_4$ on itself?
\end{problem}
\begin{solution}
No, since we can find two elements for which $g_1(g_2(x)) \not= x \cdot (g_1g_2)$. Consider $(12)$ and $(23)$ acting on the identity. In general, right multiplication is an action if and only if the group is commutative since we are ``switching the order of the multiplication.''
\end{solution}

\subsection{Right Actions}
In order to fix this ``unfairness,'' we often define something called a right action $A \times G \to A$, where the associativity of the action is specified as $a \cdot (gh) = (a \cdot g) \cdot h$. This turns right multiplication into a ``right action.'' There really isn't any distinction between the two, which is why we just speak of ``the action.''

How do we turn left actions into right actions? Suppose we have a left action $g \cdot a$. Define $a \cdot g = g^{-1} \cdot a$; that is, the right action of $g$ on $a$ is just the left action by the inverse of $g$. This works since $(gh)^{-1} = h^{-1}g^{-1}$ so the order follows the rules for right multiplication/action.

\begin{problem}
Consider $\Z$ acting on itself through left addition, where $m \cdot n \mapsto m + n$, and consider that when we turn this into a right action. Then $n \cdot m \mapsto -m + n = n - n$, and we've just invented subtraction.
\end{problem}

\begin{example}
Consider $A_3 \normal S_3$, and consider conjugating $(123) \in A_3$ by something in $S_3$. We know we'll get either another three cycle or the identity, since we know that $gNg^{-1} = N$. Then if $g \in S_3$ there there exists a $\sigma_g : S_3 \to S_3$ which acts on $g$ by conjugation. The consider $\sigma_g |_N$ restricted to acting on $N$. Then we have a map from $N$ to itself. If $g \in N$ then we get the trivial map (since this is just $Z_3$), and otherwise we must not get the trivial map and so $(123) \mapsto (132)$ and \emph{vice versa}. In the latter case, we've created not just a random map but a homomorphism from $N$ to itself. This homomorphism $x \mapsto x^3$ in the group $\langle x \mid x^3 = 1 \rangle$, which is both injective and surjective and we know that this is a homomorphism since the generators satisfy the relations under the map since $x^6 = 1$.
\end{example}

\subsection{Group Automorphisms}

\begin{definition}[Automorphism]
A group automorphism is an isomorphism from $G$ to itself.
\end{definition}

For every group $G$ there is a group $\operatorname{Aut}(G)$ which is the group of all automorphism of $G$ under composition. Miki told us to prove for ourselves that this is actually a group. Now consider $G$ acting on itself through conjugation where $g \mapsto \sigma_g : x \mapsto gxg^{-1}$. For an normal subgroup of $G$ we know that $\sigma_g|_N : N \to N$, and so we have a homomorphism $\psi$ from $G$ to $\operatorname{Aut}(N)$ where $g \mapsto \sigma_g |_N$. The kernel of $\psi$ is the set of all elements in $G$ which commute with $N$, and so $\ker \psi = C_G(N)$. Then $G/C_G(N)$ is isomorphic to a subgroup of $\operatorname{Aut}(N)$ by the First Isomorphism Theorem.

There are two things to unpack here. First, how to we know that $\psi$ is actually a homomorphism? That is, why is $\sigma_g|_N \in \operatorname{Aut}(N)$? Well, consider that $\sigma_g(nn') = gnn'g^{-1} = gng^{-1} \cdot gn'g^{-1} = \sigma_g(n)\sigma_g(n')$, and so $\sigma_g |_N$ preserves the group operation. Next, how to we know that the map $g \mapsto \sigma_g$ is a homomorphism? That is, why does $\sigma_{gg'} = \sigma_{g}\sigma_{g'}$. Well, since conjugation is a well-defined action on $G$, this forms a homomorphism. Note that the restriction to $N$ isn't important here, but the reason we require normality since we won't we able to compose the conjugations since $gHg^{-1} \not= H$.

\begin{corollary}
Take $G = N$. Then we get a homomorphism from $G$ to its own automorphism group, and so $G/C_G(G) = G/Z(G)$ is isomorphism to a subgroup of $\operatorname{Aut}(G)$.
\end{corollary}

\begin{corollary}
Let $H \leq G$ be any subgroup of $G$. Then for all $g \in G$, $gHg^{-1} \cong H$, but they are not necessarily equal to one another.
\end{corollary}

\begin{corollary}
Let $H \leq G$ be any subgroup of $G$. Then $N_G(H) / C_G(H)$ is isomorphic to a subgroup of $\operatorname{Aut}(H)$, since the centralizer is always normal in the normalizer. This is really just a general case of the preceeding statements.
\end{corollary}

\begin{proof}
Since $H \normal N_G(H)$, we just let $G' = N_G(H)$ and apply the result.
\end{proof}
\section{Monday, 22 October 2018}

\epigraph{``You all look so unhappy.''}{Miki}
\epigraph{``$p$ is going to be prime for \emph{at least} two more days.''}{Miki}

\subsection{Clasifying Automorphisms}

Let's talk automorphisms!

\begin{definition}[Inner Automorphism]
Let $g \in G$ and let $\sigma_g : G \to G : x \mapsto gxg^{-1}$ be an automorphism (i.e., an automorphism by conjugation). Then $\sigma_g$ is an \emph{inner automorphism}. The collection of all inner automorphisms forms a group $\Inn(G) \leq \Aut(G)$ which is isomorphic to $G/\mathrm{Z}(G)$ by the first isomorphism theorem.
\end{definition}

\begin{example}
Let $G = \Z/n\Z$. We proved on homework that $\Aut(G) \cong (\Z/n\Z)^\times$, and so any $\sigma \in \Aut(G)$ is uniquely determined by the map which sends $1$ to $a$ for some unit $a$. Since $G$ is commutative, conjugation doesn't really do anything, so $\Inn(G) = \sigma_1$. Put another way, $\mathrm{Z}(G) = G$, so $\Inn(G)$ is as small as it could be.
\end{example}

\begin{example}
Let $G = D_8$. The center of $D_8$ is $\mathrm{Z}(D_8) = \langle r^2 \rangle$. We know that $\Inn(G) \cong G/\langle r^2 \rangle \cong K_4$.
\end{example}

\begin{definition}[Characteristic]
A subgroup $H \leq G$ is \emph{characteristic} if $\sigma(H) = H$ for any $\sigma \in \Aut(G)$. This is like a normal subgroup, except that a normal subgroup need only be preserved under \emph{inner automorphism} while a being characteristic subgroup is a stronger condition.
\end{definition}

\begin{example}
Let $G = D_8$ and let $H = \langle r^2 \rangle$. Since $H$ is the center, this is characteristic (this is true in general). Next let $K = \langle r \rangle \leq G$. Since $\Im(r)$ is either $r$ or $r^3$ (check the order under isomorphism) then $\sigma(\langle r \rangle) = \langle r \rangle$ for any $\sigma \in \Aut(D_8)$ and so it is characteristic.
\end{example}

Just to make the point explicit, if $H$ is characteristic in $G$ then it must be normal in $G$, but the reverse is not true. Additionally, if $H$ is the unique subgroup of a particular order in $G$ then it must be characteristic since there's nothing else it could be sent to under an automorphism since it's image must be a subgroup of the same order.

\subsection{Sylow \texorpdfstring{$p$}{p}-subgroups}

\begin{definition}
Let $p$ be prime. A $p$-subgroup is a subgroupd of order $p^n$ for $n \geq 0$.
\end{definition}

\begin{definition}
Let $\abs{G} = p^am$ where $p$ does not divide $m$. If there is a subgroup of order $p^a$ (there is) then a subgroup of this order is called a Sylow $p$-subgroup. The set of all such groups is written as $\Syl_p(G)$. The number of such groups is written as $n_p(G) = \abs{\Syl_p(G)}$.
\end{definition}

\begin{example}
If $p$ does not divide $\abs{G}$ the the only Sylow $p$-subgroup is the trivial subgroup. If $\abs{G} = p^a$ then the unique Sylow $p$-subgroup is $\Syl_p(G) = \{G\}$.
\end{example}

\begin{example}
Let $G = S_3$ which has order $2 \cdot 3$. Let $p = 2$, $m = 3$, and $a = 1$. Then the largest $\Syl_p$ subgroup is $C_2$, of which there are three such subgroups (things generated by $2$-cycles). If we let $p = 3$, then there is one Sylow $p$-subgroup, generated by a $3$-cycle.
\end{example}

\subsection{Sylow Theorems}

Throughout, let $p$ be prime and let $G$ be a group of order $p^am$ where $a > 0$ and $p$ does not divide $m$.

\begin{theorem}[Sylow I]
There exists a subgroup of $P \leq G$ where $\abs{P} = p^a$.
\end{theorem}

\begin{theorem}[Sylow II]
For each $p$, the Sylow $p$-subgroups are conjugate to one another.
\end{theorem}

\begin{theorem}[Sylow III]
The number of Sylow $p$-subgroups of $G$, written $n_p(G)$, divides $m$ and is congruent to $1 \bmod{p}$.
\end{theorem}

We'll prove these next time (with a lot of chocolate). Today we'll just talk about the implications of these theorems.

\begin{corollary}
There must exist an $x \in G$ whose order is $p$.
\end{corollary}
\begin{proof}
Let $y \in P$ be not the identity. Then $\abs{y} = p$, so for some $0 < b \leq a$ we know that $x = y^{p^{b-1}}$.
\end{proof}

\begin{corollary}
The Sylow $p$-subgroups are all conjugate.
\end{corollary}
\section{Wednesday, 24 October 2018}

Today was a presentation of the proof of the Sylow theorems found in the textbook. As such, notes are omitted in favor of reading the relevant section in the book (and I'm rather tired today and I don't want to type anything up).
\section{Friday, 26 October 2018}

Didn't go to class today! Something about direct products I think.
\section{Monday, 20 October 2018}

\epigraph{``Let's write down all finitely generated abelian groups. What fun.''}{Miki}

We have two goals for today.
\begin{enumerate}
\item Is $Z_{20} \times Z_{18} \cong Z_{36} \times Z_{10}$?
\item How do we classify \emph{all} finitely generatred abelian groups?
\end{enumerate}

To start answering these, we'll begin with a proposition.

\begin{proposition}
$Z_n \times Z_m \cong Z_{mn}$ if and only if $\gcd(m,n) = 1$.
\end{proposition}

\begin{proof}
Let $d = \gcd(m,n)$, and let $Z_m = \langle x \rangle$ and let $Z_n = \langle y \rangle$. Consider $G = Z_m \times Z_n = \{(x^a, y^b)\}$. Consider $(c,f) \in G$. Then $\abs{(c,f)} = \lcm (\abs{c}, \abs{f})$. If $d = 1$ then $\abs{(x,y)} = \lcm (m,n) = mn$, so $Z_{mn} \cong \langle (x,y) \rangle \leq G$. Since the orders are the same, it is isomorphic to the whole thing.
On the other hand, if $d > 1$, let $(c,f) \in G$, and consider $(c,f)^{mn/d} = (c^{mn/d}, f^{mn/d}) = (e,e)$, so every element has order strictly less than $mn$ since $d > 1$. Therefore $G \not\cong Z_{mn}$.
\end{proof}

\begin{example}
Consider $Z_9 \times Z_6 \not\cong Z_{54}$. Note that $Z_9 \times Z_6 \cong Z_9 \times Z_3 \times Z_2 = Z_{18} \times Z_3$.
\end{example}

\begin{example}
Use the proposition we just proved to ``factor'' the groups into the same decomposition. Ta-Da!
\end{example}

fdas

\subsection{Classifying Finitely-Generated Abelian Groups}

\begin{definition}[Free Abelian Group]
Let $\Z^r = \Z \times \cdots \times \Z$ ($r$ times) be the free abelian group of rank $r$.
\end{definition}

\begin{theorem}[Classification Theorem for Finitely Genreated Abelian Groups]
Let $G$ be a finitely generated abelian group. Then there is a unique decomposition of $G$ satisfying
\begin{enumerate}
\item $G \cong \Z^r \times Z_{n_1} \times \cdots Z_{n_s}$ for $r,n_i \in \Z$,
\item $n_i > 2$ for all $i$, and 
\item $n_{i+1}$ must divide $n_i$ for all $1 \leq i \leq s-1$.
\end{enumerate}
\end{theorem}

\subsection{Classifying Finitely-Generated Abelian Groups 2, Electric Boogaloo}

\begin{example}
Consider $Z_{60} \cong Z_{2^2} \times Z_{3} \times Z_{5}$. Notice now that all the components are $p$-subgroups.
\end{example}

\begin{theorem}
Let $\abs{G} = n = \prod p_i^{a_i}$, where $a_i \geq 1$. Then we can write $G$ uniquely (up to order of primes) as $G \cong A_1 \times \cdots \times A_k$ where $\abs{A_i} = p_i^{a_i}$, and for all $A = A_i$ where $\abs{A} = p^a$, we know that $A \cong Z_{p^{b_1}} \times \cdots \times Z_{p^{b_\ell}}$ where $b_1 \geq b_2 \geq \cdots \geq b_\ell$, where the sum of all $b_i$ is $a$.
\end{theorem}

\section{Wednesday, 31 October 2018}

\epigraph{``Oh. You are a gamer.''}{Miki}

\subsection{FGAGs for Dayyyyysssss}

Miki started us off with some exercises to apply what we learned last lecture about classifying finite abelian groups.

\begin{exercise}
Conver $G = Z_{36} \times Z_{12}$ into elementary divisor notation.

\begin{solution}
$G \cong (Z_{2^2} \times Z_{3^2}) \times (Z_{2^2} \times Z_3) \cong (Z_{2^2} \times Z_{2^2}) \times (Z_{3^2} \times Z_3)$.
\end{solution}
\end{exercise}

\begin{exercise}
Convert $(Z_{16} \times Z_{4} \times Z_2) \times (Z_9 \times Z_3)$ into invariant factor notation.

\begin{solution}
Group up the $i$\textsuperscript{th} terms in each parenthetical term, so $G \cong (Z^{16} \times Z_9) \times (Z_4 \times Z_3) \times Z_2 \cong Z_{144} \times Z_{12} \times Z_{2}$. 
\end{solution}
\end{exercise}

\begin{exercise}
Classify all abelian groups of order $24$.

\begin{solution}
Let's use the invariant factor notaion. If $p$ divides the order of $\abs{G}$ then $p$ divides $n_1$. The factorization of $24$ is $2^3 \times 3$. Then $n_1$ could be $24$, and $G_1 \cong Z_{24}$. It could be that $n_1 = 4 \cdot 3$, so $n_2 = 2$ and $G_{2} \cong Z_{12} \times Z_2$. It could be that $n_1 = 2 \cdot 3$, so $n_2 = 2$ and $n_3 = 2$, (can't be $n_2 = 4$ since $4$ doesn't divide $6$), so $G_3 \cong Z_6 \times Z_2 \times Z_2$.
\end{solution}

\begin{solution}
Let's use the elementary divisor notation. Let $\abs{H} = 24$. Then $H \cong A_1 \times A_2$ where $\abs{A_1} = 2^3$ and $\abs{A_3} = 3$. Then $A_2 \cong Z_3$. For $A_1$, we must take all non-increasing partitions of $3$, so $3 = 3, 2 + 1, 1+1+1$. The the possibilities for $A_1$ are $Z_{2^3}$, $Z_{2^2} \times Z_2$, and $Z_2 \times Z_2 \times Z_2$. Then $H$ is either $Z_3 \times Z^{2^4} \cong Z_24$ or $Z_3 \times Z_{2^2} \times Z_2 \cong Z_{12} \times Z_2$ or $Z_3 \times Z_2 \times Z_2 \times Z_2 \cong Z_6 \times Z_2 \times Z_2$.
\end{solution}
\end{exercise}

\subsection{The Shape of Things to Come}

Over the next few lectures, we'll cover how to take the product of groups which aren't abelian, and understanding how we can ``factor'' non abelian groups in the same way that we now know how to factor abelian groups.
Warning: it'll be the hardest single thing we do in this class.

\subsection{Commutators}

Let $G$ be a group with $x,y \in G$. We defined the \emph{commutator} to be $[x,y] = xyx^{-1}y^{-1}$. The commutator of any two elements of $G$ is one if and only if $xy = yx$. The commutator subgroup $G' = \langle [x,y] \mid x,y \in G \rangle$, which is normal in $G$ and the quotient $G/G'$ is abelian.

Suppose we had a homomorphism $\phi$ from $G$ to $H$, where $H$ is abelian. Then it must be that $G' \leq \ker \phi$, since $\phi(x)\phi(y) = \phi(y)\phi(x)$, so $[\phi(x),\phi(y)] = \phi([x,y])$ for all $x,y \in G$.

The quotient of $G$ by $G'$ is the largest abelian quotient of $G$, which means that the commutator group is the smallest subgroup for which the quotient is abelian.

\begin{example}
Let $G = S_3$. What is $G'$? Given the sign map $\pi : S_3 \to Z_2$, we know that $\ker \pi = A_3$, so we know that $G' \leq A_3$ since $Z_2$ is commutative. Then $G' = A_3$.
\end{example}

\begin{example}
Let $G = D_{12}$. Is it a direct product of some proper subgroups? Consider $K = \langle r^3 \rangle \leq Z(G)$ and let $H = \langle s, r^2 \rangle$. Notice that $\langle H,K \rangle \leq G$ and $\langle H,K \rangle$ contains $s$ and $r$, so it is equal to the whole group. It's also true that $H$ and $K$ commute with one another. As it turns out (note quite a proof yet) $G \cong K \times H$. 
\end{example}

\begin{theorem}
Let $G$ be a group and let $H,K$ be subgroups of $G$ satisfying
\begin{enumerate}
\item $H \normal G$ and $K \normal G$, 
\item $H \cap K = \{e\}$, and 
\item $\langle H,K \rangle = HK = G$.
\end{enumerate}
Then $G \cong H \times K$.
\end{theorem}

\begin{proof}
First, we show that $H$ and $K$ commute with one another. Consdier $[h,k]$. Notice that $(hkh^{-1})k^{-1} \in K$ and  $h(kh^{-1}k^{-1}) \in H$, so it's in both $H$ and $K$, but the intersection of $H$ and $K$ is $\{e\}$ which means that the commutator is the identity, which happens if and only if $H$ and $K$ commute with one another. Next, consider that $\abs{HK} = \abs{H}\cdot\abs{K}/ \abs{\{e\}} = \abs{H} \cdot \abs{K}$. Next, create a map $\phi : H \times K \to G$ where $(h,k) \mapsto hk$. We show that this is a homomorphism (and an isomorphism). Notice that $\phi(h,k)\phi(h',k') = hh'kk' = \phi(hh',kk')$ so $\phi$ is a homormphism. It is also injective. Suppose that $\phi(hk) = 1$, which tells us that $h = k^{-1}$, which means that $h = k = 1$ since $h \in H \cap K$. It is surjective, since the sizes of the groups are the same. Then $G \cong H \times K$ through $\phi$.
\end{proof}



\section{Friday, 2 November 2018}

\epigraph{``You're not going to like this.''}{Miki}

\subsection{Semidirect Products}
We want to generalize our understanding of the direct product to non-commutativity.

\begin{example}
Let $G = D_6$, and let $H = \langle r  \rangle \cong Z_3$ and let $K = \langle s \rangle \cong Z_2$. Notice $H \normal G$. Since their intersection is trivial, $HK = G$. Here, we know that $G \not\cong H \times K$ (since then it would be abelian) but it's really similar, since we can write any $g$ as a product of $h$ and $k$.
\end{example}

We kinda cheated with this example since we already know how $r$ and $s$ relate to one another, but what if we don't know how to ``conjugate'' them? What if we don't know how to multiply words?

\begin{definition}[Semidirect Product]
Let $H$ and $K$ be groups, and let $\phi : K \to \Aut(H)$ be a homomorphism (aside from the trivial one, this may not exist). Then $\phi$ defines an action of $K$ on $H$ where $k \cdot h = \phi(k)(h)$. Then $G = H \rtimes_\phi K$ is called the semidirect product, where $g \in G = (h,k)$ and $(h_1,k_1)(h_2,k_2) = (h_1(k_1 \cdot h_2), k_1,k_2)$. This is a generalization of conjugation.
\end{definition}

There are some important properties of this. First, $G$ is a group where $\abs{G} = \abs{H} \cdot \abs{K}$. Second, $H \leq G$ via $h \mapsto (h,1)$ and $K \leq G$ via $k \mapsto (1,k)$. Third, $H \cong \langle (h,1) \rangle \normal G$. Fourth, $H \cap K = \{e = (1,1)\}$. Finally, for all $k \in K$ and $h \in H$ we know that $k \cdot h = khk^{-1}$.

\begin{example}
Let $K = \langle k \rangle \cong Z_2$ and $H = \langle h \rangle  \cong Z_3$. What could $H \rtimes K$ be? First, let's find $\Aut(H) \cong (\Z/(3))^\times$, so there are only two possible choices for our defining homorphism; it can be either the identity, or one other map. Then let $\phi : k \mapsto e$, so $h \cdot h = h$ and the groups commute, so $H \rtimes_\phi K \cong H \times K$. Or, we could take that $\psi : k \mapsto \chi$ where $k \cdot h = h^2$ and where $\chi$ is the other element of $\Aut(H)$, and so $H \rtimes_\varphi K \cong S_3$. Notice that $H \rtimes_\phi K \not\cong H \rtimes_\varphi K$.
\end{example}

\begin{example}
What if we keep $H$ and $K$ but make $H \cong Z_2$ and $K \cong Z_3$ (i.e., we flip the group order)? Well, $k \cdot 1 = 1$ and then, by exhaustion, $k \cdot h = h$ (since that's all we can do with a homomorphism), so $H \rtimes K \cong H \times K$. This tells us that $H \rtimes K$ is, in general, not the same thing as $K \rtimes H$.
\end{example}
\section{Monday, 5 November 2018}

\epigraph{``Now everybody is happy. Or not, but it works.''}{Miki}

\subsection{More Semidirect Products}

A continuation of the discussion from last lecture.

\begin{example}
Suppose $H \cong Z_{20} \times Z_{45}$ (or any abelian group $\abs{H} > 2$). Since $H$ is abelian we know that the map $\gamma \in \Aut(H) : x \mapsto x^{-1}$ is a homomorphism. Then $\gamma$ is of order two. Let's take some semidirect products. Consider 
\begin{parts}
\part[prt:sd:a] $H \rtimes Z_2$ where $\phi(k) = \gamma$. Then $k \cdot h = h^{-1}$ for all $h \in H$.
\part[prt:sd:b] $H \rtimes Z_4$. Here we could also send $k$ to $\gamma$ since $\gamma^4 = 1$. Then $\phi(k^2) = 1$, so $k \cdot h = h^{-1}$, and $k^2 \cdot h = h$. This sends $Z_4$ to $Z_4 / \langle k^2 \rangle \cong Z_2$ to $\langle \gamma \rangle$.
\part[prt:sd:c] $H \rtimes S_3$. We can quotient $S_3$ by $A_3$, which is isomorphic to $Z_2$, and then we map this into $\Aut(H)$. Composition of these maps yields a nontrivial homomorphism $\phi$. Here, $\psi : S_3 \to S_3/A_3$ is the sign map, so for all $\sigma in \S_3$, we say that $\psi :\sigma \mapsto \gamma$ if the sign of sigma is $-1$, and $\sigma \mapsto 1$ if the sign of sigma is positive. 
\end{parts}
\end{example}

\subsection{Identitfying Groups as Semidirect Products}

Given a group $G$, how can we tell what its semidirect product decomposition could be?

\begin{theorem}
Let $G$ be a group with $H,K \leq G$ such that 
\begin{enumerate}
\item $H \normal G$,
\item $H \cap K = \{1\}$, and 
\item $G = HK$.
\end{enumerate}
Then $G \cong H \rtimes K$ where $k \cdot h = khk^{-1}$ for all $h \in H$ and for all $k \in K$.
\end{theorem}

This is really similar to the requirements for $G$ being the \emph{direct product} of $H$ and $K$; we just drop the requirement that $K \normal G$.

\begin{proof}
We know that $H \cap K = \{1\}$ so $\abs{H}\abs{K} = \abs{G}$, so for every $g \in G$ there is a unique way of writing it as $hk$. Then we can create a well-defined map $HK = G \to H \rtimes K$ via $\pi : g \mapsto (h,k)$. To show that this map is onto, observe that we can get all elements of the form $(1,k)$ and $(h,1)$, so any $(h,k)$ is in the image of $hk$ under our map. We know that $\abs{HK} = \abs{H \rtimes K}$ so they are isomorphic if we can show that $\pi$ is a homomorphism (turns out, it is). Proof of this is left as an exercise to the reader.
\end{proof}

\subsection{Classification}

Let $\abs{G} = pq$ where $p \leq q$. We showed with the Sylow theorems that $P \in \Syl_p(G)$ and $Q \in \Syl_q(G)$ imply that $Q \normal G$. We also showed that if $p$ doesn't divide $q - 1$ then $n_p = 1$ and $P \normal G$. In these cases, $G \cong Z_{pq}$. What happens if $p$ \emph{does} divide $q-1$? In this case we don't need to create a $Z_{pq}$, and this is where semidirect products come in. Here, we can apply the previous theorem to construct groups of the form $Q \rtimes_\phi P$ for some $\phi$.

Let's look at $\Aut(Q)$, which we know is isomorphic to $(\Z_q)^\times$ so the order is $q-1$. We also know\footnote{since $q$ is prime} that $\Aut(Q)$ is cyclic, so $\Aut(Q) \cong Z_{q-1}$. We know that $p$ divides $q-1$, so there is a unique subgroup $\langle \gamma \rangle \leq \Aut(Q)$ which is isomorphic to $Z_p$. We need to map a generator of $P$ into $\Aut(Q)$, so it must have order $1$ or $p$. Let $P = \langle x \rangle$. We have a few options. We know that $\phi(x) = \gamma^i$ for some $0 \leq i \leq p-1$ since we need $\phi(x)^p = 1$.
\begin{enumerate}
\item The case where $i = 0$, wherein have the trivial action. Then $P$ and $Q$ commute one another since $x \cdot y = y$, and so $G \cong Q \times P$.

\item The case where $i = 1$. Then $G \cong Q \rtimes_{\phi_1} R$ where $x \cdot y = \gamma(y)$. This involves mapping $x$ to $\gamma$. This semidirect product is nonabelian. What exactly $\gamma$ is depends on $P$ and $Q$, but we konw that it does exist since $p$ divides $q-1$, and so a $p$-order subgroup of $\Aut(Q)$ must exist.

\item The case where $i > 1$. We know that $x \mapsto \gamma^i$ since $P \cong \langle \gamma \rangle$. Then call $x'$ the element which is mapped to $\gamma$. Then we revert to the case where we defied $\phi_1$ by $x'$, so all cases where $i \not= 0$ are isomorphic, and the exact value depends on which $x'$ you choose. 
\end{enumerate}
\section{Wednesday, 7 November 2018}

\subsection{Classify Groups of Order \texorpdfstring{$12$}{12}}
Online. Look it up.

\subsection{``A Simple Song''}

This may be the best thing I have ever seen. Lyrics to follow shortly.

\begin{raggedright}
\raggedright\itshape\small
What are the orders of all simple groups? \\
I speak of the honest ones, not of the loops. \\
It seems that old Burnside their orders has guessed. \\
Except for the cyclic ones, even the rest. \\ \hspace{1em} \\

Groups made up with permutes will produce
some more: \\
For A\textsubscript{n} is simple, if n exceeds 4. \\
Then, there was Sir Matthew who came into
view \\
Exhibiting groups of an order quite new. \\ \hspace{1em} \\

Still others have come on to study this thing. \\
Of Artin and Chevalley now we shall sing. \\
With matrices finite they made quite a list. \\
The question is: Could there be others they've missed? \\ \hspace{1em} \\

Suzuki and Ree then maintained it's the case \\
That these methods had not reached the end of
the chase. \\
They wrote down some matrices, just four by
four, \\
That made up a simple group. Why not make
more? \\ \hspace{1em} \\

And then came the opus of Thompson and Feit, \\
Which shed on the problem remarkable light. \\
A group, when the order won't factor by two, \\
Is cyclic or solvable. That's what is true. \\ \hspace{1em} \\

Suzuki and Ree had caused eyebrows to raise, \\
But the theoreticians they just couldn't faze. \\
Their groups were not new: if you added a twist, \\
You could get them from old ones with a flick of the wrist. \\ \hspace{1em} \\

Still, some hardy souls felt a thorn in their side. \\
For the five groups of Mathieu all reason defied; \\
Not $A_n$, not twisted, and not Chevalley, \\
They called them sporadic and filed them away. \\ \hspace{1em} \\

Are Mathieu groups creatures of heaven or hell? \\
Zvonimir Janko determined to tell. \\
He found out what nobody wanted to know: \\
The masters had missed 1 7 5 5 6 0. \\ \hspace{1em} \\

The floodgates were opened! New groups were the rage! \\
(And twelve or more sprouted, to greet the new age.) \\
By Janko and Conway and Fischer and Held \\
McLaughlin, Suzuki, and Higman, and Sims. \\ \hspace{1em} \\

No doubt you noted the last lines don't rhyme. \\
Well, that is, quite simply, a sign of the time. \\
There's chaos, not order, among simple groups; \\
And maybe we'd better go back to the loops
\end{raggedright}

\subsection{Introductions to Rings}

\begin{definition}[Ring]
A \emph{ring} is a tuple $(R,+,\times)$ where $(R,+)$ is an abelian group, $\times$ is an associative binary operaation, and the distributive laws hold.
\end{definition}
\section{Friday, 9 November 2018}

\epigraph{``I'm an advertisement for Czech people.''}{Miki}

\subsection{Warmups}

\begin{exercise}
Are the following objects rings?
\begin{parts}
\part The set of all even integers with the operations you expect.
\part The set of all odd integers with the operations you expect.
\end{parts}
\end{exercise}

\begin{solution}
The even integers do form a ring (we don't require a ring to have identity) while the odd integers don't (there is no additive identity and it isn't even closed).
\end{solution}

\subsection{Quaternions, Division Rings, and Fields, Oh My!}

\begin{exercise}
What is $\C$?
\end{exercise}

\begin{solution}
We can think of it as a vector space over $\R$ with a basis $\{1,i\}$, which gives us how to multiply and add things in $\C$ if we keep in mind that $i^2 = 1$.
\end{solution}

We can extend this idea of $\C$ into a higher-dimensional object $\mathbb{H}$, known as the \emph{Hamiltonian quaternions}. This forms a vector space over $\R$ with a basis $\{1,i,j,k\}$ where the ring multiplication is defined by the group $Q_8$. Some ``fun'' facts about $\mathbb{H}$:
\begin{enumerate}
\item $\mathbb{H}$ is not commutative;
\item $\mathbb{H}$ is a ring with identity. For any $a + bi + cj + dk$, the inverse is $(a - bi - cj - dk)/(a^2+b^2+c^2+d^c)$.
\end{enumerate}

We can extend these definitions even further.

\begin{definition}[Division Ring]
A ring $R$ with identity $1$ is a \emph{division ring} if for all nonzero $a \in R$ there exists an $a^{-1} \in R$
\end{definition}

\begin{definition}[Field]
A \emph{field} is a commutative division ring.
\end{definition}

About all rings we can make some claims.

\begin{proposition}
Let $R$ be a ring. Then
\begin{enumerate}
\item $0a = a0 = 0$ for all $a \in R$;
\item $(-a)b = a(-b) = -(ab)$ for all $a,b \in R$;
\item $(-a)(-b) = ab$ for all $a,b \in R$;
\item If $R$ has an identity the it is unique and satisfies $(-a)1 = 1(-a) = -a$ for all $a \in R$.
\end{enumerate}
\end{proposition}

\begin{proof}
If you took 230, you've already done this. If not, it's a good if somewhat tedious exercise.
\end{proof}

\begin{definition}[Zero Divisor]
An nonzero $a \in R$ is a zero divisor if there exists a nonzero $b \in R$ for which $ab = 0$ or $ba = 0$.
\end{definition}

\begin{definition}[Unit]
Let $R$ be a ring with identity. A unit is any element $x \in R$ if there exists a $u = x^{-1} \in R$ for which $xu = ux = 1$. We denote the set of units of $R$ by $R^\times$.
\end{definition}

\begin{example}
Consider $R = \Z/6\Z$. Find all units and zero divisors.
\end{example}

\begin{solution}
The units are $1,5$, and the zero divisors are $2,3,4$.
\end{solution}

\begin{proof}[Proof that $2$ is not a unit]
Suppose by way of contradiction that there exist $x \in R$ such that $2x = 1$. Then $3(2x) = 3 = (3 \cdot 2)x = 0$, which is a contradiction. Generally, if $x \in R$ is a zero divisor then it is not a unit. However, the converse is not true (consider $2 \in \Z$, which is neither a unit nor a zero divisor).
\end{proof}

\begin{definition}[Integral Domain]
A commutative ring $R$ with identity is an integral domain if it has no zero divisors. This gives us the cancellation law! [The proof is super short and intuitive.]
\end{definition}

\begin{proposition}
Any finite integral domain is a field. The proof is pretty chill. Check it out some time!
\end{proposition}

\begin{definition}[Subring]
A subring $S \subseteq R$ is an object where $(S,+) \leq (R,+)$ and $S$ is closed under multiplication. For example, $2\Z \subset \Z$, $\Z \subset \Q$, $\Q \subset \R$, $\R \subset \C$, the set of continuous functions from $\R$ to $\R$ is a subring of all functions from $\R$ to $\R$.
\end{definition}

\begin{example}
Consider $\Q(i) = \{a + bi \mid a,b \in \Q\}$ where $i^2 = 1$. We know how to add and multiply this from middle school. This is actually a field equal to $\Q[i]$, where $\Q[i]$ is actually the field of rational polynomials over $1$ and $i$. Now consider $\Z[i]$ (this won't be a field), which is integral polynomials in $1$ and $i$. This isn't a field since $2$ has no inverse.
\end{example}

\begin{definition}[Norm]
Let the norm of $\Q(i)$ be a map $N : \Q(i) \to \Q$ defined by $a + bi \mapsto (a+bi)(a-bi) = a^2+b^2$. This map is multiplicative.
\end{definition}

\begin{proposition}
$N(x) \in \Z$ for all $x \in \Z[i]$, and $x$ is a unit if and only if $N(x) = \pm 1$.
\end{proposition}
\section{Monday, 12 November 2018}

\begin{example}
Consider $\Z[\sqrt{5}] \subset \Q(\sqrt{5})$ where $N(a+b\sqrt{5}) = a^2 + 5b^2$. Then $(a+b\sqrt{5})^{-1} = (a-b\sqrt{5})/(a^2-5b^2)$ for anything where $a+b\sqrt{5} \not=0$. Then $x \in \Z[\sqrt{5}]$ is a unit if and only if the norm is $\pm 1$. Notice that this method depended on the fact that $5$ was square-free (no repeated primes).
\end{example}

\subsection{Polynomial Rings}

Polynomial rings are the focus of Galois Theory which will be covered in depth next semester. Let $R$ be a commutative ring with identity, and form $R[x]$. This is the ring of polynomials with coefficients in $R$. Let $p(x) \in R[x]$ be a nonzero polynomail, and let ${\partial p} = n$ be the degree of the polynomial. We say that $p(x)$ is monic if $a_n = 1$.

\begin{example}
\begin{enumerate}
\item Let $R = \Z/2\Z$. Then all polynomials have coefficient $1$ or $0$. Then there are $2^n$ distinct polynomials of degree $5$.

\item Let $R = \Z/4\Z$. This ring has zero devisors, so $(2x)^2 = 0$. Also consider $(2x+1)^2 = 1$, so $2x + 1$ is a unit in $R[x]$.

\item Notice that $R$ is a subring of $R[x]$ since it's just the ring of constant polynomails.
\end{enumerate}
\end{example}

\begin{proposition}
Let $R$ be a commutative unital ring which is an integral domain, and let $p,q \in R[x]$. Then the following hold.
\begin{enumerate}
\item $\partial (pq) = \partial p + \partial q$.
\item $R[x]^\times = R^\times$.
\item $R[x]$ is also an integral domain.
\end{enumerate}
\end{proposition}

\begin{proof}[Proof of (1.) and (3.)]
Let $p = a_0 + \cdots + a_nx^n$ and let $q = b_0 + \cdots + b_nx^m$. Notice that $pq = a_0b_0 + \cdots + a_nb_mx^{n+m}$, and since $a_nb_m \not= 0$ then $\partial(pq) = \partial p + \partial q$.
\end{proof}

\begin{proof}[Proof of (2.)]
Notice that $R^\times \subset R[x]^\times$ trivially, since we didn't do anything to inverses. To show the other direction, we show that all units of $R[x]$ have degree zero. Suppose $p \in R[x]^\times$, so there exists a $q \in R[x]$ such that $pq = 1$. Since degrees are additive, it must be the case that $\partial(pq) = \partial p + \partial q = 0$ so $\partial p = 0$.
\end{proof}

\subsection{Matrix Rings}

Pick a ring $R$, any ring! Also let $n \in \Z_{>0}$. Then $M_n(R)$ is the ring of matricies with entries in $R$. Notice that even in $R$ is commutative, $M_n(R)$ won't be for all $n \geq 2$. Fun fact, if $R \not= 0$ then this ring \emph{will have} zero divisors so it cannot be an integral domain. As an example, consider 
\[
\begin{pmatrix}
0 & 1  \\ 0 & 0
\end{pmatrix}^2 = 0,
\]
for $n = 2$. Just as $R \subset R[x]$ is a subring, so to $R \subset M_n(R)$, where $r \in R$ is isomorphic to the diagonal matrix with all nonzero entries $r$. The units of $M_n(R)$ form a ring $\mathrm{GL}_n(R) = \left(M_n(R)\right)^\times$.

\subsection{Group Rings}

\begin{example}
Let $G = Z_2$ be a group. Consider a ring $R = \Z$. We're gonna stick them together to form $RG$ by ``forming a vector space of $G$ over $R$'' (pretty sure this is actually a module). Then $RG = \{a1 + bx \mid z,b \in \Z\}$ where component addition is inherited from $R$, and components add only if they are the same ``variable,'' and multiplication is inherited from $G$.
\end{example}

Let $(G, \cdot) = \{g_1, \dotsc, g_n\}$ be any finite group, and let $R$ be any commutative ring with identity. Then we form $RG = \{a_1g_1 + \cdots + a_ng_n \mid a_i \in R\}$ where we add and multiply in the expected way.

\begin{problem}
What is the copy of $R$ inside of $RG$? It's $\{r1_G\}$, so exactly the constant terms just like in the polynomials.
\end{problem}

\begin{example}
Let $G = D_6$ and let $R = \Z$. Consider $(3s + 2rs)(r^2 + s) = 3sr^2 + 3s^2 + 2rsr^2 +2rs^2 = 3rs + 4r + s + 3$. This actually has zero divisors! Consider $(1+s)(1-s) = 0$.
\end{example}

In general, for $g \in RG$ where $\abs{g} = m$, we have that $(1-g)(1 + g + \cdots + g^{m-1}) = 1 - g^{m} = 0$. Then if $m > 0$, $1-g$ is a zero divisor.

I want to know that the analogous concept to a group action will be. Miki says to ask again in four lectures.

\subsection{Why doesn't \texorpdfstring{$\mathbb{H} = \R Q_8$}{H = RQ8}?}

Check the orders. Notice that $\abs{\R Q_8} = 8$ while $\abs{\mathbb{H}} = 4$. This problem arises because $\R Q_8$ treats $i$ and $-i$ as \emph{different variables}. Also, note that there are zero divisors in $\R Q_8$, but there are no zero divisors in $\mathbb{H}$ since it is a division ring. In fact, since we already proved that $RG$ contains zero divisors in $\abs{G} > 1$ then we know that $\mathbb{H} \not= RG$ for \emph{any} ring $R$ and group $G$.
\section{Friday, 16 November 2016}

\epigraph{``No one really cares about left ideals.''}{Miki}

\begin{problem}[Challenge Problem]
What is $\Aut(Z_6 \times Z_2)$?
\end{problem}

\subsection{Ring Maps}

\begin{definition}[Ring Homomorphisms]
A \emph{ring homomorphism} s a map $\phi : R \to S$ where $\phi(a + b) = \phi(a) + \phi(b)$ and $\phi(ab) = \phi(a)\phi(b)$. The kernel of $\phi$ is the underlying group kernel; that is, $\ker \phi = \{a \in R \mid \phi(a) = 0\}$. If $\phi$ is bijective it is a ring isomorphism.
\end{definition}

\begin{example}
\begin{enumerate}
\item Consider $\Z \to \Z/n/Z$ via $a \mapsto \bar{a}$. Then $\bar{a}\bar{b} = \overline{ab}$ and $\bar{a} + \bar{b} = \overline{a + b}$.
\item Consider $\Q[x] \to \Q$ via $f(x) \mapsto f(0)$. That is, we evaluate the function at zero. Multiplying and adding polynomials works so this map is a homomorphism.
\end{enumerate}
\end{example}

\begin{proposition}
If $\phi : R \to S$ is a homomorphism then $\Im(\phi)$ is a subring of $S$ and $\ker \phi$ is a subring of $S$.
\end{proposition}

\begin{corollary}
The important fact about the kernel is that if $a \in \ker\phi$ then $\phi(a)\phi(x) = 0$ for \emph{any} $x \in R$. This is a really strong property. It means that the kernel isn't just closed under multiplication, its closed under multiplication by \emph{anything}.
\end{corollary}

\begin{definition}[Ideals]
A left ideal $I \subseteq R$ is a subring of $R$ with the property that $rI \subseteq I$ for all $r \in R$. A similar definition exists for a right idea, where $Ir \subseteq I$ for all $r \in R$. If $I$ is both a left and right ideal, then it is just called an \emph{ideal}.
\end{definition}

\begin{example}
For any ring $R$ and any homomorphism $\phi$, then $\ker \phi$ is an ideal in $R$.
\end{example}

\subsection{Quotient Rings}

Consider $\phi : \Z \mapsto \Z/2\Z$. Then let $\ker\phi = I$, so $(I,+) \leq (\Z,+)$. We know that we have two cosets, $0 + I$ and $1 + I$ which are derived from our cosets of $\Z/2\Z$. Then we can form a new object, $\Z/I$. We know from group structure that the multiplication is well defined, and multiplication also works since $\textsc{odd} \times \textsc{even} = \textsc{even}$ and $\textsc{even} \times \textsc{even} = \textsc{even}$.

\begin{example}[Example where this doesn't work]
Let $R = \Z[x]$ and $S \subset R$ be the polynomials in $x^2$, so $s = 1 + x^2 + 3x^6$, for example. Notice that $R/S$ doesn't work. For example, we want that $\bar{1} \cdot \bar{x} = \bar{x}$. But $(1 + x^2) \cdot x = x + x^3$ which is not in $x + S$.
\end{example}

The punchline for this is that quotient rings only work when $S$ is an ideal. This is kind of like the condition for normality for groups. For any $S \subseteq R$ we know that $(R/S,+)$ is okay since $(S,+) \normal (R,+)$ since $R$ is commutative in addition, but the multiplication is where we get tripped up. For any $r_1,r_2 \in R$ and any $a \in S$ we want that $r_1(r_2 + a) \in r_1r_2 + S$. But $r_1(r_2 + a) = r_1r_2 + r_1a$ which means that $r_1a$ must be in $S$ for \emph{any} $r_1$ which happens only if $S$ is a left  ideal. Similarly, we need that $(r_1 + a)r_2 = r_1r_2 + ar_2$, which tells us that $S$ must be a right idea, so it must be an idea.

\begin{proposition}
If $I \subseteq R$ is an ideal then $R/I$ is a ring and $(r + I) + (s + I) = (r+s) + I$ and $(r+I) + (s + I) = rs + I$.
\end{proposition}

\begin{example}
\begin{parts}
\part Consider $\Q[x] \to \Q$ again where $p(x) \mapsto p(0)$. The kernel of this is the set of polynomials with a zero constant term. Notice then that $\Q[x]/\ker\phi \cong \Q$ which looks a hell of a lot like the first isomorphism theorem for groups.
\part Consdier $\Q[x]$ and $I = \{\text{polynomials of degree $\geq 2$}\}$. This doesn't work out well.
\end{parts}
\end{example}

\subsection{Using Quotient Rings}

Consider $\pi : \Z \to \Z/p\Z = \F_p$. This induces a map $\Z[x] \to \F_p[x]$ where $a_nx^n \mapsto \bar{a}_nx^n$ (looking at the coefficients modulo $p$). Suppose there exists an $a \in Z$ such that $f(a) = 0$. Then $\bar{f}(\bar{a}) = \bar{0}$.

\begin{example}
Consider $f(x) = 3x^{52} + 4x^{49} - 16x^{20} - 17x - 1755$. Let's look at it modulo $2$. Then $\bar{f}(x) = x^{52} + x + 1$. Then $\bar{f}(\bar{0}) \not= \bar{0}$ and $\bar{f}(\bar{1}) \not= \bar{0}$ so there are no integral roots.
\end{example}

\begin{problem}
If we have a ring homomorphism $\phi : R \to S$, can we always induce a homomorphism $\psi : R[x] \to S[x]$.
\end{problem}
\section{Monday, 26 November 2018}

\epigraph{``I can hear your excitement all the way over hear.''}{Miki}

\subsection{Warm-Up}

Let $I$ and $J$ be ideals of $R$. Then 
\begin{itemize}
\item $I +J = \{i+j \mid i \in I, j \in J\}$,
\item $IJ = \{ij \mid i \in I, j \in J\}$,
\item $I^n = I \cdots I$.
\end{itemize}

\begin{problem}[Warm-Up]
Let $R = \Z$ and let $I = 2\Z$ and let $J = 3\Z$.
What are the following objects?
\begin{parts}
\part $I+J = \Z$;
\part $IJ = 6\Z$;
\part $I^n = 2^n\Z$.
\end{parts}
\end{problem}

\subsection{Ideals Generated by Subset}

For all of today, let $R$ have $1 \not= 0$.

\begin{definition}[Generated Ideal]
Let $A \subset R$ be a subset. Then $(A)$ is the smallest ideal containing $A$, also called the \emph{ideal generated by $A$}. Then let $RA = \{r_1a_1 + \cdots r_na_n \mid r_i \in R, a_i \in A\}$, and analogously construct $AR$. These are the left and right ideals generated by $A$, respectively. Then $RAR$ is the ideal generated by $A$, so $(A) = \bigcap I$ where $A \subset I$.
\end{definition}

\begin{definition}[Principal Ideal]
An ideal $I \subset R$ is finitely generated if there exists an $A \subset I$ with $I = (A)$. An ideal is principal if there exists an $a \in I$ such that $I = (a)$, so $I$ is finitely generated by a single element.
\end{definition}

\begin{example}
\begin{parts}
\part Let $R = \Z$. An ideal $I \subset R$ is an abelian subgroup, which means that $I = n\Z$ for some $n \in \Z$. Then all ideals of $\Z$ are of the form $(n)$, and so all are principal.

\part Let $R = \Z[x]$, and let $I = (2,x)$. This generates any polynomial which is a multiple of $x$, and if it has a constant term it must be even, so $RA = 2 \cdot p(x) + x \cdot q(x)$ for any integral polynomials $p,q$. Does there exist a single generator of $(2,x)$? Suppose that $I = (a(x))$. Then $2 = a(x)b(x)$ for some $b(x) \in \Z[x]$. Since degrees add in an integral domain, it must be that $a(x)$ is constant and $\partial a = 0$. Then $a = \pm 1$ or $\pm 2$. Since $I \not= R$ $a \not= \pm 1$, and if $a = \pm 2$ then $x \notin (a)$, so $(2,x)$ is not principal.
\end{parts}
\end{example}

\begin{definition}[Maximal Idea]
An ideal $M \subset R$ is maximal if $M \not= R$ and the only ideals containing $M$ are $M$ and $R$.
\end{definition}

\begin{corollary}
The only maximal ideals of $\Z$ are $(p)$ for some prime $p$.
\end{corollary}

\begin{proposition}
If $I \subset R$ is a proper ideal then there exists a maximal ideal $M \subset R$ such that $I \subset M$.
\end{proposition}

\begin{proof}[Proof in $\Z$]
Take any prime factor of $n$, so that $(n) \subset (p)$.
\end{proof}

\subsection{Fields}

\begin{proposition}
Let $R$ be a commutative ideal. Then $R$ is a field if and only if $R$ has exactly two ideals, $(0)$ and $R$.
\end{proposition}

\begin{proof}
First, assume that $R$ is a field. Note that $0 \not= 1$ so $(0)$ and $R$ are two distinct ideals. Then let $I \not= (0)$ be an ideal of $R$. Then there exists some $a \not= 0 \in I$. Since $R$ is a field, every nonzero element has an inverse, so $a \in R \setminus \{0\} = R^\times$. Then $I = R$, since $aa^{-1} = 1$ which generates $R$.

In the other direction, assume that $R$ has exactly two ideals. Let $a \in R\setminus \{0\}$, and let $I = (a)$. Since $a \not= 0$, it must be that $I = R$. Then $1 \in (a)$, which means that $1$ is a multiple of $a$, which means that $a$ has an inverse for any nonzero $a$. Then $R^\times = R \setminus \{0\}$, so $R$ is a field.
\end{proof}

\begin{corollary}
If $\phi : R \to S$ is a homomorphism where $R$ is a field, then $\phi$ is the zero map or $\phi$ is injective. 
\end{corollary}

\begin{proof}
Notice that $\ker \phi$ is an ideal of $R$, so it is $(0)$ or $R$.
\end{proof}

\begin{proposition}
Let $R$ be commutative and let $I \subset R$ be a proper ideal. Then $I$ is maximal if and only if $R/I$ is a field.
\end{proposition}

\begin{proof}
$R/I$ is a field if and only if there are exactly two ideals. Then the ideals of $R/I$ correspond one to one to the ideals of $R$ which contain $I$, by the second/third? isomorphism theorem. Then there can only by two ideals of $R$ which contain $I$ which must be $R$ and $I$, so $I$ is maximal in $R$.
\end{proof}


\section{Wednesday, 28 November 2018}
\section{Friday, 30 November 2018}

Didn't take notes :(
\section{Monday, 3 December 2018}

\epigraph{``Proof: You just monkey around with it.''}{Miki}

Suppose that $R$ is an integral domain.

\begin{definition}[Reducible]
 An element $r \in R$, $r \not= 0$, $r \in R^\times$ is reducible if there exist $a,b \in R \setminus R^\times$ such that $ab = r$. Otherwise, the element is irreducible.
\end{definition}

\begin{definition}
An element $r \in R$ is prime if $(r)$ is a prime idea. This means that if $r$ divides $xy$ then it must divide $x$ or $y$.
\end{definition}

\begin{definition}
An elemenet $r$ is associate to $s$ if there exists a $u \in R^\times$ such that $r=us$.
\end{definition}

\begin{proposition}
Let $R$ be an integral domain. If $p \in R$ is prime then $p$ is irreducible.
\end{proposition}

\begin{proof}
Suppose that $(p)$ is a prime ideal. Take any $a,b \in R$ such that $p = ab$. We want to show that at least one is a unit. If $ab = p$ then $ab \in (p)$, which means that either $a \in (p)$ or $b \in (p)$ since $(p)$ is a prime ideal. Without a loss of generality, let $a \in (p)$. Then $a = px$ for some $x \in R$. Then $p = ab = pxb$. Since $R$ is an integral domain, so $p(1-vb) = 0$ implies that $xb = 1$ which implies that $b \in R^\times$. 
\end{proof}

\begin{proposition}
There exist irreducible elements which are not prime.
\end{proposition}

\begin{proof}
Consider $\Z[\sqrt{-5}]$ with norm $N(a+b\sqrt{-5}) = a^2 + 5b^2$. First, we show that $3$ is irreducible. Suppose by way of contradiction that $3=ab$ for non-unit $ab$. Since this can't happen in $\Z$, we can assume that at least one of $a,b$ can be written as $x + y\sqrt{5}$ where $y \not=0$. Note that $N(3) = N(ab) = 9$, and since $b$ is not a unit its norm must be at least $2$, while the norm of $ab$ is then at least $10$, so $3$ is irreducible. Next, we show that $3$ is not prime. Note that $(1+\sqrt{-5})(1+\sqrt{-5}) = 6 \in (3)$, but $1 \pm \sqrt{-5} \notin (3)$ since $N(1 \pm \sqrt{-5}) = 6$ while $N((3)) = 9$.
\end{proof}

\begin{proposition}
If $R$ is a PID then $p \in R^\times, p \not= 0$ is irreducible implies that $p$ is prime.
\end{proposition}

\begin{proof}
Assume that $p$ is irreducible. We will show that $(p)$ is maximal which, in a PID, implies that it is prime. Let $I$ be an ideal such that $(p) \subset I \subset R$. Since $R$ is a PID, $I = (x)$ for some $x \in R$. If $p \in I$ then $p = rx$ for some $r \in R$. Since $p$ is irreducible, either $x$ or $r$ is a unit. Suppose that $r \in R^x$. Then $pr^{-1} = x \implies x \in (p)$, so $I \subset (p)$ and $I = (p)$. On the other hand, if $x \in R^\times$ and $(x) = I$ then $I = R$ since units generate the whole ring.
\end{proof}

\begin{definition}[Unique Factorization Domain]
A Unique Factorization Domain (UFD) is an integral domain where for every nonzer0 element $a \in R \setminus R^\times$ can be written as a finite product of irreducible elements, and this decomposition is unique up to order and associates.
\end{definition}

\begin{example}
\begin{parts}
\part $\Q$ and $\Q[x]$ are both UFDs.
\part $\Z[\sqrt{-5}]$ is \emph{not}, since $6 = 2 \times 3 = (1+\sqrt{-5})(1-\sqrt{-5})$.
\part $\Q[x_1, x_2, \dots]/(x_1-x_2^2, x_2-x_3^2, \dots)$ is \emph{not} a UFD since $x_1$ has \emph{no irreducible decomposition oh god why is this allowed to happen???} 
\end{parts}
\end{example}

\begin{proposition}
If $R$ is a UFD then every irreducible element is prime.
\end{proposition}

\begin{proof}
Suppose that $x$ is irreducible, and suppose that $x$ divides $ab$, where neither are units. We can factor $a$ and $b$ into irreducible $a = a_1 \cdots a_k$ and $b = b_1 \cdots b_\ell$, and $x$ divides $a_1 \cdots a_k b_1 \cdots a_\ell$. Since this is unique, and $x$ is irreducibe, $x$ must be associate to one of these elements, it must divide either $a$ or $b$.
\end{proof}

\begin{proposition}
All PIDs are UFDs.
\end{proposition}

\begin{proposition}
$\Z$ is a UFD.
\end{proposition}

\begin{proof}
$\Z$ is a PID.
\end{proof}

\begin{proposition}
GCDs don't imply a Euclidean Domain.
\end{proposition}
\section{Wednesday, 5 December 2018}

\epigraph{``I'm trying to think of how this is \emph{good} news...''}{Miki}

\epigraph{``You cannot send your computer to have tea with the bank's computer and discuss encryption.''}{Miki}

\subsection{A Sketch of Real Life}
The fact that authors define rings differently can really screw things up. For example, if a subring doesn't need to have the same identity as $R$, we could get something like $x$ is the identity in $S = \Z[x]/(x^2-x)$ but $x$ is not the identity in $\Z[x]$. In Galois Theory, it's common to say that $2x+2$ is irreducible even though by our definition it is reducible into $2(x+1)$ since our definition coincides with definitions of what ``prime'' means. 

\textsc{tl;dr:} Real life is messy. 

\subsection{Cryptography and RSA}

RSA is a security protocol that we use to communicate securely, and it hopefuly won't be broken any time soon (spoiler, it probably will). The advantage is that it allows us to choose our encryption scheme in a very public manner while still remaining secure.

\begin{theorem}
Let $m$ and $n$ be relatively prime to one another. Then $m^{\varphi(n)} = 1 \pmod{n}$. Recall that $\bar{m} \in (\Z/n\Z)^\times$.
\end{theorem}

\begin{proposition}
If $n = p$ is prime, and you take $k \in \Z$, then $m^{k\varphi(n) + 1} = m \pmod{n}$.
\end{proposition}

\begin{example}[Non RSA Example]
Let $n = p = 11$, and let $\lambda = \varphi(n) = 10$. Pick a unit $e \in (\Z/\lambda\Z)^\times = \{1,3,7,9\}$ (say $e = 7$). Calculate the inverse of $e$. sp $d = e^{-1} = 3$ in our example.

\noindent
Let $(n,e)$ be \textsc{public information} and keep $d$ as \textsc{private information}.

\noindent
Let your message be an integer $0 \leq m < n$, and encode it by $c = m^e$.

\noindent
Decode it by $c^d = m^{ed}$. Since $ed \equiv 1 \pmod{\varphi(n)}$, $c^d = m$.

\noindent
This is really stupid from a security perspective since we can easily calculate $d$ from $(n,e)$, but it illustrates how the encryption-decryption scheme works.
\end{example}

\begin{proposition}
Let $n = pq$ where $p,q$ are prime, and let $k \in \Z$. Then $m^{k \varphi(n) + 1} \equiv m \pmod{n}$. This is very similar to the previous proposition except that now we don't care that $m$ and $n$ are relatively prime.
\end{proposition}

\begin{theorem}[RSA]
Let $n = pq$ where $p,q$ are distinct primes with very many digits. Note that $\lambda = \varphi(n) = (p-1)(q-1)$, which is very easy to calculate if you know that $p$ and $q$ are but very hard otherwise. Then pick an $e \in (\Z/\lambda\Z)$. Picking this isn't always trivial since we don't know how to factor quickly, so we turn to the Euclidean algorithm to tell us if a chosen $e$ is actually relatively prime to $\lambda$. This is very efficient in the number of digits, so it's okay from a use perspective. Then, we need to find a $d = e^{-1} \in (\Z/n\Z)^\times$. Since we have the Euclidean Algorithm, we know that $1 = xe + y\lambda$, and modulo lambda we know that $d = x$. Then our \textsc{public information} is again $(n,e)$ and our \textsc{private information} is $d$, but now it is extremely hard to calculate $d$ from $(n,e)$. Then you can post your public key online and take your message $M$ and break it into $\{m\}$ where $0 \leq m < n$ for all $m$, and encrypt each part of the message in turn by $c = m^e$. Then we can decode it by $m = c^d$.
\end{theorem}
\section{Friday, 7 December 2018}

\epigraph{``$a$ is a bad person who lies to you.''}{Miki}

The last day of class :(

\subsection{RSA Continued}

The real security of RSA is the fact that it is very hard to factor large numbers, so long as we can choose our initial primes in a good manner. This involves determining if a number is prime, which is suprisingly nontrivial. There are a couple of ways to do this. One is due to Fermat:
\begin{enumerate}
\item Take $n$ and hope it's prime.
\item if $n$ is prime, then for any $a \in \Z$ where $(a,n) = 1$ then $a^{n-1} \equiv 1 \pmod{n}$.
\item Suppose on the other hand that $n$ is composite. Take $0 \leq a \leq n$ and evaluate $a^{n-1} \pmod{n}$. If you get something other than $1$, you know that $n$ isn't prime, and you call $a$ a Fermat witness. If you \emph{do} get $1$, then you don't know that it's not prime, and you call $a$ a Fermat liar since it lies to you that $n$ is prime, like a little liar.
\item Repeat this process a bunch of times. If you keep getting $1$, then maybe $n$ is prime after all.
\end{enumerate}
This is a probabalistic test for how likely it is that $n$ is prime. How good of an estimate this is relates to how many liars there could be. For ``most'' $n$, at most half of the numbers in the range $0 < a < n$ are liars, so the chance of picking $k$ liars is around $(1/2)^k$. We should qualify what ``most $n$'' actually means. In fact, there exist $n$ called \emph{Carmichael numbers}, which have the terrible property that any $a$ which is relatively prime to $n$ is a Fermat liar. This means that if you choose such an $n$ your kinda screwed if you use this test.

\begin{theorem}
A number $n$ is Carmichael if and only if it is square-free and if $p$ is a prime which divides $p-n$ then $p-1$ divides $n-1$.
\end{theorem}

This classification doesn't help you if you can't factor $n$, but it does give us an idea of how many Carmichael numbers there are.

\subsection{Fermat's Theorem}

\begin{theorem}
Let $p$ be an odd prime. Then there exist integers $a,b$ such that $p = a^2 + b^2$ if and only if $p \equiv 1 \pmod{4}$.
\end{theorem}

\begin{proof}[Forward direction]
Suppose that $p$ is a sum of squares. If $x \in \Z$ then $x^2 \equiv 1$ or $0$ modulo $4$, which we check by squaring all $\bar{x} \in \Z/4\Z$. Since $p$ is the sum of two squares, $p$ is either $0$, $1$, or $2$ modulo $4$, and since $p$ is odd it must be $1 \pmod{4}$.
\end{proof}

Brief interlude. Recall when we constructed numbers like $\Z[\sqrt{D}] = a + b\sqrt{D}$, and wiht then we had a norm $N(a + b\sqrt{D}) = (a + b\sqrt{D})(a - b\sqrt{D}) = a^2-b^2D$. We showed that $N$ might not be a norm since $N(x)$ could be less than $0$, but we can fix this by just taking the absolute value of this. We also showed that $N$ is multiplicative and that $N(x) = \pm 1$ if and only if $x$ is a unit. It's also worth remembering that not every norm makes $\Z[\sqrt{D}]$ a Euclidean domain.

\begin{proof}[Reverse Direction]
Assume that $p \equiv 1 \pmod{4}$. First, let's construct $R = \Z[i]$ and $N(a+bi) = a^2 + b^2$. We claim that $p$ is nor a prime element in $R$, so there exist $x,y$ such that $p$ divides $xy$ but neither one individually. We know that $p \equiv 1 \pmod{4}$ so $4$ divides $p-1$. Consdier $(\Z/p\Z)^\times \cong Z_{p-1}$. Then there exists an $n \in \Z$ such that $\bar{n}$ has order $4$ in $(\Z/p\Z)^\times$. Then if $\abs{\bar{n}} = 4 \pmod{p}$ then $\abs{\bar{n}^2} = 2 \pmod{p}$ so $\bar{n}^2 = -1 \pmod{4}$ and $p$ divides $n^2 + 1$. In the Gaussian integers, then we know that $p$ divides $n^2 + 1 = (n-i)(n+i)$ but $p$ does not divide $n \pm i$.

Note that $R$ is a PID, which implies that all irreducible elements are prime, so $p$ is not irreducible in $R$, o there exist $x,y \in R$ such that $x,y \notin R^\times$ but $xy = p$. Recall that $N(p) = p^2 = N(x)N(y)$, and we know that $N(x),N(y) \not= \pm 1$. Then both are either $+p$ or $-p$. We can write $x$ as $a+bi$, so $N(x) = a^2 + b^2 = p$.
\end{proof}

\subsection{The End}

That's all, folks. Thanks to Miki for a great semester!

\end{document}