\section{September 14, 2018}

\epigraph{``Why do we get struck by lightning when we reach a contradiciton? I don't know, it's usually a good thing.'' \lightning}{Miki}

\begin{definition}[Center]
The center of a group $G$ is $Z(G) = \{g \in G \mid gs = sg \text{ for all $s \in G$}\}$; i.e., $Z(G) = C_G(G)$, so it's the centralizer of the whole group.
\end{definition}

Why do we care so much about conjugation? We give all these special names to the sets of conjugation, like the Normalizer, Stabalizer, and Centralizer. We also know that conjugation preserves the order of an element, so $\abs{a} = \abs{gag^{-1}}$.

\begin{problem}
What is the center of $D_8$? We know the identity must be in the center. What about $r^2$? We know it commutes with $s$, and $sr^2 = r^{-2}s = r^2s$, so it commutes with $s$ as well; Since $r$ and $s$ generate the group, we know that it be in the center as well. So $Z(D_8) = \{1, r^2\}$.
\end{problem}

\subsection{Cyclic Groups}

\begin{proposition}
Let $G$ be a group, and let $x \in G$. For $m,n \in \Z$, if $x^n = x^m = 1$ then $x^d = 1$ where $d = \gcd(m,n)$.
\end{proposition}
\begin{proof}
Use the Euclidean Algorithm. We know there are integers $a,b \in \Z$ where $d = am + bn$, so $x^d = x^{am+bn} = (x^a)^m (x^b)^n = 1^a1^b = 1$.
\end{proof}
\begin{corollary}
If $x^m = 1$ then $\abs{x}$ divides $m$ if $m$ is finite.
\end{corollary}
\begin{proof}
If $m = 0$, we are done since everything divies zero. Assume that $1 \leq m < \infty$. Let $n = \abs{x} \leq m < \infty$ be finite. Let $d = \gcd(m,n)$, so $x^d = 1$. We know that $d$ divides $n$, and since $n$ is the smallest power of $x$ to be the identity, we know that $d = n$. \emph{A priori,} we know that $d$ divides $m$ so $d$ must divide $m$ as well.
\end{proof}
\begin{proposition}
Let $x \in G$, and let $a \in \Z \setminus \{0\}$.
\begin{enumerate}
\item If $\abs{x} = \infty$, then $\abs{x^a} = \infty$;
\item If $\abs{x} = n < \infty$, then $\abs{x^a} = n / \gcd(a,n)$.
\end{enumerate}
\end{proposition}
\begin{proof}
The proof of (1) is ommitted, and left as an exercise to the student. For (2), let's focus on the special case that $a$ divides $n$. If $x^n = 1$ then $(x^a)^{n/a} = x^n = 1$. Then $\abs{x^a}$ is at most $n/a$. Suppose by way of contraction that the order $d$ is strictly less than $n / a$. Then $x^{ad} = 1 \implies 1 \leq ad < n$, but $\abs{x} = n$. This is a contradiction, so the order of $x^a$ must be exactly $n/a$. In the case that $a$ does not divide $n$, play around with this to get the more general conclusion (the logic is the same).
\end{proof}

\begin{definition}[Cyclic Group]
A group $G$ is cyclic if there exists an $x \in G$ such that $G = \langle x \rangle$. As a note, it's not always easy to tell since there could be other presentaitons of a group which are not single elements. Always remember that presentations are not unique.
\end{definition}

\begin{problem}
Let $G = \langle a,b \mid a^2 = b^3 = 1, ab = ba \rangle$. Show that $G$ is cyclic.
\end{problem}

\begin{corollary}
All cyclic groups must be abelian, since any $g \in G$ is generated by some $x^a$, and $x$ always commutes with itself.
\end{corollary}

\begin{example}[Infinite Cyclic Groups]
Throughout, let $G = \langle x \rangle$, and assume that $\abs{x} = \infty$.

\begin{proposition}
The order of $G$ is $\infty$. Then $x^m \not= x^n$ for all distinct $m,n \in \Z$.
\end{proposition}

\begin{proof}
Let $m < n$. Suppose by way of contradiction that $x^m = x^n$. Then $x^{n-m} = 1$, which cannot happen since $n-m > 0$ and $\abs{x} = \infty$. Then $\abs{G} = \infty$.
\end{proof}

\begin{proposition}
Such $G$ must be isomorphic to $(\Z, +)$.
\end{proposition}
\begin{proof}
Define a map $\phi : \Z \to G : n \mapsto x^n$. This map is well defined. It also respects multiplication since $m+n \mapsto x^mx^n$. It is injective by Proposition~1, and it is surjective by Proposition~1 since $G$ is generated completely by $x$. Then $\phi$ is an isomorphism.
\end{proof}

\begin{proposition}
Such a group $G$ is generated by $x^n$ if and only if $n = \pm 1$.
\end{proposition}
\begin{proof}
Left as an exercise to the student.
\end{proof}

\begin{proposition}
Every subgroup of $G$ is cyclic of the form $H = \langle x^n \rangle$ for some $n \in \Z$.
\end{proposition}
\begin{proof}
Suppose that $x^n = 1$. Then $H$ is obviously cyclic. On the other hand, if $H \not= \langle 1 \rangle$. Let $n = \min\{k > 0 \mid x^k \in H\}$. This can't be empty, so there is an $n$. Then $\langle x^n \rangle \subset H$. Take some other element $x^m \in H$, and let $d = \gcd(m,n) = am + bn$. Then $x^d = (x^m)^a(x^n)^b \in H$ but $1 \leq d \leq n$, so $d = n$. Then $n$ divides $m$, and so $x^m \in \langle x^n \rangle$. The $\langle x^n \rangle = H$, and so $H$ is cyclic.
\end{proof}
\end{example}
\begin{corollary}
Every non-trivial subgroup of $\Z$ is isomorphic to $\Z$.
\end{corollary}
\begin{corollary}
For some cyclic $G$, we know that $\langle x^n \rangle = \langle x^{-n} \rangle \subset G$. Then all non-trivial subgroups correspond to $\Z_{>0}$.
\end{corollary}
