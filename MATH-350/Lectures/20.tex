\section{Monday, 22 October 2018}

\epigraph{``You all look so unhappy.''}{Miki}
\epigraph{``$p$ is going to be prime for \emph{at least} two more days.''}{Miki}

\subsection{Clasifying Automorphisms}

Let's talk automorphisms!

\begin{definition}[Inner Automorphism]
Let $g \in G$ and let $\sigma_g : G \to G : x \mapsto gxg^{-1}$ be an automorphism (i.e., an automorphism by conjugation). Then $\sigma_g$ is an \emph{inner automorphism}. The collection of all inner automorphisms forms a group $\Inn(G) \leq \Aut(G)$ which is isomorphic to $G/\mathrm{Z}(G)$ by the first isomorphism theorem.
\end{definition}

\begin{example}
Let $G = \Z/n\Z$. We proved on homework that $\Aut(G) \cong (\Z/n\Z)^\times$, and so any $\sigma \in \Aut(G)$ is uniquely determined by the map which sends $1$ to $a$ for some unit $a$. Since $G$ is commutative, conjugation doesn't really do anything, so $\Inn(G) = \sigma_1$. Put another way, $\mathrm{Z}(G) = G$, so $\Inn(G)$ is as small as it could be.
\end{example}

\begin{example}
Let $G = D_8$. The center of $D_8$ is $\mathrm{Z}(D_8) = \langle r^2 \rangle$. We know that $\Inn(G) \cong G/\langle r^2 \rangle \cong K_4$.
\end{example}

\begin{definition}[Characteristic]
A subgroup $H \leq G$ is \emph{characteristic} if $\sigma(H) = H$ for any $\sigma \in \Aut(G)$. This is like a normal subgroup, except that a normal subgroup need only be preserved under \emph{inner automorphism} while a being characteristic subgroup is a stronger condition.
\end{definition}

\begin{example}
Let $G = D_8$ and let $H = \langle r^2 \rangle$. Since $H$ is the center, this is characteristic (this is true in general). Next let $K = \langle r \rangle \leq G$. Since $\Im(r)$ is either $r$ or $r^3$ (check the order under isomorphism) then $\sigma(\langle r \rangle) = \langle r \rangle$ for any $\sigma \in \Aut(D_8)$ and so it is characteristic.
\end{example}

Just to make the point explicit, if $H$ is characteristic in $G$ then it must be normal in $G$, but the reverse is not true. Additionally, if $H$ is the unique subgroup of a particular order in $G$ then it must be characteristic since there's nothing else it could be sent to under an automorphism since it's image must be a subgroup of the same order.

\subsection{Sylow \texorpdfstring{$p$}{p}-subgroups}

\begin{definition}
Let $p$ be prime. A $p$-subgroup is a subgroupd of order $p^n$ for $n \geq 0$.
\end{definition}

\begin{definition}
Let $\abs{G} = p^am$ where $p$ does not divide $m$. If there is a subgroup of order $p^a$ (there is) then a subgroup of this order is called a Sylow $p$-subgroup. The set of all such groups is written as $\Syl_p(G)$. The number of such groups is written as $n_p(G) = \abs{\Syl_p(G)}$.
\end{definition}

\begin{example}
If $p$ does not divide $\abs{G}$ the the only Sylow $p$-subgroup is the trivial subgroup. If $\abs{G} = p^a$ then the unique Sylow $p$-subgroup is $\Syl_p(G) = \{G\}$.
\end{example}

\begin{example}
Let $G = S_3$ which has order $2 \cdot 3$. Let $p = 2$, $m = 3$, and $a = 1$. Then the largest $\Syl_p$ subgroup is $C_2$, of which there are three such subgroups (things generated by $2$-cycles). If we let $p = 3$, then there is one Sylow $p$-subgroup, generated by a $3$-cycle.
\end{example}

\subsection{Sylow Theorems}

Throughout, let $p$ be prime and let $G$ be a group of order $p^am$ where $a > 0$ and $p$ does not divide $m$.

\begin{theorem}[Sylow I]
There exists a subgroup of $P \leq G$ where $\abs{P} = p^a$.
\end{theorem}

\begin{theorem}[Sylow II]
For each $p$, the Sylow $p$-subgroups are conjugate to one another.
\end{theorem}

\begin{theorem}[Sylow III]
The number of Sylow $p$-subgroups of $G$, written $n_p(G)$, divides $m$ and is congruent to $1 \bmod{p}$.
\end{theorem}

We'll prove these next time (with a lot of chocolate). Today we'll just talk about the implications of these theorems.

\begin{corollary}
There must exist an $x \in G$ whose order is $p$.
\end{corollary}
\begin{proof}
Let $y \in P$ be not the identity. Then $\abs{y} = p$, so for some $0 < b \leq a$ we know that $x = y^{p^{b-1}}$.
\end{proof}

\begin{corollary}
The Sylow $p$-subgroups are all conjugate.
\end{corollary}