\section{Monday, 20 October 2018}

\epigraph{``Let's write down all finitely generated abelian groups. What fun.''}{Miki}

We have two goals for today.
\begin{enumerate}
\item Is $Z_{20} \times Z_{18} \cong Z_{36} \times Z_{10}$?
\item How do we classify \emph{all} finitely generatred abelian groups?
\end{enumerate}

To start answering these, we'll begin with a proposition.

\begin{proposition}
$Z_n \times Z_m \cong Z_{mn}$ if and only if $\gcd(m,n) = 1$.
\end{proposition}

\begin{proof}
Let $d = \gcd(m,n)$, and let $Z_m = \langle x \rangle$ and let $Z_n = \langle y \rangle$. Consider $G = Z_m \times Z_n = \{(x^a, y^b)\}$. Consider $(c,f) \in G$. Then $\abs{(c,f)} = \lcm (\abs{c}, \abs{f})$. If $d = 1$ then $\abs{(x,y)} = \lcm (m,n) = mn$, so $Z_{mn} \cong \langle (x,y) \rangle \leq G$. Since the orders are the same, it is isomorphic to the whole thing.
On the other hand, if $d > 1$, let $(c,f) \in G$, and consider $(c,f)^{mn/d} = (c^{mn/d}, f^{mn/d}) = (e,e)$, so every element has order strictly less than $mn$ since $d > 1$. Therefore $G \not\cong Z_{mn}$.
\end{proof}

\begin{example}
Consider $Z_9 \times Z_6 \not\cong Z_{54}$. Note that $Z_9 \times Z_6 \cong Z_9 \times Z_3 \times Z_2 = Z_{18} \times Z_3$.
\end{example}

\begin{example}
Use the proposition we just proved to ``factor'' the groups into the same decomposition. Ta-Da!
\end{example}

fdas

\subsection{Classifying Finitely-Generated Abelian Groups}

\begin{definition}[Free Abelian Group]
Let $\Z^r = \Z \times \cdots \times \Z$ ($r$ times) be the free abelian group of rank $r$.
\end{definition}

\begin{theorem}[Classification Theorem for Finitely Genreated Abelian Groups]
Let $G$ be a finitely generated abelian group. Then there is a unique decomposition of $G$ satisfying
\begin{enumerate}
\item $G \cong \Z^r \times Z_{n_1} \times \cdots Z_{n_s}$ for $r,n_i \in \Z$,
\item $n_i > 2$ for all $i$, and 
\item $n_{i+1}$ must divide $n_i$ for all $1 \leq i \leq s-1$.
\end{enumerate}
\end{theorem}

\subsection{Classifying Finitely-Generated Abelian Groups 2, Electric Boogaloo}

\begin{example}
Consider $Z_{60} \cong Z_{2^2} \times Z_{3} \times Z_{5}$. Notice now that all the components are $p$-subgroups.
\end{example}

\begin{theorem}
Let $\abs{G} = n = \prod p_i^{a_i}$, where $a_i \geq 1$. Then we can write $G$ uniquely (up to order of primes) as $G \cong A_1 \times \cdots \times A_k$ where $\abs{A_i} = p_i^{a_i}$, and for all $A = A_i$ where $\abs{A} = p^a$, we know that $A \cong Z_{p^{b_1}} \times \cdots \times Z_{p^{b_\ell}}$ where $b_1 \geq b_2 \geq \cdots \geq b_\ell$, where the sum of all $b_i$ is $a$.
\end{theorem}
