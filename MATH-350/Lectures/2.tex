\section{August 31, 2018}

As always, Miki began class at precicely 10:25 \textsc{am}. She wrote a review of last lecture on the bard, and then posed the following question as a warm up. She also talked about how the DUS department is arguing over whether money should be spent on T-shirts or chocolate (Miki thinks chocolate).

\begin{problem}[Warm Up]
Are these groups?
\begin{parts}
\part[w:1a] $(\Z/n\Z, \times)$;
\part[w:1b] $(\Z/n\Z \setminus \{0\}, \times)$
\end{parts}
\end{problem}

\begin{solution}
The solutions to the warm-up
\begin{subproof}[Solution to~\ref{w:1a}]
No, since $0$ has no inverse.
\end{subproof}

\begin{subproof}[Solution to~\ref{w:1b}]
No, this only works when $n$ is prime. For any factors $a,b$ of $n$, $a \times b = 0$, which isn't in the group. We say that $(\Z/p\Z, \times)$ is a group for all prime $p$.
\end{subproof}
\end{solution}

\begin{theorem}[Fermat's little theorem]
For prime $p$ and composite $a = np$, hen $a^{p-1} \equiv 1 \pmod{p}$. 
\end{theorem}

\begin{lemma}
If $\bar{a} \in \Z/p\Z \setminus \{0\}$, then $\bar{a}$ has an inverse in $(\Z/p\Z \setminus \{0\}, \times)^*$.
\end{lemma}

\begin{definition}[Units]
A unit is something which has an inverse. The units of a group are denoted by putitng an asterisk after teh group, eg $(\Z/p\Z \setminus \{0\}, \times)^*$.
\end{definition}

\begin{example}
For integers modulo 4, $(\Z/4\Z, \times)^* = \{\bar{1}, \bar{3}\}$.
\end{example}

\begin{problem}[On Homework]
What are the conditions for determining the units of a group? We know it must have an inverse, but that's hard to check. Instead, we know that $a$ is a unit if and only if $\gcd(a,n) = 1$. Prove this.
\end{problem}

\subsection*{Symmetries of a regular $n$-gon}
Miki is angry with the book because she doesn't like how it treats symmetries, I think because she wants $D_{2n}$ to be called $D_n$.

Miki drew a triangle on the board, and began talking about the different operations we can preform on that triangle to preserve symmetries. She introduced $s$ to mean a reflection, and $r$ to mean a rotation. For a triangle, there are three distinct reflections,
\[ s = \{s_1, s_2, s_3 \}, \]
where $s_i$ is the reflection across the line $OA_1$. We can also rotate the triangle in two directions.

We know that these are all the symmetries, since we can count the permutations of the triangle. We've exhauseted then, so we know that there can't be any more elements of the triangle-symmetry group $D_6$. In fact, because of the permutation fact, we konw that $\abs{D_{2n}} = 2n$. Some other observations about $D_{2n}$:
\begin{itemize}
\item $s^2 = e \implies s = s^{-1}$;
\item rotating twice clockwise is the same as rotating counterclockwise, so these aren't unique elememnts;
\item $r^n = e$
\item $rs = s_2$, so $s_n$ is just a combination of $r$ and $s$ --- then we can generate the entire group with just $r$ and $s$.
\end{itemize}
These things lead us to discover a new object.

\begin{definition}[Generators]
For a group $G$, the generators of $G$ is a set $ S = \{a,b,\dotsc : a,b,\dotsc \in G\}$ where $G$ is equal to all possible sombinations of elements of $S$. For $D_{2n}$, we could say that $D_{2n}$ is generated by $r$ and $s$.
Usually there isn't a way to guess the generators of a group easily.
\end{definition}

\begin{definition}[Relations]
A relation is a way of writing equivalent elements of groups. For example, in $D_{2n}$,
\[ r^3 \equiv 1, \qquad s^2 \equiv 1, \qquad sr \equiv r^2s. \]
Relations allow us to define how we can commute elements of the group.
\end{definition}

\begin{definition}[Presentation]
A presentation of a group are the generators combined with the relations necessary to create the group. The largest group which is generated from the generators and which satisfies the relations, and has no other relations, is our group. A presentation is written as $\langle a,b \mid \text{relations between $a$ and $b$} \rangle$, where $a$ and $b$ are the generators of the group.
\end{definition}

Now Miki told us that the group of the  symmetires of a regular $n$-gon is the dihedral group of order $2n$, written either as \{$D_{2n}$ or $D_{n}$\}, depending on if you are a representation theorist or not.

\begin{problem}[HW]
Why is the order of $D_{2n}$ always $2n$?
\end{problem}

\subsection*{Symmetric group on $n$ elements}

Miki defined the symmetric group on $n$ elements $S_n$, which is just the permutations of $n$ elemnts. Notice that $D_{2n}$ is a subgroup of $S_n$. We know that the order of $S_n = n!$ and the order of $D_{2n} = 2n$.

[Insert diagrams of different ways to denote permuations, like the cycle notation]
