\section{Monday, 26 November 2018}

\epigraph{``I can hear your excitement all the way over hear.''}{Miki}

\subsection{Warm-Up}

Let $I$ and $J$ be ideals of $R$. Then 
\begin{itemize}
\item $I +J = \{i+j \mid i \in I, j \in J\}$,
\item $IJ = \{ij \mid i \in I, j \in J\}$,
\item $I^n = I \cdots I$.
\end{itemize}

\begin{problem}[Warm-Up]
Let $R = \Z$ and let $I = 2\Z$ and let $J = 3\Z$.
What are the following objects?
\begin{parts}
\part $I+J = \Z$;
\part $IJ = 6\Z$;
\part $I^n = 2^n\Z$.
\end{parts}
\end{problem}

\subsection{Ideals Generated by Subset}

For all of today, let $R$ have $1 \not= 0$.

\begin{definition}[Generated Ideal]
Let $A \subset R$ be a subset. Then $(A)$ is the smallest ideal containing $A$, also called the \emph{ideal generated by $A$}. Then let $RA = \{r_1a_1 + \cdots r_na_n \mid r_i \in R, a_i \in A\}$, and analogously construct $AR$. These are the left and right ideals generated by $A$, respectively. Then $RAR$ is the ideal generated by $A$, so $(A) = \bigcap I$ where $A \subset I$.
\end{definition}

\begin{definition}[Principal Ideal]
An ideal $I \subset R$ is finitely generated if there exists an $A \subset I$ with $I = (A)$. An ideal is principal if there exists an $a \in I$ such that $I = (a)$, so $I$ is finitely generated by a single element.
\end{definition}

\begin{example}
\begin{parts}
\part Let $R = \Z$. An ideal $I \subset R$ is an abelian subgroup, which means that $I = n\Z$ for some $n \in \Z$. Then all ideals of $\Z$ are of the form $(n)$, and so all are principal.

\part Let $R = \Z[x]$, and let $I = (2,x)$. This generates any polynomial which is a multiple of $x$, and if it has a constant term it must be even, so $RA = 2 \cdot p(x) + x \cdot q(x)$ for any integral polynomials $p,q$. Does there exist a single generator of $(2,x)$? Suppose that $I = (a(x))$. Then $2 = a(x)b(x)$ for some $b(x) \in \Z[x]$. Since degrees add in an integral domain, it must be that $a(x)$ is constant and $\partial a = 0$. Then $a = \pm 1$ or $\pm 2$. Since $I \not= R$ $a \not= \pm 1$, and if $a = \pm 2$ then $x \notin (a)$, so $(2,x)$ is not principal.
\end{parts}
\end{example}

\begin{definition}[Maximal Idea]
An ideal $M \subset R$ is maximal if $M \not= R$ and the only ideals containing $M$ are $M$ and $R$.
\end{definition}

\begin{corollary}
The only maximal ideals of $\Z$ are $(p)$ for some prime $p$.
\end{corollary}

\begin{proposition}
If $I \subset R$ is a proper ideal then there exists a maximal ideal $M \subset R$ such that $I \subset M$.
\end{proposition}

\begin{proof}[Proof in $\Z$]
Take any prime factor of $n$, so that $(n) \subset (p)$.
\end{proof}

\subsection{Fields}

\begin{proposition}
Let $R$ be a commutative ideal. Then $R$ is a field if and only if $R$ has exactly two ideals, $(0)$ and $R$.
\end{proposition}

\begin{proof}
First, assume that $R$ is a field. Note that $0 \not= 1$ so $(0)$ and $R$ are two distinct ideals. Then let $I \not= (0)$ be an ideal of $R$. Then there exists some $a \not= 0 \in I$. Since $R$ is a field, every nonzero element has an inverse, so $a \in R \setminus \{0\} = R^\times$. Then $I = R$, since $aa^{-1} = 1$ which generates $R$.

In the other direction, assume that $R$ has exactly two ideals. Let $a \in R\setminus \{0\}$, and let $I = (a)$. Since $a \not= 0$, it must be that $I = R$. Then $1 \in (a)$, which means that $1$ is a multiple of $a$, which means that $a$ has an inverse for any nonzero $a$. Then $R^\times = R \setminus \{0\}$, so $R$ is a field.
\end{proof}

\begin{corollary}
If $\phi : R \to S$ is a homomorphism where $R$ is a field, then $\phi$ is the zero map or $\phi$ is injective. 
\end{corollary}

\begin{proof}
Notice that $\ker \phi$ is an ideal of $R$, so it is $(0)$ or $R$.
\end{proof}

\begin{proposition}
Let $R$ be commutative and let $I \subset R$ be a proper ideal. Then $I$ is maximal if and only if $R/I$ is a field.
\end{proposition}

\begin{proof}
$R/I$ is a field if and only if there are exactly two ideals. Then the ideals of $R/I$ correspond one to one to the ideals of $R$ which contain $I$, by the second/third? isomorphism theorem. Then there can only by two ideals of $R$ which contain $I$ which must be $R$ and $I$, so $I$ is maximal in $R$.
\end{proof}

