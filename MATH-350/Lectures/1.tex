\section{Wednesday, 29 August 2018}

Most of today's lecture was administrata covering how the course will be run.

Towards the end of the period we began to play with the very basic concepts of group theory. First and foremost is the definition of a group.

\begin{definition}[Group]
A group is an ordered pair $(G, \star)$, where $G$ is a set and $\star$ is a binary operation on $G$, which obeys the following axioms.
\begin{itemize}
\item There exists an element $e \in G$ known as the \emph{identity} with the property that $e \star g = g \star e = g$ for all $g \in G$.
\item For all $g \in G$ there exists a $g^{-1} \in G$ known as the \emph{inverse} of $g$ which has the property that $g \star g^{-1}= g^{-1} \star g = e$.
\item The operation $\star$ is associative.
\end{itemize}
\end{definition}

You will sometimes see a fourth axiom included in this list, namely that $G$ is closed under $\star$, but since $\star$ is a binary operation on $G$ which definitionally means it is a map $\star : G \times G \to G$, and so $G$ is always implicitly closed under $\star$. When \emph{checking} whether or not $(G, \star)$ is a group, though, it is a very good idea to check that $\star$ actually is a binary operation.

You're probably already familiar with lots of groups already. Consdier the rational numbers $\Q$, the real numbers $\R$, the set of symmetries of the square, and the integers (under addition, which is important).