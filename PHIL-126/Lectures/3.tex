% !TEX root = ../notes.tex

\section{Friday, January 18}

\begin{multicols}{2}
\begin{enumerate}
\item Reasons for doubt
\item Principles
\item Piece of paper
\item Dream argument
\item Coherence
\item Beliefs left standing
\item The big guns
\item Make my day
\item Coming and going
\item Rock bottom
\item Archimedes
\item \emph{Sum}
\item \emph{Cogito ergo sum}
\end{enumerate}
\end{multicols}

Descartes arrives at a radical skepticism, through which he attempts to further his three main aims. He wants to get rid of the skeptical doubts by the end of his meditations. Some original belifes come back, but his (old Aristotelian) views on physical objects do not. He doesn't want to doubt for the sake of doubt, but wants to have \emph{reasons} for doubt. He also knows he has infinitely many beliefs, so instead of going through each he goes after the ``fundamental pinciples'' of his beliefs. In particular, he scruitinizes beliefs which arise from his senses. Our eyes can be tricked (oasis, liquid on a hot road, square tower in the distance may look round, Dick Cheeny thinking his friend's face was a bird). However, this isn't a general reason to doubt the senses since there are always specific nonoptimal situations which give rise to the doubt, but under optimal circumstances we have no reason for doubt; for example, being here sitting by the fire with a piece of paper in my hand. To doubt this would make one insane. HOWEVER, what if I were dreaming? This does give rise to a valid reason for doubt.

\begin{definition}[Valid Argument]
A argument in which the conclusion necessarily follows from the premises.
\end{definition}

\begin{definition}[Sound Argument]
A valid argument in which the premises are true.
\end{definition}

\begin{enumerate}
\item[\textbf{Premise 1}]\label{des:p1} In order to be certain that I'm sitting, I must \emph{first} be certain that I'm not dreaming.
\item[\textbf{Premise 2}]\label{des:p2} I can't be certain that I'm not dreaming.
\item[\textbf{Conclusion}]\label{des:conc} I can't be certain that I'm sitting.
\end{enumerate}

This calls into doubt Physics, Astronomy, Medicine, and Empirical Science. Mathematics and Logic, however, he feels are safe from this doubt. This is a valid argument since \textbf{Premise 1} and \textbf{Premise 2} necessarily imply \textbf{Conclusion}. Descartes will eventually argue that it isn't sound though since it would imply that God is a deceiver, a premise which he later refutes in \emph{Meditations 3 \& 4}.