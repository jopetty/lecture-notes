% !TEX root = ../notes.tex

\section{Monday, February 4}

\begin{multicols}{2}
\begin{enumerate}
\item Mind-Body Interaction
\item Pineal gland
\item Intelligibility
\item Princess E.\ 1618--1680
\item Bohemian Rhapsody
\item E.'s Finest Hour
\item Mind-Body Union
\item Malebranche 1638--1715
\item What's the occasion?
\item Not just mind-body, but body-body too
\item No substantial forms
\item Causation + conceptual connection
\item M.'s finest hour
\item Spinoza 1632--1677
\end{enumerate}
\end{multicols}

\subsection{Bohemian Rhapsody}

Descartes was also a scientist in his day, and he conjectured about the physical relation between the mind and the body. He thought that the \emph{pineal gland} was the locus of this interaction. Elisabeth of Bohemia doubted Descartes account of the mind interacts with the body; in particular she wondered how a nonphysical, nonextended thing like the mind could cause interactions with a physical, extended thing like the body. She thought that contact and extension were necessary to induce movement or feeling in a body. Descartes' response was to doubt that these were actually necessary for interaction. He thinks that two things of different natures can interact. This ability is attributed to God, ever the helpful fellow that he is. These causal connections are in our interest, and so that is the reason for their existence. Elisabeth doubts this, and Descartes comes back to say ``Don't think about it too much, it wouldn't make sense for use to have a good idea of the connection anyways.'' Descartes calls this a \emph{primitive notion} which seems like a way for Descartes explain away the flaws in his reason.

\begin{problem}
How does this not violate the principle of sufficient reason?

\begin{solution}[Descarte's Response]
Yeah, it does. Oops. But I only use the PSR to justify God, not to justify anything else.
\end{solution}
\end{problem}

\subsection{Malebranche (1638--1715)}

Malebranche was a Cartesian philosopher who came a bit after Descartes. He was mainly concerned with Theodicy. He believed in the Mind-Body distinction but he had a hard time accepting Descartes' account for how the two interact. Malebranche supposes that, on the occasion of physical interaction with our bodies, a feeling appears in our mind due to God. When we intend to move our body, God moves our body in response to this intention. This divine intervention occurs in a regular fashion. For Malebranche, this is a general theory that also explains body-body interaction; all interactions are resultant from God's will. This theory of intervention by God when certain things happen is called \emph{occasionalism}.

\begin{problem}
Doesn't this just mean that God is the ultimate cause of all suffering and pain?

\begin{proof}[Malebranche's Response]
God is only acting on our intentions; they are the ultimate cause of these things. The only causal power finite objects have is the will to accept the good which God presents to us or to turn away from it.
\end{proof}
\end{problem}

\begin{problem}
What about the phantom limb problem?

\begin{proof}[Malebranche's Response]
God set up the system in the simplest way possible. Extra checks to ensure that the effects are valid are superfluous to the ultimate design of the system.
\end{proof}
\end{problem}

Malebranche, like Descartes, wants to get rid of the  substantial forms of Plato. To do this, they stripped bodies of all causal power and delegate causal power to God, or maybe to the mind in the case of will. For Malebranche, a cause is something which by its nature is connected to the effect.