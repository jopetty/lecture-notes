% !TEX root = ../notes.tex

\section{Monday, January 28}

\begin{multicols}{2}
\begin{enumerate}
\item Degrees of Reality;
\item Formal Reality \& Objective Reality;
\item Mister Ed;
\item Two causal principles \& and the PSR;
\item Whodunit;
\item D.'s finest hour;
\item Not a deceiver;
\item Cartesian Circle;
\item Intellectual Problem of Evil;
\item Belief \& the Will.
\end{enumerate}
\end{multicols}

\subsection{Degrees of Reality}

Descartes' proof of God's existence hinges on different degrees of reality, which in some sense is a measure of independence; in Descartes' view, God (an infinite substance) ought to be independent on anything else while everything else ought to be dependent on God. Finite and/or extended substances (table, mind, Mr.\ Ed, etc.) all depend on God, and in turn the shape of a table depends on the table itself.

\begin{definition}[Formal Reality]
The reality something has by virtue of its existence. Usually a measure of how independent a thing is. Often also derives from the complexity of the thing; a machine would have more formal reality than a rock.
\end{definition}

\begin{definition}[Objective Reality]
The objective reality of an idea is equal to the formal reality of the object of the idea has, or would have if it existed. Only ideas can have objective reality.
\end{definition}

\begin{example}
Consider the idea of Mr.\ Ed. This has relatively low formal reality since it's just a though in a mind, but has relatively high objective reality since Mr.\ Ed himself has a relatively high formal reality. The idea of god has infinite objective reality.
\end{example}

\subsection{Two Causal Principles}

\begin{proposition}
The formal reality of a cause is greater than or equal to the formal reality of the effect.
\end{proposition}

\begin{proposition}
The formal reality of the cause is greater than or equal to the objective reality of the effect.
\end{proposition}

From this, God asks ``What causes my idea of God?''