% !TEX root = ../notes.tex

\section{Monday, January 14}

\epigraph{``Do you know what Pop-tarts are?''}{Michael Della Rocca}

\subsection*{Lecture Outline}
\begin{multicols}{2}
\begin{enumerate}
\item 10 Philosophers
\item Our Problems
\item Dialogue
\item Mind-Body Problems
\item Naturalism
\item Freedom
\item Causation
\item Skepticism
\item Idealism
\item God
\item Metaphysics \& Epistemology
\item PSR \& The Finest Hour
\item My Motto 
\item Copernicus
\item Mechanism
\item Old Science
\begin{enumerate}
\item Final Causes
\item Substatial Forms
\item Trust the senses
\end{enumerate}
\end{enumerate}
\end{multicols}

This class will cover 17\textsuperscript{th} and 18\textsuperscript{th} century philosophers, including René Descartes, Nicolas Malebranch, Benedict de Spinoza, Anne Conway, Gottfried Leibniz, Emilie du Châtelet, John Locke, George Berkeley, David Hume, and Immanuel Kant. These philosophers defined the themes and methods of modern philosophy today. They tried to understand what the world is like and what our place in it is.

Philosophy is not teleological; older philosophers may have a \emph{better understanding} of some ideas than contemporary philosophers do --- this is one reason why we study older ideas. There is also a rich tradition of dialogue with older philosophers. Much of philosopher is thinkers responding to and criticizing older philosophers.

\begin{problem}[The Mind-Body Problem]
What is the relation, if any, between the physical body and the mind? How do mental states relate to bodily states? Are the two distinct or the same? Is the mind destroyed if the body is destroyed? Are we the same person we were in previous years? How to the mind and body interact?
\end{problem}

\begin{definition}[Naturalism]
The belief that everything plays by the same rules, and that the laws of nature apply to everything (including immaterial things, like the mind). Advocated by Spinoza and Hume. 
\end{definition}

\begin{problem}[Freedom]
Does free will exist? Are we ever truly free? How are our choices actually made? How should we be held responsible for our actions? If determinism is true, how can we be free in any sense?
\end{problem}

\begin{definition}[Determinism]
Everything that takes place now was determined in the past by earlier events. Past states necessitate the current state of the world. If you know a given state and all the natural laws, you can predict with certainty how the system will progress.
\end{definition}

\begin{problem}[Causation]
What does it mean for one thing to determine another thing? How does one billiard ball cause another to move when they strike one another? When a rock breaks a window, does the rock cause the breakage, or does God?
\end{problem}

\begin{problem}[Skepticism]
Do we really know that things exist or the state of things? What does knowing something actually mean? How do we know that we aren't all dreaming? Is there a deceiving God which invents a false reality for use to perceive? Do we really know future events, like the sun will rise tomorrow or if the eraser is let go then it will fall to the ground?
\end{problem}

\begin{definition}[Idealism]
Physical objects which exist in the world (objective things) are dependent on the mind, and the perception of these objects by the mind is what instantiates them.
\end{definition}

\begin{problem}[God]
Does God exist? How do we know? Can we prove it? Is God beholden to natural laws? Is God nature itself?
\end{problem}

\begin{definition}[Metaphysics]
The study of what exists.
\end{definition}

\begin{definition}[Epistemology]
The study of how do we know that something exists.
\end{definition}

\begin{definition}[Principle of Sufficient Reason]
The claim that for everything that exists and every occurance, there is an explanation for it. There is always a way to understand something even if we don't understand it yet. Invoked in a philosophers' \emph{Finest Hour}.
\end{definition}