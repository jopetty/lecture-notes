% !TEX root = ../notes.tex

\section{Wednesday, January 30}

\begin{multicols}{2}
\begin{enumerate}
\item Getting out of the circle;
\item Intellectual Problem of evil;
\item Belief \& the Will;
\item Do bodies exist?
\item Whodunit again?
\item The Real distinction
\item Leibniz's Law
\item Mind-body problem
\item Elisabeth (1618--1680)
\item E's finest hour and D's darkest hour
\item Mind-body union
\end{enumerate}
\end{multicols}

\subsection{Theodicy and Free Will}

It sure seems like Descartes' thinking involves circular reasoning, so much so that there's a name for it: `The Cartesian Circle.' Do try and get out of this, he somewhat restricts the scope of his doubt to no longer be skeptical of his current clear and distinct ideas.

\begin{problem}
Why can't God deceive our current clear and distinct ideas? Descartes doesn't really have an answer for that.
\end{problem}

\begin{problem}
Why should a good God allow me to make any mistakes at all?
\end{problem}

Answering these problems is a form of \emph{Theodicy} --- trying to get God off the hook for the problems in the world. Descartes does this by saying that the problems are our fault entirly. Being mistaken is a function of a disconnect between the \emph{intellect} and the \emph{will}. When the will supercedes the intellect and makes judgements for which it has no justification then we get a false belief. But we shouldn't fault God for giving us extra will, so he is therefore off the hook for the problems of our intellectual errors.

\begin{problem}
I can't assent to the idea that I can fly.

\begin{solution}[Response]
Descartes says that you freely don't assent to this idea. You could if you want to, but you choose to weigh your past experiences against your current beliefs. The only things the will doesn't have control over are clear and distinct ideas, like $2+2 = 4$. Descartes ardently defends the freedom of will in almost all circumstances.
\end{solution}
\end{problem}

\subsection{The Nature of Bodies (Meditation 5)}

A body is something which is extended --- it exists in three dimensions --- and has size, shape, and motion. We don't have a clear and distinct idea about the existence of bodies, so we cannot use this to prove their existence. Descartes does argue that the mind must be distinct from the body, if we are to have a body in the first place. However, the fact that we have ideas of anything means that they must be caused by something. Furthermore, our sensory ideas of bodies come to us against our will so they cannot originate in the mind. They can't come from God since this would mean that God is a deciever for two reasons: first, if God does this we have no recourse to figure out the truth against our will, and this is a deception. Second, God has given us a propensity to believe our senses, and if these were fully false then God would be saddling us with a mistake which we have no means of correcting and a great propensity to give our assent to these ideas. This is not the action of a good and undeceptive God. Thus the only remaining possibility is that the cause of our sensory ideas of bodies must be the bodies themselves.

\subsection{Mind-Body Distinction}

Descartes' strategy for the distinguishing the mind and the body is extremely influential, and is still used by contemporary philosophers today. In Aristotelian tradition, the mind and body were intimately related; the mind was just the form of the body. For Descartes, they are very distinct things. Descaret's argues that
\begin{enumerate}
\item The mind is essentailly thinking; one cannot separate thoughts from the mind itself.
\item The mind is not essentially extended; we don't need a physical conception of the mind to understand its function and for it to do its job.
\item Unlike the mind, the body must be extended.
\item The body is not essentailly thinking; there are many kinds of bodies which have no thoughts.
\end{enumerate}
This argument relies on something which will come to be called \emph{Leibniz's Law}.

\begin{theorem}[Leibniz's Law]
If $A$ and $B$ have different properties then $A$ and $B$ are not identical.
\end{theorem}