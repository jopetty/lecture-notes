% !TEX root = ../notes.tex

\section{Wednesday, January 16}

\epigraph{``''}{Michael Della Rocca}

\subsection*{Lecture Outline}
\begin{multicols}{2}
\begin{enumerate}
\item Copernicus
\item Mechanism
\item Old Science
\begin{enumerate}
\item Final Causes
\item Substantial Forms
\item Trust the senses
\end{enumerate}
\item Circular Explanations
\item Dormitive Virtues
\item Matter and motion
\item Other Qualities
\item Philosophical Implications
\item Descartes 1596--1650
\item 3 Main Aims
\begin{enumerate}
\item Existence of God
\item Mind/body distinctions
\item New Science
\end{enumerate}
\item Skepticism \& the 3 aims
\item Reasons for doubt
\item Principles
\item A piece of paper
\end{enumerate}
\end{multicols}

This was a time of great upheaval in science and reasoning. Aristotelian science was being replaced with a new mechanistic science. The 16\textsuperscript{th} century was coming out of the Renaissance, and Copernicus introduced the heliocentric model in 1543. Galileo Galilee also got in trouble with the Church and was placed under house arrest in 1632, when Descartes was 36 years old. He came out with his own heliocentric model in \emph{Le Monde}, but he chose not to publish out of fear of retribution. Other changes were going on as well; natural philosophy was embracing \emph{mechanism} as a means to explain the natural world.

\subsection*{Old Science \textsc{\&} Critiques}
\begin{enumerate}
\item \textbf{Appeals to final causes} --- God created the Earth in order to create a place where beings could worship God. Perhaps hurricanes exist to punish humans. Generally, any action which takes place \emph{must have a purpose}. Rocks are heavy and their purpose it to seek the center of the earth. Dogs bark because \emph{it is their nature to do so}. Humans are rational, and their final cause is to reason.

\item \textbf{Appeals to substantial forms} --- All rocks are heavy; all humans can reason, so that is their form; a dog's form is to bark. Generally, every \emph{substance} has a \emph{form}. A pan becomes hot when it is put on a stove because the fire has a certain quality has the form of heat and the pan acquires that form. If you paint a wall red, it acquires the form of redness from the paint. When we see the wall, our eyes and soul acquire the form of redness, so forms can be acquired without a transfer of matter.

\item \textbf{Trusting the senses} --- we perceive things as they actually are. Senses give an accurate description of reality without deception.
\end{enumerate}

Mechanists critique these ideas based on circular reasoning; they feel that saying ``the pan becomes hot becomes it acquires heat'' is a tautology. Likewise with ``walls become red when they acquire redness.'' These justifications are not illuminating. Molière mocked this by saying sleeping pills work because they have \emph{dormitive virtues}, which is circular. Without identifying \emph{how} these process work, no insight is gained. Modern philosophers reject these ideas because they don't explain anything: for something to exist, it must do explanatory work. Instead of forms, modern philosophers appealed to matter an motion: pans become hot when the kinetic energy of their molecules increases. This also explains what coldness is.

The relevant characteristics of particles are their size, shape, movement, and so on. Things seem red only because they reflect light at certain wavelengths. Color, taste, sound, smell, all of these were explained by size and shape and motion. The fundamental characteristics are all easily measured, allowing mathematics to be introduced to natural philosophy. There are many implications of this. \emph{Skepticism} was introduced (or rejuvenated) on account of the lack of trust which was placed in our senses. Another implication is that perception doesn't involve mind or soul becoming like the things it perceives. This means that the soul no longer has any characteristics (size, shape, color) while the body does, meaning that the two \emph{must} be distinct. Free will also became an acute worry on account of the rather deterministic account of the mechanistic world-view. Finally, this new science introduced new methodology to philosophy itself, axiomatizing philosophical discussions. Spinoza adopts these tools, but Descartes and Leibniz also used it. Philosophy becomes modeled on science and mathematics, adopting the same laws as these disciplines. This partly led to \emph{Naturalism} where people seek to discover the laws of man, trying to emulate Newton and other scientists.

\subsection*{Descartes}

Came from a well-off family. Mother died while young, educated thoroughly at a Jesuit school in the old, Aristotelian science. He wanted to be a mathematician (and he succeeded) and he wanted to apply mathematical methods to nature. In 1610 he had a series of dreams which inspired this quest. By the late 1620s he had begun considering metaphysics.

The \emph{Meditations} are modeled after religious meditations in which he gradually comes to certain realizations. He starts at the beginning with a common sense interpretation of the world and works to embrace a new, metaphysical view of the world. Before they were published, the \emph{Meditations} were circulated by his agent to leading philosophers and in turn Descartes published his replies. This makes the \emph{Meditations} more of a back-and-forth dialog between Descartes and many leading thinkers. Descartes wrote in Latin first, and then in French.

Descartes corresponded with Princess Elisabeth and many other philosophers through letters (the blogs and Facebook posts of the day). He died in 1650 

\subsection*{The 3 Main Aims of \emph{Meditations}}

In \emph{Meditations} Descartes set out to do the following.
\begin{enumerate}
\item Prove the Existence of God;
\item Prove that the Mind and Body are Distinct; and
\item Establish the New Science. This one was subtle and more implicit; he wanted to destroy the Aristotelian science by inculcating in people and acceptance of his principles before they realized it overturned Aristotle.
\end{enumerate}

Notably absent from this list is raising up Skepticism, which is one of the main motifs of the work.
