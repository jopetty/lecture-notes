\documentclass[code]{notes}

% Demonstration packages
\usepackage{kantlipsum}

\title{The Good Life}
\place{School of Life}
\speaker{Immanuel Kant}
%\speakeremail{immanuel.kant@yale.edu}
\scribe{Jackson Petty}
%\scribeemail{jackson.petty@yale.edu}
\term{Michælmas}
\year{2018}
\courseid{Life 101}

\begin{document}
	\section{This is a section}
	\kant[1]
	\note{You might notice that the page is asymmetric. That is so you have ample room for margin notes like this.}
	\subsection{This is a subsection}
	\kant[2-4]
	\section{This is another section}
	\kant[5-7]
	\section{Examples of Algorithms}
	Let's look at a few examples of some common, useful algorithms.
	\begin{algorithm}
	    \caption{Euclid’s algorithm}
	    \label{euclid}
	    \begin{algorithmic}[1] % The number tells where the line numbering should start
	        \Procedure{Euclid}{$a,b$} \Comment{The gcd of $a$ and $b$}
	            \State $r\gets a \bmod b$
	            \While{$r\not=0$} \Comment{We have the answer if $r$ is 0}
	                \State $a \gets b$
	                \State $b \gets r$
	                \State $r \gets a \bmod b$
	            \EndWhile\label{euclidendwhile}
	            \State \textbf{return} $b$\Comment{The gcd is $b$}
	        \EndProcedure
	    \end{algorithmic}
	\end{algorithm}
	\kant[8]
	Let's see an example of the \texttt{equation} environment in action\footnote{Footnotes are also placed in the margin!}:
	\begin{equation}
		\lim_{n\to\infty} \frac{1}{n} \sum_{k=0}^{n-1} f(T^k x) = \frac{1}{\mu(X)} \int f \dd{\mu}.
	\end{equation}
	This is the \emph{ergodic theorem}. \kant[2]

	\subsection{Types of Theorems}
	Let's explore how the various kinds of theorem-like environments are formatted.
	You can, of course, change how these are styled if you prefer. \kant[1]
	\begin{definition}[Group]
		A group is an ordered pair $(G,\star)$, where $\star$ is a binary operation on a set $G$, which satisfies the following four axioms.
		\begin{enumerate}
			\item The element $a \star b$ is in $G$ for all $a,b \in G$.
			\item For all $a,b,c \in G$, $a \star (b \star c) = (a \star b) \star c$.
			\item There exists an element $e \in G$ known as the idendity element for which $a \star e = e \star a = a$.
			\item For all $a \in G$, there exists an element $a^{-1} \in G$ such that $a \star a^{-1} = a^{-1} \star a = e$.
		\end{enumerate}
	\end{definition}
	\kant[1]
	\begin{problem}\label{prob:p-root-rat}
			If $p$ is a prime prove that there do not exist nonzero integers $a$ and $b$ such that $a^2 = pb^2$ (i.e., $\sqrt{p}$ is not a rational number).
		\end{problem}
	\begin{lemma}
		Let $a$ and $b$ be square integers. If there exists an integer $z$ such that $a = z \cdot b$, then $z$ must be square as well.
	\end{lemma}
	\begin{proof}
		Every composite integer can be written as the product of its prime factors.
		Since this decomposition is guaranteed to be unique, we know that if $a = (p_i \cdot p_2 \cdots p_n)^2$, then the prime factors of $b$ must be a subset of those listed.
		If, by associativity, we decompose $a$ into the factors of $b$, we will still be left with a square term; that is, if $b = (p_j \cdots p_j)^2$, then 
		\[ a = (p_1 \cdots p_n)^2 \longrightarrow \underbrace{(p_1 \cdots p_j)^2}_{b} \cdot \underbrace{(p_k \cdots p_n)^2}_{z}. \]
		Notice that no matter how we group the factors, we are left with a second term which is also square.
		This is a direct coneqsuence of the restriction that $b$ be square.
	\end{proof}
	\begin{proof}[Proof of~\ref{prob:p-root-rat}]
	This follows directly from the above lemma. Since $p$ is prime it cannot be square, and so there are no square integers for which $a^2 = pb^2$.
	\end{proof}
	
	\begin{problem}[Szemerédi, p. 42]
		Determine which of the following functions are well defined.
		\begin{parts}\relax
			\part\relax $f(a/b) = a$;
			\part\relax $f(a/b) = a^2/b^2$.
		\end{parts}
	\end{problem}
	\begin{solution}
		Sometimes, instead of a proof, it's more appropriate to submit a solution.
	\end{solution}
	\begin{example}[Non-analytic smooth functions]
		Consider the functions 
		\[ f(x) = \begin{cases}
						e^{1/(x+1)} & x < -1 \\
						0 & -1 \leq x \leq 1 \\
						e^{-1/(x-1)} & x > 1
		           \end{cases},\qquad g(x) = 0. \]
		Both functions are $C^\infty$ and equal one another if and only if $x \in [-1,1]$.
	\end{example}
	In addition to the more typical theorem-like environments, \texttt{notes} also defines some more unusual environments which may be more or less useful to you depending on your classes.
	\begin{notation}[$\times$]
		The \emph{cartesian product} of two sets. For two sets $A$ and $B$, let $A \times B$ be defined as $A \times B \coloneqq \{(a,b) \mid a \in A \land b \in B\}$.
	\end{notation}
	Along with the \emph{notation} environment, we provide the closely-related \emph{abuse of notation} environment.
	\begin{abuse}
		We define $\sin^{-1}(x)$ to be the inverse $\sin$ function, and $\sin^2(x)$ to mean $(\sin(x))^2$. What does $\sin^{-2}(x)$ mean? Who knows.
	\end{abuse}

	\begin{lemma}[Sperner's]
		This is a statement of Sperner's lemma.
	\end{lemma}
\end{document}