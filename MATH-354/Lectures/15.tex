% !TEX root = ../notes.tex

\section{Monday, 4 March 2019}

It's been a while.

Let $p$ be prime. Recall the following definitions:
\begin{itemize}
\item $\Pi(x) = \sum_{p \leq x} 1$;
\item $\vartheta(x) = \sum_{p \leq x} \log p$;
\item $\psi(x) = \sum_{p^n \leq x} \log p = \sum_{n \leq x} \Lambda(x)$.
\end{itemize}

We had two important results from these definitions.

\begin{theorem}
$\abs{\vartheta(x) - \psi(x)} \leq O(\sqrt{x}\log^2x)$.
\end{theorem}

\begin{theorem}
$\Pi(x) = \vartheta(x)/\log x - \int_1^x \vartheta(x)/t\log^2x \dd{t}$.
\end{theorem}

\begin{theorem}
The following are equivalent:
\begin{itemize}
\item $\psi(x) \sim x$,
\item $\vartheta(x) \sim x$,
\item $\Pi(x) \sim x/\log x$.
\end{itemize}
\end{theorem}

\begin{lemma}
$M_{\log x} = \sum_{n \leq x} \log n = x\log x - x + O(\log x)$.
\end{lemma}

\begin{proof}
Consider 
\[ \sum_{n \leq x} \log n \cdot 1. \]
We apply Abel summation using $f(n) = 1$ and $\phi(x) = \log x$ to get that 
\[ \sum_{n \leq x} \log n \cdot 1 = \lfloor x \rfloor \log x - \int_1^x \frac{\lfloor t \rfloor}{t} \dd{t}, \]
noting that $M_1 = \lfloor x \rfloor$. Notice that 
\[ \int_1^x \frac{\lfloor t \rfloor}{t} \dd{t} = \int_1^x 1 - \frac{t - \lfloor t \rfloor}{t} \dd{t} = x - 1 - O\qty(\int_1^x \frac{1}{t} \dd{t}) = x + O(\log x). \]
This acheives the result we wanted.
\end{proof}

\begin{theorem}[Chebyshev]
The inequality 
\[ x \log 2 + O(\log x) \leq \psi(x) \leq x\log 4 + O(\log^2x) \] holds.
\end{theorem}

\begin{proof}
Recall that $M_{\log(x)} = \sum_{n \leq x} \log n$, and that 
\[ \log n = \sum_{d \mid n} \Lambda(d). \]
Then 
\[ M_{\log(x)} = \sum_{n \leq x} \sum_{d \mid n} \Lambda(d) = \sum_{dq \leq x} \Lambda(d) = \sum_{q \leq x}\sum_{d \leq x/q} \Lambda(d) = \sum_{q \leq x} \psi\qty(\frac{x}{q}). \]
Define the quantity
\[ D(x) = M_{\log}(x) - 2M_{\log}\qty(\frac{x}{2}) = \log {x \choose x/2}. \]
On one had, we can show that 
\[ D(x) = x\log(2) + O(\log x), \]
and on the other had we know that 
\[ D(x) = \sum \psi(x/q) - 2\sum\psi(x/2q) = \psi(x) - \psi(x/2) + \psi(x/3) + \cdots. \]
Note that $\psi(x)$ is ``sorta monotone increasing''
\end{proof}