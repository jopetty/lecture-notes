% !TEX root = ../notes.tex

\section{Wednesday, January 30}

\subsection{Pollard-\texorpdfstring{$\rho$}{rho} Factorization}

We want to factor $n$. If we can find $0 < a,b <n$ such that $a \equiv b \pmod{p}$ then $(b-a,n)=p$ is a nontrivial factor of $n$. Our first idea was to try numbers at random, and after about $\sqrt{p}$ samplings we'll find two $a,b$ which are congruent mod $p$. But since there were ${\sqrt{p} \choose 2}$ pairs so it takes about $p\log p$ steps.

New idea: Start at $x_0 = 2$. For $i \geq 1$, let $x_{i+1} = x_i^2 + 1 \pmod{n}$. This will replace our random numbers, and the hope is that this sequence $x_1, x_2, \dotsc$ is ``random enough'' for our uses. Now if $x_j \equiv x_i \pmod{p}$ for any $p$ dividing $n$ then $x_{j+1} = x_j^2 + 1 \pmod{n} \equiv x_j^2 + 1 \pmod{p} \equiv x_i^2 + 1 \pmod{p} \equiv x_{i+1} \pmod{p}$. This forms a nice cycle modulo $p$. 

\subsection{Floyd's Cycle Finding}

Given a sequence $a_1, a_2,\dotsc$ which is eventually periodic (where repeats indicate multiples of the period), how do we find the period of the sequence? That is, we have no guarantee that $a_1$ will reappear, but we know that the sequence will eventually have a repeating part. Ideas:
\begin{enumerate}
\item Pick pairs at random. Really inefficient.
\item Fix $a_1$ and check all the others in the list until you find a match. But we have no idea when the cycle starts.
\end{enumerate}
Instead we use a ``tortoise adn the hare method,'' where we have two pointers in our sequence. The slow pointer $t$ moves through the list while the fast pointer $h$ moves twice as fast as the tortoise. Eventually both of the following will happen:
\begin{enumerate}
\item $t_i = x_i$ in the sequnce, and 
\item the length of the cycle $\ell$ divides $i$ 
\end{enumerate}
and then $t_i = h_i$ and we have a multiple of the cycle length.

\subsection{Algorithm}

Let $x_0 = 2$ and let $x_{i+1} = x_i^2 + 1 \pmod{n}$.
\begin{enumerate}[label = {Step~\arabic*.}]
\item Compute $(x_{2i} - x_i,n)$. If this equals $1$, continue. Otherwise it is a nontrivial factor of $n$.
\end{enumerate}

If the $x_i$ are sufficiently random, then with high probability there are two $j,k \leq O(\sqrt{p})$ such that $x_j \equiv x_k \pmod{p}$ where $p$ is any divisor of $n$. After $O(\sqrt{p})$ steps we will have
\begin{itemize}
\item $k-j$ divides $i$;
\item $i \geq \min(j,k);$
\end{itemize}
These imply that $x_i$ is in the cycle and that $x_{2i} \equiv x_i \pmod{p}$. Then $x_i \equiv x_{i + (k-j)} \pmod{p}$. Then $(x_{2i}-x_i,n)$ is at least $p$, a nontrivial factor.

Pollard-$\rho$ runs in about $O(n^{1/4}\log n)$, which is $O(n^{1/4})$ computations of the \textsc{gcd}, and it runs especially quickly in the case that $n$ has small prime factors since those determine the cycle length.

\subsection{Some cool things}

\begin{theorem}
If $p$ is prime then $(p-1)! \equiv -1 \pmod{p}$.
\end{theorem}

\begin{proof}
Every number can be paired with its mutliplicative inverse in $\Z/p\Z$. Then $(p-1)! = \prod a = \prod_{(a,a^{-1})}(a \cdot a^{-1}) \cdot -1 = -1$ (this double counts when $a = a^{-1}$, so when $a=\pm 1$).
\end{proof}

\begin{theorem}[Fermat's Little Theorem]
For prime $p$, $x^p \equiv x \pmod{p}$ for any $x \in \Z$.
\end{theorem}

\begin{proof}
Recall $\varphi(n)$ is the number if numbers less than $n$ which are relatively prime to $n$, and let $\Z/m\Z^\times$ is the set of units in $\Z/m\Z$, which is the set of numbers relatively prime to $m$ equipped with multiplication modulo $m$.
\end{proof}