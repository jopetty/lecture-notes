% !TEX root = ../notes.tex

\section{Monday, 25 March 2019}

Recall from last lecture (I don't) that we have a theorem:

\begin{theorem*}
	There are infinitely many primes $p$ satisfying $p \cong 3 \pmod{4}$.
\end{theorem*}

\begin{lemma*}
	Let $q$ be prime. There are infinitely many primes $p \cong 1 \pmod{q}$.
\end{lemma*}

\begin{proof}
	Look at $\Phi_q(x) = \frac{x^q-1}{x-1} = x^{q-1} + \cdots + 1$.
	We know that if $p \mid \Phi_q(x)$ then $p \cong 1 \pmod{q}$.
	Note that $p \mid x^q - 1$ or $x \cong 1 \pmod{p}$.
	In the former case, we know that the order of $x$ is $q$, so $q \mid p-1$, 
	which implies that $p \cong 1 \pmod{q}$.
	In the latter case, we would get that $p \mid q$ which is really problematic 
	since $p$ and $q$ are distinct primes.

	To prove that there are infinitely many such primes, suppose we have some 
	finite list of primes $p_1, \dotsc, p_\ell$.
	Notice that $\Phi_q(x) \cong 1 \pmod{x}$ for all $x$, so let $x$ be
	the product of these finite primes. Then any prime which divides $\prod p_i$
	must itself be congruent to $1$ modulo $q$, and since these prime factors
	must exist we know that there exist infinitely many primes which are 
	congruent to $1$ modulo any prime $q$.
\end{proof}

\subsection*{Outline of Dirichlet's Proof}

\begin{theorem}
	For any $q$ and any $(a,q) = 1$ there are infinitely many primes of the form
	$p = a + kq$.
\end{theorem}

First, some definitions:

\begin{definition}[Character]
	A character is a homomorphism from an Abelian group $G$ to the complex
	numbers $\C$.
\end{definition}

\begin{example}[Example of a character]
	Let $G \isom (\Z/m\Z, +)$.
	There are $m$ characters of the form 
	\[ \psi_a(x) = \zeta^{ax}, \]
	where $\zeta$ is the $m$\textsuperscript{th} root of unity.
	One may trivially check that this is, in fact, a homomorphism.
\end{example}

A list of facts:

\begin{enumerate}
	\item The characters of $G$ themselves form an Abelian group $\widehat{G}$, 
	and there are exactly $\abs{G}$ of them.
	\item Orthogonality relations.
	\begin{itemize}
		\item \[ \sum_{a \in G} \psi(g) = \begin{cases}
			\abs{G} & \text{if $\psi \cong 1$,} \\
			0 & \text{otherwise}.
		\end{cases} \]
		\item \[ \sum_{\psi \in \widehat{G}} \psi(g) = \begin{cases}
			\abs{G} & \text{if $g = 0$,} \\
			0 & \text{otherwise}.
		\end{cases} \]
	\end{itemize}
\end{enumerate}

Let $\chi$ be some character of the group $(\Z/q\Z)^\times$.

\begin{definition}[Dirichlet Character]
	The Dirichlet character is a map $\chi\colon \N \to \C$ which extends
	a regular character by saying that
	\[ \chi(m) = \chi(\bar{m}) \quad \text{if $m \cong \bar{m} \pmod{q}$}, \]
	and that $\chi(m) = 0$ if $(m,q) \not= 1$.
\end{definition}

\begin{example} Let $q$ = 5.
	\begin{itemize}
		\item The trivial character: 
		\[ \chi_o = 1,1,1,1,0,1,1,1,1,0,\dotsc \]
		\item The Legendre symbol
		\[ \qty(\frac{m}{5}) = \begin{cases}
			-1 & \text{$m$ non residue,} \\
			1 & \text{$m$ residue,} \\
			0 & \text{$m \cong 0 \pmod{5}$.}
		\end{cases} \]
	\end{itemize}
\end{example}

The character is a completely multiplicative function.

\begin{definition}[$L$-function]
	Given a Dirichlet character $\chi$, we define the Dirichlet $L$-function 
	$L(s,\chi)\colon \text{``$\C$''} \to \C$ by 
	\[ 
		L(s,\chi) = \sum_{n=1}^\infty \frac{\chi(n)}{n^s},
	\]
	which is just the Dirichlet series of $\chi$.
\end{definition}

\begin{example}
	Let $\chi = \chi_0$ and let $q$ be prime.
	Then $L(s, \chi_0) = \sum 1/n^s$ where $n \not\cong 0 \pmod{q}$.
\end{example}

\begin{remark*}
	Why care about these characters? Well, they are really good at picking out
	numbers which are $1$ modulo $q$.
\end{remark*}

\begin{theorem*}
	There are infinitely numbers $x \cong a \pmod{q}$.
	Yeah, this is easy.
\end{theorem*}

\begin{proof}
	Let's look at $\psi$ characters of $(\Z/m\Z, +)$.
	Extend to Dirichlet character $\psi\colon \N \to \C$.
	Look at 
	\begin{align*}
		\sum_{\psi} L(s,\psi) &= \sum_\psi \sum_{n=1}^\infty \frac{\psi(n)}{n^s} \\
							&= \sum_{n=1}^\infty \frac{1}{n^s}\sum_\psi \psi(n) \\
							  &= m \sum_{n =km } \frac{1}{n^s}.
	\end{align*}
	If there were finitely many $n \cong 0 \pmod{m}$, then the right hand side
	would need to be finite.
	However, we can show rather easily that
	\[ \lim_{s \to 1^+} \sum_\psi L(s,\psi) = \infty, \]
	and so there must be infinitely many $n$.
	The obstacles we'll face are 
	\begin{itemize}
		\item Actually analyze the above sum;
		\item We want to sum over only primes which are congruent to $1$ modulo
		$q$;
		\item We want to restrict our attention to things that are only 
		relatively prime to $q$.
	\end{itemize}
\end{proof}

\subsubsection*{Convergence of \texorpdfstring{$L$}{L}-functions}

Let $\chi$ be a character of $G \isom (\Z/q\Z)^\times$.
Assume that $\chi \not= \chi_0$.
We know that 
\[ 
	\sum_{x\in G} \chi(x) = 0 \implies \sum_{x=0}^{q-1} \chi(x) = 0, \tag{by orthogonality}
\]
so consider 
\[ 
	\sum_{n \leq x} \chi(n) = \sum_{n \leq kq} \chi(n) + \sum_{kq+1}^x \chi(n)
	\leq \varphi(q),	
\]
where the first term must be $0$, and in the latter, there are at most 
$\varphi(q)$ summands which are relatively prime to $q$.
This implies that the $L$-functions always converge.

\begin{problem}[Homework]
	Let $f(n)$ be a monotonically decreasing positive function. 
	Then $\sum_{n =M}^N f(n)\chi(n) \leq 3\varphi(q)f(M)$.
	\textbf{Hint:} We know that $\chi$ is a periodic function.
	Use summation by parts.
\end{problem}

\begin{corollary}
	If $\chi$ is not $\chi_0$ and $s > 0$ then 
	\[ L(s,\chi) = \sum \frac{\chi(n)}{n^s} \]
	converges.
\end{corollary}

\begin{proof}
	Use the homework problem to get that 
	\[ \abs{L(s,\chi) - \sum_{n \leq x} \frac{\chi(n)}{n^s}} \leq \frac{3\varphi(q)}{n^s}, \]
	which goes to zero.
\end{proof}

Since this function converges, we get an Euler product
\begin{align*}
	L(s, \chi) &= \prod_{\text{$p$ prime}} \qty(1 + \frac{\chi(p)}{p^2} + \frac{\chi(p^2)}{p^{2s}} + \cdots) \\
	&= \prod_{\text{$p$ prime}} \qty(\frac{1}{1-\chi(p)/p^s}).
\end{align*}

In particular, 
\[ L(s,\chi_0) = \zeta(s)\prod_{p \mid q}\qty(1-\frac{1}{p^s}). \]