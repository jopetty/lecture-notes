% !TEX root = ../notes.tex

\section{Wednesday, 23 February}

Recall the question of when is $(a/p) = \pm1$ when $a\in\Z$, $(a,p) = 1$, and $p$ is prime. We defined $\abs{x} = \min\{x, p-x\}$ for $0 \leq x \leq p-1$. Recall Gauss' Lemma, which states

\begin{lemma}[Gauss]
Let $s$ be the number of $\ell$ such that $1 \leq \ell \leq (p-1)/2$ such that $\abs{a\ell} = -a\ell$. Then $(a/p) = (-1)^s$.
\end{lemma}

\begin{example}
Let $p = 17$ and let $a = 2$. Then $\{a\ell\}$ is $\{2, 4, 6, 8, 10, 12, 14, 16\}$. We see that $(p+1)/2 = 9$. Then $s=4$ and so $(2/p) = 1$.
\end{example}

\begin{theorem}
Let $p$ be an odd prime. Then 
\[
	\qty(\frac{2}{p}) = \begin{cases}
		1 & p \cong_8 \pm 1, \\
		-1 & p \cong_8 \pm 3.
	\end{cases}
\]
\end{theorem}

\begin{proof}
We just need to look at how many even numbers are between $1$ and $(p-1)/2$ verses $(p+1)/2$ and $p$.
\begin{enumerate}
\item[\textbf{Case 1:}] $p = 1 + 8k$. Then $(p-1)/2 = 4k$, so $s = 2k$ and $(2/p) = (-1)^{2k} = 1$.
\item[\textbf{Case 3:}] $p = 3 + 8k$. Then $(p-1)/2 = 4k + 1$, so there are $2k$ even numbers ebtween $1$ and $(p-1)/2$ and $2k+1$ even numbers between $(p+1)/2$ and $p$, so $(2/p) = -1$. \qedhere
\end{enumerate}
\end{proof}

What about $(p/q)$ when $p$ and $q$ are odd primes.

\begin{theorem}[Quadratic Reciprocity]
If $p \cong_4 1$ or $q \cong_4 1$ then $(p/q) = (q/p)$. Otherwise, if $p \cong_4 q \cong_4 3$ then $(p/q) = -(q/p)$. Equivalently,
\[ \qty(\frac{p}{q})\qty(\frac{q}{p}) = (-1)^{(p-1)/2 \cdot (q-1)/2} \]
\end{theorem}

\begin{proof}
Consider $f(z) = 2i\sin(2\pi z)$, which has some nice properties. It's odd, so $-f(z) = f(-z)$. It's also $1$-periodic, so $f(z) = f(z+1)$. Note that $i \sin(z) = \sinh(iz)$, so 
\[ f(z) = e^{2\pi i z} - e^{-2\pi i z}. \]
Define $\zeta = \zeta_p$ to be the $p$\textsuperscript{th} root of unity $2^{2\pi i/p}$. Note that $\zeta^m \cdot \zeta^n = \zeta^{m+n \mod p}$ and $\qty(\zeta^m)^\ell = \zeta^{m\ell \mod p}$

We also have the following Proposition and Lemma, listed after the proof.

Now let $p,q$ be odd primes. Then 
\[ \qty(\frac{q}{p}) = \prod_{\ell=1}^{(p-1)/2} \frac{f(q\ell/p)}{f(\ell/p)} = \prod_{\ell=1}^{(p-1)/2}\prod_k^{(q-1)/2} f\qty(\frac{\ell}{p} + \frac{k}{q})f\qty(\frac{\ell}{p} - \frac{k}{q}). \]
Notice how this expression is almost symmetric in $p$ and $q$, with only one difference in the final term. In fact, switching them out only requires $(-1)^{(p-1)(q-1)/2}$.
\end{proof}

\begin{proposition}\label{prop:periodicity}
Consider that
\[ \prod_{\ell = 1}^{(p-1)/2} f\qty(\frac{a\ell}{p}) = \qty(\frac{a}{p}) \cdot  \prod_{\ell = 1}^{(p-1)/2} f\qty(\frac{\ell}{p}).\]
The justification comes from the periodicity of $f$.
\end{proposition}

\begin{proof}
Consider that if $1/2 < a\ell/p < 1$ then 
\[ f\qty(\frac{a\ell}{p}) = -f\qty(\frac{\abs{a\ell}}{p}). \]
Then 
\[ \prod_{\ell = 1}^{(p-1)/2} f\qty(\frac{a\ell}{p}) = (-1)^s \prod_{\ell = 1}^{(p-1)/2} f\qty(\frac{\abs{a\ell}}{p}) = \qty(\frac{a}{p})\prod_{\ell = 1}^{(p-1)/2} f\qty(\frac{\abs{a\ell}}{p}). \]
Then recall that the sequence $\{\abs{a}, \abs{2a}, \dotsc, \abs{(p-1)a/2}\}$ is just $\{1, 2, \dotsc, (p-1)/2\}$, which gets us that 
\[ \qty(\frac{a}{p})\prod_{\ell = 1}^{(p-1)/2} f\qty(\frac{\abs{a\ell}}{p}) = \qty(\frac{a}{p})\prod_{\ell = 1}^{(p-1)/2} f\qty(\frac{\ell}{p}). \qedhere \]
\end{proof}

\begin{lemma}
If $n$ is odd then \[
x^n-y^n = \prod_{k=0}^n \qty(x\zeta^k - y\zeta^{-k}),
\]
where $\zeta = \zeta_n$.
\end{lemma}
\begin{proof}
It suffices to look at $(x/y)^n - 1 = z^n - 1$, which factors as 
\[ z^n - 1 = \prod_{k=0}^{n-1} \qty(z-\zeta^k) = \prod_{k=0}^{n-1} \qty(z-\zeta^{-2k}). \]
Then 
\[ x^n - y^n = \prod \qty(x-y\zeta^{-2k}) = \qty[\prod \qty(x\zeta^k - y\zeta^{-k})] \zeta^{-(n-1)/2}, \]
where $\zeta^{-(n-1)/2} = 1$ since $n$ is odd.
\end{proof}

\begin{lemma}
The value
\[ \frac{f(nz)}{f(z)} = \prod_{k=1}^{(n-1)/2} f\qty(z + \frac{k}{n})f\qty(z - \frac{k}{n}) \]
if $n$ is odd.
\end{lemma}
\begin{proof}
Notice that $f(nz) = e^{2\pi i zn} - e^{-2\pi i zn}$. Now just apply the previous lemma.
\end{proof}

\subsection{Applications of QR}
\pagebreak
\begin{example}[We can use it to compute Legendre Symbols!]
Compute
\[ \qty(\frac{713}{1009}). \]
We factor as 
\[ \qty(\frac{23 \cdot 31}{1009}) = \qty(\frac{23}{1009})\qty(\frac{31}{1009}) = \qty(\frac{1009}{23})\qty(\frac{1009}{31}) = \qty(\frac{20}{23})\qty(\frac{17}{31}). \]
\end{example}

\begin{definition}[Jacobi Symbol]
Let $(q/n)$ be the Jacobi symbol, where $n$ is a product of primes, then it is multiplicative.
\[
\qty(\frac{q}{n}) = \begin{cases}
0 & (a,n) \not= 1, \\
\prod \qty(\frac{a}{p_i})^{1/i} & (a,n) = 1.
\end{cases}
\]
\end{definition}

\begin{theorem}[Jacobi Reciprocity]
Some facts:
\begin{itemize}
\item $(-1/n) = (-1)^{(n-1)/2}$
\item $(2/n) = (-1)^{(n^2-1)/8}$
\item If $m,n$ odd then $(m/n)(n/m) = (-1)^{(m-1)(n-1)/4}$
\end{itemize}
\end{theorem}

\subsubsection*{Next Class}
\begin{itemize}
\item Use Jacobi to talk about when $a$ is a quadratic residue for almost all primes
\item RSA/Diffie-Hellman, Zero Knowledge Proofs
\end{itemize}