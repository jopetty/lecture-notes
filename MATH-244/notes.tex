\documentclass{notes}

% Biblography
\usepackage{biblatex}
\addbibresource{references.bib}

\title{Discrete Mathematics}
\courseid{MATH 244}
\place{Yale University}
\term{Fall}
\year{2018}

\blurb{
	These are lecture notes for MATH 244a, ``Discrete Mathematics,'' taught by Ross Berkowitz at Yale University during the fall of 2018.
	These notes are not official, and have not been proofread by the instructor for the course.
	These notes live in my lecture notes respository at 
	\[\text{\url{https://github.com/jopetty/lecture-notes/tree/master/MATH-244}.}\] 
	If you find any errors, please open a bug report describing the error, and label it with the course identifier, or open a pull request so I can correct it.
}

\begin{document}

\section*{Syllabus}

\begin{center}
\begin{tabular}{@{}rp{10cm}@{}}
\toprule
\textbf{Instructor} & Ross Berkowitz, \url{ross.berkowitz@yale.edu} \\
\textbf{Lecture} & 11:35 \textsc{am} -- 12:35 \textsc{pm}, WLH 201 \\
\textbf{Textbook} & \fullcite{JJ} \\
\textbf{Final} & Sunday, December 16, 2018, 2:00 \textsc{pm} \\
\bottomrule
\end{tabular} \\[3ex]
\end{center}

Discrete  math  is  the  study  of  discrete,  and  frequently  finite,  mathematical  structures.   It  is  a broad subject, and is perhaps best understood by seeing the problems it considers.  In this course the three main topics we will cover are enumerative combinatorics, graph theory, and probability. Enumerative combinatorics is,  narrowly speaking,  the art of counting and estimating the size of various structures, sequences, or sets.  Graphs encode information about pairwise relations between objects (consider a network of people with a connecting line between them if they are friends), and their properties form a rich subject of study.  Finally, probability theory, the study of chances, both deals with finite objects directly, and is a surprisingly useful tool for studying discrete structures of all stripes. Just  as  important  as  the  material  itself,  this  course  will  serve  as  a  primer  in  mathematical thinking. We will see important problem solving methods and learn how to make rigorous arguments by seeing and by doing. We will break the course down into five sections,
\begin{itemize}
\item \textbf{Preperatory Material:}  Mathematical notation, functions, induction, proofs;
\item \textbf{Enumeration:}  The binomial theorem, permutations, stars and bars, estimates, inclusion exclusion;
\item \textbf{Graphs:} Eulerian graphs, connectivity, Turan’s theorem, trees, planar graphs, graph coloring;
\item \textbf{Generating Functions:}  Recurrence relations, rational generating functions, algebraic manipulation;
\item \textbf{Probability:}  linearity of expectation, first moment method.
\end{itemize}

\printbibliography

% !TEX root = ../notes.tex

\section{January 14, 2019}

Given an interval $(a,b) \subset \R$, we know that the size of this interval is $b-a$. The focus of this course will be the study of the generalization of this idea using the \emph{Lebesgue measure} on $\R$. Equipped with this, we can talk of the \emph{Lebesgue integral} of ``nice'' functions, which is more powerful than the Riemannian equivalent.

\subsection{The Metric Space}

\begin{definition}[Metric Space]
Given a set $X$, a metric function $d$ is a function $d : X \times X \to \R$ obeying the following three properties.
\begin{enumerate}
\item \textbf{Positivity:} $d(x,y) \geq 0$ and $d(x,y) = 0$ if and only if $x = y$;
\item \textbf{Symmetry:} $d(x,y) = d(y,x)$ for all $x,y \in X$;
\item \textbf{Triangle Inequality:} $d(x,y) \leq d(x,z) + d(z,y)$ for all $x,y,z \in X$. 
\end{enumerate}
A metric space is a pair $(X,d)$ where $d$ is a metric function on $X$.
\end{definition}

\begin{example}[Metric Spaces]
\begin{parts}
\part In $\R$, we have the traditional $d(x,y) = \abs{x-y}$.
\part In $\R^2$, we have $d\qty((x_1,x_2),(y_1,y_2)) = \sqrt{(x_1-y_1)^2 + (x_2-y_2)^2}$.
\part In $\R^2$, we also have $d\qty((x_1,x_2),(y_1,y_2)) = \max\{\abs{x_1-y_2}, \abs{x_2-y_2}\}$.
\part The discrete metric on a set $X$ is defined by 
\[ d(x,y) = \begin{cases}
0 & \text{if $x = y$,} \\
1 & \text{otherwise.}
\end{cases} \]
\part Given a metric space $(X,d)$ and $Y \subset X$ then $(Y,d)$ is also a metric space where $d$ is restricted to $Y \times Y$.
\end{parts}
\end{example}

\begin{definition}[Neighborhood]
Fix a metric space $(X,d)$. For some $r \geq 0$, the $r$-neighborhood of $x$ is $B(x,r)$, the set $\{y \in X \mid d(x,y) < r\}$. Notice that this depends on the metric! In $\R$ with the discrete metric, $B(0,1) = \{0\}$ while $B(0,2) = \R$ which is not what we expect from the traditional metric.
\end{definition}

\begin{definition}[Interior Points]
Let $A \subset X$. A point $x \in A$ is an interior point of $A$ if there exists some $r > 0$ such that $B(x,r) \subset A$. That is, we can draw a ball around $x$ which lies entirely in $A$.
\end{definition}

\begin{example}
If $A = [0,1)$, then the interior points of $A$ are $(0,1)$ but $0$ is not an interior point.
\end{example}

\begin{definition}[Open Sets]
A subset $A \subset X$ is open if every point in $A$ is interior. The empty set is vacuously open.
\end{definition}

\begin{proposition}
For any $x \in X$ the $r$-neighborhood of $x$ is an open subset of $X$.
\end{proposition}

\begin{proof}
Let $y \in B(x,r)$. Let $r_0 = r - d(x,y)$. Then $r_0 > 0$ and $B(y,r_0) \subset B(x,r)$ regardless of which $y$ is chosen since for any $z \in B(y,r_0)$ we know that $d(x,z) \leq d(x,y) + d(r,z) < d(x,y) + r_0 < r$. Then every point of $B(x,r)$ is interior and so it is open.
\end{proof}

Using this, we can now call $B(x,r)$ the open ball of radius $r$ centered at $x$.

\begin{example}
In $\R^2$ with the standard metric, an open ball looks like an open disc. With the maximum metric, it looks like an open square. In $\R$, we can look at the set of all rational numbers $\Q$. This set is not open since for all $q \in \Q$ and all $r > 0$ there exists an $x \in B(q,r)$ where $x \notin \Q$.
\end{example}

\begin{proposition}\label{prop:1-2}
The intersection of finitely many open sets is open. The union of any open sets is open.
\end{proposition}

\begin{example}
The intersection of infinitely many open sets is not necessarily open. Consider $\bigcap\, (0,1/n)$ as $n \to \infty$. The intersection is simply $\{0\}$ which is not an open set.
\end{example}

\begin{proof}[Proof of Proposition~\ref{prop:1-2}]
Let $A_1, \cdots, A_k$ be open subsets of $X$. Let $x \in A_1 \cap \cdots \cap A_k$. Since each $A_i$ is open we know that $x$ is an interior point of $A_i$, so there exists some $r_i$ such that $B(x,r_i) \subset A_i$. Let $r$ be the minimum of all such $r_i$. Then $B(x,r) \subset A_i$ for all $i$, so this open ball is contained in the intersection.

Now let $\{A_\alpha \mid \alpha \in I \}$ be a collection of open subsets. Let $x \in \bigcup\,A_\alpha$. Then $x$ is contained in some open $A_\alpha$, and so there exists an $r_\alpha$ such that $B(x,r_\alpha) \subset A_\alpha$, so $B(x,r_\alpha) \subset \bigcup\,A_\alpha$.
\end{proof}

\begin{definition}[Interior of a Set]
For $A \subset X$, the set of all interior points of $A$ is called the interior of $A$. This is usually written as $\operatorname{Int}(A)$ or $A^\circ$.
\end{definition}

\begin{example}
If $A = [a,b]$ then $A^\circ = (a,b)$. If $A = \Q$ then $\Q^\circ = \emptyset$.
\end{example}

\begin{proposition}
For all $A$, the interior of $A$ is open. Furthermore, $A^\circ$ is the largest open subset of $A$ in the sense that it contains all other open subsets of $A$.
\end{proposition}

\begin{proof}
It's just the definitions.
\end{proof}

\begin{proposition}
If $A \subset B$ then $A^\circ \subset B^\circ$.
\end{proposition}

\begin{corollary}
A set $A$ is open if and only if $A = A^\circ$.
\end{corollary}

\begin{definition}[Limit Point]
Let $A \subset X$. A point $x \in X$ is a limit point of $A$ if for any $r > 0$ we know that $B(x,r) \cap A \not= \emptyset$. Notice that every point $a \in A$ is a limit point of $A$.
\end{definition}

\begin{example}
Let $A = [0,1)$. Then $0$ is a limit point of $A$ since every open ball centered centered at $0$ interesects $A$. Furthermore, $1$ is also a limit point for the same reason. If $A = \Q$, then the set of limit points of $\Q$ is all of $\R$.
\end{example}

\begin{definition}[Closed Set]
A set $A \subset X$ is called closed if every limit point of $A$ is contained in $A$.
\end{definition}

\begin{example}
The interval $[0,1]$ is closed but $[0,1)$ is not. To show that something isn't a limit point, use the minimum distance between this point and the interval. This must be positive since otherwise it would be in the interval. Then let your $r$ be smaller than this, and the open ball with this radius centered at this point will not intersect the original interval. Generalize to higher dimensions as needed.
\end{example}

\begin{corollary}
Given any metric space $X$, we know that $\emptyset$ is closed. Furthermore, $\bar{B}(y,r) = B[y,r] = \{y \mid d(x,y) \leq r \}$ is closed for any $r$.
\end{corollary}

\begin{proposition}
Let $A \subset X$. We know that $A$ is open if and only if $A^\complement$ is closed.
\end{proposition}

\end{document}