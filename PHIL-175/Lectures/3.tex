% !TEX root = ../notes.tex

\section{Tuesday, January 22}

\epigraph{``If you planted a bomb in a school yard, I guess that would be a pretty bad thing''}{Shelly}

A lot of what drives moral philosophy is ``why be moral,'' but to answer this we first must have an idea of what morality actually is. Our first framework will be \emph{Utilitarianism}, popularized by Mill and Bentham, which will we will build up to. Shelly made a note that although we find objections to this or any view, this doesn't mean that they aren't worth considering, and since we can find objections about any view there isn't a ``perfect'' answer to these questions. Our goal is to understand the flaws and strengths of the arguments and gain a better understanding of morality and ethics through this.

Mill does not own Utilitarianism, he merely wrote arguments to support it. Mill's words on the matter are not God's words on it (the same holds for Kant and \emph{Kantianism}), so it's important not to take this perspective.

Philosophy isn't like physics; this isn't a ``learn it or fail'' kind of course. You will need to decide for youself what the right answers are and justify why you think that. There aren't any \emph{knock-down} arguments in Philosophy but this doesn't mean that there aren't right answers. To compare to Theology, there are disagreements about whether or not god exists but that doesn't mean there is a truth value to that statement. Shelly wants us to know that he has bias himself and so do we, so it's important to be aware of this. He'll try to keep his cards close to his chest, so to speak.

\begin{problem}
If we believe something but there isn't a set of arguments which will pursuade any rational person of it, does that mean that our belief comes from a source outside of reason?
\end{problem}

Shelly used the analogy of a jury deliberating to illustrate how there can be rational conflicts.

\subsection{Why have morality?}

There are many possible answers. One possible answer is that different actions have different results, and we are not indifferent to the results of actions. We care about what happens in the world. This makes possible the thought that one should consider the results of one's actions; in a moral sense, we should consider whether the results are good or bad when deciding what one ought to do. Shelly calls this idea \emph{Resultism}.

\begin{definition}[Resultism]
The results of an action are morally relevant to determine the moral value of the action.
\end{definition}

The modest statement of this is that the results are \emph{one of the things} which are relevent to this question; the strong statement says that results are the \emph{only} things one should consider. This \textbf{bold resultism} is an agreement in Utilitarianism. Pretty much every moral theory has the modest version at least.

\begin{example}[Weak Resultism]
If you read the paper and saw that many people died, almost everyone would agree that the results of that earthquake were bad (having been given no other information).
\end{example}

In order to use the results as a basis for morality, we need some way to measure the goodness or badness of results. This forms what's called \textbf{a theory of the good}. This lets us rank outcomes from better to worse. Here are some elements we may want to include in such a theory (intuitively plausible ideas, acceptible \emph{prima facie}):
\begin{itemize}
\item What are the short term results?
\item What are the long term results?
\item What are the effects on \emph{everybody}? (constasts with the idea that morality should only consider how one's actions affect oneself, known as \emph{ethical egoism})
\end{itemize}

\begin{example}[Qualities we should care about?]
\begin{enumerate}
\item Is there a difference between planting a bomb in a school yard which will go off today, tomorrow, next year, in 100 years? Would any length of time be acceptible? [Long Term Results]
\item If we store nuclear waste in a cavern in Utah which won't leak for a thousand years, is that still acceptible if it kills 10,000 Utahans \emph{eventually}? [Long Term Results]
\item Can we put our nuclear waste in Kenya so that no Americans are killed? [Who is affected]
\end{enumerate}
\end{example}

The various flavors of ethical egoism:
\begin{enumerate}
\item Rational egoism: Asking ``what should I do'' as a matter of rationality; this is the in-house religion of the department of economics.
\item Ethical egoism: Should I consider how my actions affect someone else, or just consider what is good for me.
\item Psychological egoism: Our actions are entirely determined by what we think is good for ourselves individually.
\end{enumerate}
You don't need to believe one to believe the other. Bentham believed rational but not ethical.

\begin{example}
Suppose there is a lever which you can pull to get a candy bar. However, when you pull the lever 10,000 people in France are electrocuted. For an ethical egoist, this isn't a moral dilemma: you should just pull the lever.
\end{example}