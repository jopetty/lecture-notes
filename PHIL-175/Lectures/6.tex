% !TEX root = ../notes.tex

\section{Thursday, January 31}

\epigraph{``I went to the pound since Tuesday's class and bought a stray cat. Brought it home, doused it in gasoline, and lit it on fire while it shreiked and died an agonizing death....and that seems wrong.''}{Shelly}

The results of the experience machine experiment tell us that most people doen't really buy Hedonism --- there's more to well-being than just the experience. There are many different theories which propose ideas about what the missing pieces are. One alternative is that we must have the things we want, and when we accomplish this our life is better than when we don't. This theory explains why the experience machine fails; we want the real things, not merely the experience. These Preference Theories are the in-house philosophy of the department of economics.

It also seems like it ought to matter how welfare is aggregated. In some cases this is easy; if everybody is better off in world $A$ than in world $B$ its clear than $A$ presents a better existence. But it's often not so clear cut. Perhaps we ought to use the maximum happiness in the world, or perhaps we should use the average amount of happiness. There are scenarios which make either look bad. This has been going on for forty years and philosophers still don't have a consensus. These theories all require us to have interpersonal comparisons of happiness or welfare, which may not even be possible.

\begin{problem}
Do animals count in the welfare aggregation?
\end{problem}