% !TEX root = ../notes.tex

\section{Thusrday, February 7}

\epigraph{``They're graduate students in philosophy, they won't contribute \emph{anything} to society.''}{Shelly}

\subsection*{Does Utilitarianism Give Plausible Answers?}

\begin{problem}
Why is murder wrong?

\begin{solution}[Utilitarianism Response]
Because you had other options, and the option to murder someone decreased happiness/welfare.
\end{solution}
\end{problem}

\begin{problem}
Should we steal from the blind man in the subway?

\begin{solution}[Utilitarianism Response]
No.
\end{solution}
\end{problem}

\begin{problem}
We are sailing a boat when we see someone drowning; we turn to rescue them, but then we notice five people drowning somewhere else. We don't have time to save everyone. What do we do?

\begin{solution}[Utilitarianism Response]
It's better to save five than to save one.
\end{solution}

What if we tweak the scenario? What if the one was about the find the cure for cancer, and the five were all graduate students in philosophy. What about the impacts of what these people contribute to society?

\begin{solution}[Utilitarianism Response]
Those things matter. We can't say that everything is equal, as before.
\end{solution}
\end{problem}

What if we aren't sure of the outcomes? We can't know everything, after all. What should a utilitarianism (or a consequentialist) do in light of this? What if you accidentally save Hitler from drowning?

\begin{example}
There's a distinction here in what we're asking: Did you do the right thing or are you a morally good person? Utilitarianism and consequentialism are concerned with actions, and so in and of themselves do not concern themselves with whether or not people are good. We can develop a utilitarian theory of this, but it is supplemental to the core of the philosophy.
\end{example}