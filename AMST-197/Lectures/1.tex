% !TEX root = ../notes.tex

\section{January 14, 2019}

\epigraph{``This Land is Your Land''}{Sharon Jones and the Dap Kings}

The course covers a wide field of different themes, including:

\begin{multicols}{2}
\begin{itemize}
\item Buildings
\item Architects and Designs
\item Urban Development
\item City Planning
\item The Production of Space
\item Cultural Landscapes
\item Storytelling
\end{itemize}
\end{multicols}

\subsection*{Knights of Columbus Tower}

Rather stark example of Brutalism, or maybe Modernism. Looks a lot like Kline Biology Tower. The pillars are made of brick and the steel structural beams extend outside of the habitable space. The stairways, bathroom, and storage areas are moved to the pillars to keep the habitable center open. Made of poured concrete with slip forms. Built for the Knights of Columbus, a Catholic fraternal organization (like the Masons or the Elks) and an insurance organization (mostly or Catholics who were being discriminated against). This building wasn't the first occupant of the land --- much of the city was torn down to create it during periods of Urban Renewal. Much of the U.S. is still in the hangover of urban renewal. The way the buildings are design inform and are informed by the transportation infrastructure adjacent to the them. How do these buildings play into the division of neighborhoods and gentrification?

\subsection*{Michigan Central Station}
Built by the same architects as Grand Central Station, now serves as a ruin in Detroit. Ironically, financed by the money which came from the industry which destroyed trains, and then the building was sold to Ford as their center for electric car development. Built in the Beaux-Arts Classical style.

\subsection*{Themes}

\begin{enumerate}
\item \textbf{The Social Production of Architecture} --- Who besides the architect is responsible for the buildings? Are the makers homogenious? How can we enrich the stories we tell and diversify the cast of characters who tell it?
\item \textbf{Work} --- What is the function of the building? How are buildings designed for work?
\item \textbf{Dwelling} --- How are places constructed for living? How are they changed with time? How are communal dwellings different from private dwellings?
\item \textbf{Mobility and Infrastructure} --- How are buildings connected from one another? How are they accessed? How does infrastructure influence buildings and how do buildings influence infrastructure?
\item \textbf{Expression} --- How do buildings take on individual identities? How do buildings fit together to form a pattern? How do buildings reflect the scale and pattern of human life?
\end{enumerate}